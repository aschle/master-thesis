% ---------------------------------------------------
% ----- Abstract (English)
% ---------------------------------------------------
%  Freie Universität Berlin, Institute of Computer Science, Human Centered Computing. 
%
\pagestyle{empty}
\providecommand{\keywords}[1]{\textbf{\textit{Keywords---}} #1}

\subsection*{Abstract}
The BeesBook system automatically tracks individual honey bees inside a hive over their entire life and provides high-resolution data of bee movements.
This thesis focuses on designing and implementing a network pipeline to extract interaction networks from this data.
Spatial proximity is used as an indicator for interactions between bees. Social network analysis methods were applied to investigate the static and dynamic properties of the resulting social networks of honey bees on a global, intermediate and local level.
The resulting networks were characterized by a low hierarchical structure and a high density. The global structure of the colony seems to be stable over time. The local structure is highly dynamic, as bees change communities as they age. 
Communities in the honey bee network represent age groups that show a high spatial fidelity.
The findings are in line with the established state of research that the colonies' organization is shaped by the age-based task division of individuals.
The results of the analysis validate the implemented pipeline and the inferred networks.
Consequently, this work provides a foundation for future research focusing on temporal network analysis aspects.

\vspace{5mm}
\keywords{social insects, spatial proximity network, interaction network, Apis mellifera, community detection, social network analysis}

\cleardoublepage
