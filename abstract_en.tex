% ---------------------------------------------------
% ----- Abstract (English)
% ---------------------------------------------------
%  Freie Universität Berlin, Institute of Computer Science, Human Centered Computing. 
%
\pagestyle{empty}
\providecommand{\keywords}[1]{\textbf{\textit{Keywords---}} #1}

\subsection*{Abstract}
The BeesBook system automatically tracks individual honey bees inside a hive over their entire life and provides a high-resolution dataset of bee movements of a single colony.
This thesis focuses on the inference of interaction networks, by implementing a network pipeline.
Spatial proximity is using as an indicator for interactions between bees.
Social network analysis methods were applied to investigate the static and dynamic properties of the resulting social networks of honey bees on a global, intermediate and local level.
The resulting networks were characterized by a low hierarchical structure and a high density.
The global structure of the colony seems to be stable over time.
The local structure is highly dynamic, as bees change communities as they age.
Communities in the honey bee network represent age groups with a high spatial fidelity.
The findings are in line with established state of research that a colonies organization is shaped by the age-based task division of individuals.
The results of the analysis validate the implemented pipeline and inferred networks and consequently provide an excellent foundation for future work focusing more on temporal network analysis aspects.

\vspace{5mm}
\keywords{social insects, spatial proximity network, social network, interaction network, honey bee, behavioural tracking, Apis mellifera, community detection, social network analysis}

\cleardoublepage
