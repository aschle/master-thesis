% ---------------------------------------------------
% ----- Abstract (English)
% ---------------------------------------------------
%  Freie Universität Berlin, Institute of Computer Science, Human Centered Computing. 
%
\pagestyle{empty}
\providecommand{\keywords}[1]{\textbf{\textit{Keywords---}} #1}

\subsection*{Abstract}
The BeesBook system provides high-resolution data about bee movements within a single colony by automatically tracking individual honey bees inside a hive over their entire life.
This thesis focuses on the process of designing and implementing a network pipeline to extract interaction networks from this data.
Spatial proximity is used as an indicator for interactions between bees.
Social network analysis methods were applied to investigate the static and dynamic properties of the resulting social networks of honey bees on a global, intermediate and local level.
The resulting networks were characterized by a low hierarchical structure and a high density.
The global structure of the colony seems to be stable over time.
The local structure is highly dynamic, as bees change communities as they age.
Communities in the honey bee network are formed by age groups that show a high spatial fidelity.
The findings are in line with the established state of research that colonies are organized around age-based task division.
The results of the analysis validate the implemented pipeline and the inferred networks.
Consequently, this work provides an excellent foundation for future research focusing on temporal network analysis.

\vspace{5mm}
\keywords{social insects, spatial proximity network, interaction network, Apis mellifera, community detection, social network analysis}

\cleardoublepage
