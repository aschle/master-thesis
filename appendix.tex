\chapter{Appendix Stuff}
\label{ch:appendix}

\begin{table}
\centering
\caption[XXX]{\textbf{XXX} \url{https://docs.google.com/spreadsheets/d/1eKuPU-XmqwrHkS_5-TgS8UnO5O-Hwe1kyRIpareywP4/edit?usp=sharing}}
\label{tab:studies}
\vspace*{5mm}
\begin{tabular}{ccc}
	\toprule
	{}  & TODO & TODO \\
	\midrule

	x & x & x\\
	x & x & x\\
	\bottomrule
\end{tabular}
\end{table}

\begin{table}
\centering
\caption[Network measures of studies]{\textbf{Network measures of studies} \url{https://docs.google.com/spreadsheets/d/1eKuPU-XmqwrHkS_5-TgS8UnO5O-Hwe1kyRIpareywP4/edit?usp=sharing}}
\label{tab:studies-measures}
\vspace*{5mm}
\begin{tabular}{ccc}
	\toprule
	{}  & TODO & TODO \\
	\midrule

	x & x & x\\
	x & x & x\\
	\bottomrule
\end{tabular}
\end{table}

\begin{figure}[htb]
	\centering
	\includegraphics[width=1.0\textwidth]{Figures/study-measures}
	\caption[XXX]{\textbf{XXX} XXX}
	\label{fig:study-measures}
\end{figure}

\begin{figure}[htb]
	\centering
	\includegraphics[width=1.0\textwidth]{Figures/study-nwtype}
	\caption[XXX]{\textbf{XXX} XXX}
	\label{fig:study-nwtype}
\end{figure}

\begin{figure}[htb]
	\centering
	\includegraphics[width=1.0\textwidth]{Figures/study-study}
	\caption[XXX]{\textbf{XXX} XXX}
	\label{fig:study-study}
\end{figure}


\begin{figure}[htb]
	\centering
	\includegraphics[width=1.0\textwidth]{Figures/tagging_period}
	\caption[Tagging frequency]{\textbf{Tagging frequency} The bees were primarily tagged during the week. On average 48 bees were tagged each day, considering only tagging days, the average is about 91. [TODO: combine with other image or make nicer!]}
	\label{fig:tagging-period}
\end{figure}

\begin{figure}[htb]
	\centering
	\includegraphics[width=1.0\textwidth]{Figures/recording}
	\caption[Recording season with maintainance and failures]{\textbf{Recording season with maintainance and failures} \emph{Green} indicates recording went without any big interruption; \emph{Yellow} indicates maintainance work or technical failures of one or all cameras. This is calculated using the expected number of files produced by each camera per hour. [TODO, reduzieren auf eine Info pro Tag (keine stuendliche aufloesung), kombinieren mit anzahl der getaggten bienen pro tag, und welchen Zeitraum hab ich nun verwendet], ausserdem Zeit von links nach rechts!, evtl. kein Datum, sonder Tage durchnummerieren}
	\label{fig:observation-period}
\end{figure}

[TODO: Figure: Timeline - recording for each camera]

[TODO: ]