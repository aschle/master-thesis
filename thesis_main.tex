\documentclass[
		pdftex,
		a4paper,	% page format A4
		12pt,		% font size 12pt 
		DIV=calc,
		oneside,	% enable one side
		BCOR5mm,	% add additional padding
		english,	% set language to english
		toc=bibliography,	% include bibliography in TOC
		halfparskip,
		chapterprefix,
		numbers=noenddot,
		titlepage
	]
	{scrbook}   
% ----- weitere Optionen 
%draft,			% Entwurfsmodus zum Anzeigen zu leerer/voller Boxen 
%DIV=calc
%DIV12,			% Seitengröße (siehe Koma Skript Dokumentation !) 
%BCOR5mm,		% Zusätzlicher Rand auf der Innenseite 
%twoside,		% Seitenränder werden an doppelseitig angepasst 
%fleqn,			% Formeln werden linksbündig (und nicht zentriert) angezeigt 
%titlepage,		% Titel wird in einer 'titlepage' Umgebung gesetzt 
%bigheadings,	% Große Überschriften (normal, small-headings) 
%halfparskip-	% Absatz wird nicht eingerückt, dafür aber um eine halbe Zeile nach unten gerückt
%
%---------------------------------------------------
%----- Packages
%---------------------------------------------------
%

\usepackage{layouts}
\usepackage{xcolor,colortbl}
\usepackage{arydshln}
\definecolor{usethiscolorhere}{rgb}{0.92,0.92,0.92}
\usepackage[
		style=numeric,
		citestyle=numeric,
		maxalphanames=1,
		maxcitenames=2,
		backend=bibtex,
		doi=false,
		isbn=false,
		url=false
	]
	{biblatex}
\usepackage[
		font=small,
		format=plain,
		labelfont=bf,
		up,
		textfont=normal,
		justification=justified,
		singlelinecheck=false
	]
	{caption}
\usepackage{subcaption}
\usepackage{standalone}
\usepackage{lscape}
\usepackage[T1]{fontenc} 
\usepackage[utf8]{inputenc}
\usepackage[english]{babel} 
\usepackage{ae} 
\usepackage{epigraph}
\usepackage{acronym}
\usepackage[toc,page]{appendix}
\usepackage{fancyhdr} % Define simple headings 
\usepackage{xcolor}
\usepackage{url}
\usepackage{listings}
\usepackage{vmargin} % Adjust margins in a simple way
\usepackage{color}
\usepackage{colortbl}
\usepackage{amsmath}
\usepackage{amssymb}
\usepackage{csquotes}
\usepackage{wrapfig} % Paket zur Positionierung einbinden
\usepackage[pdftex]{graphicx}
\usepackage{pgfplots}  
\usepackage{smartdiagram}
\usepackage{metalogo}
\usetikzlibrary{mindmap,trees}
%\usepackage{dtklogos}
\usepackage{hyperref} % turn all your internal references into hyperlinks
%\usepackage[pdfstartview=FitH,pdftitle={<<Titel der Arbeit>>}, pdfauthor={<<Autor>>}, pdfkeywords={<<Schlüsselwörter>>}, pdfsubject={<<Titel der Arbeit>>}, colorlinks=true, linkcolor=black, citecolor=black, urlcolor=black, hypertexnames=false, bookmarksnumbered=true, bookmarksopen=true, pdfborder = {0 0 0}]{hyperref}
%
% table settings 
\usepackage{booktabs}  
\usepackage{tabularx}
\usepackage{array}   
\usepackage{float}
\usepackage{rotating}
\usepackage{longtable}
\usepackage{pdflscape}
\usepackage{multirow} %multi row
\usepackage{rotating} %for rotating table
\usepackage{color}
\usepackage{adjustbox}

\usepackage{tikz}
\usetikzlibrary{positioning}
\usetikzlibrary{shapes.geometric}
\usetikzlibrary{shapes.misc}

\usepackage{framed} 

%---------------------------------------------------
%----- Bibliography setup
%---------------------------------------------------
%
%\bibliographystyle{ieetr}  % citation style
\bibliography{references} % bib file

% the following is needed for syntax highlighting
  
\definecolor{dkgreen}{rgb}{0,0.6,0}
\definecolor{gray}{rgb}{0.5,0.5,0.5}
\definecolor{mauve}{rgb}{0.58,0,0.82}

\definecolor{git_keyword}{HTML}{000000} 
\definecolor{git_key}{HTML}{108888}
\definecolor{git_string}{HTML}{DD1144}
\definecolor{git_tag}{HTML}{121289}
\definecolor{git_attribute}{HTML}{0A8585}
 
\lstdefinelanguage{JSON}{
	keywords={false,true},
	alsoletter=0123456789.,
  	sensitive=false,
  	morecomment=[l]{//},
  	morecomment=[s]{/*}{*/},
  	morestring=[b]',
  	morestring=[b]",
  	keywordstyle=\color{git_key}\bfseries,
    commentstyle=\color{git_string},
	stringstyle=\color{git_string},
}

\lstdefinelanguage{HTML}{
    sensitive=true,
    keywords=[1]{svg, g, path, image},
    otherkeywords={<, \/>, >},   
    keywords=[2]{class, xmlns, xmlns:xlink, xlink:href, width, height, d, x, y},   
    morecomment=[l]{//},
    morecomment=[s]{/*}{*/},
    morecomment=[s]{<!}{>},
    morestring=[b]',
    morestring=[b]",    
    alsoletter={-},
    alsodigit={:},
    keywordstyle=[1]\color{git_tag}\bfseries,
   	keywordstyle=[2]\color{git_attribute},
    commentstyle=\color{git_string},
	stringstyle=\color{git_string},
}

\lstset{
   	backgroundcolor=\color{white},
   	frame=tb,
   	rulecolor=\color{darkgray},    
   	basicstyle=\footnotesize,
   	extendedchars=true,
   	showstringspaces=false,
   	showspaces=false,
   	showtabs=false,
   	numbers=none,
   	tabsize=2,
   	breaklines=true,
   	captionpos=b
}

%
%---------------------------------------------------
%----- PDF and document setup
%---------------------------------------------------
%
\setlength{\parskip}{6pt}
\hypersetup{
	pdftitle={Temporal Analysis of Honey Bee interaction Networks based on Spatial Proximity },  % please, add the title of your thesis
    pdfauthor={Alexa Schlegel},   % please, add your name
    pdfsubject={Master thesis, Institute of Computer Science, Freie Universität Berlin}, % please, select the type of this document
    pdfstartview={FitH},    % fits the width of the page to the window
    pdfnewwindow=true, 		% links in new window
    colorlinks=false,  		% false: boxed links; true: colored links
    linkcolor=red,          % color of internal links
    citecolor=green,        % color of links to bibliography
    filecolor=magenta,      % color of file links
    urlcolor=cyan           % color of external links
}
%
%---------------------------------------------------
%----- Customize header and footer\pagestyle{fancy} 
%---------------------------------------------------

\fancypagestyle{plain}{ % 'plain' page style (used for first page of chapter)
  \fancyhf{} % clear all header and footer fields
  \fancyfoot[LE,RO]{\thepage}
}

\pagestyle{fancy}

\fancyhf{}  % delete all existing header formating
\fancyhead[LE,RO]{\leftmark}  % represent the current chapter heading in uppercase
\renewcommand{\chaptermark}[1]{ % adapt the shown chapter name: show it in lower case and with chapter number 
\markboth{\thechapter.\ #1}{}}   

%\fancyhead[RO]{\rightmark}   % % represent the current section heading in uppercase 
\renewcommand{\sectionmark}[1]{% adapt the shown section name: show it in lower case and with section number 
\markboth{\thesection.\ #1}{}}

\renewcommand{\headrulewidth}{0pt} % remove lines from header
\renewcommand{\footrulewidth}{0pt} % remove lines from header

\newcommand{\tab}{\hspace*{2em}}

% Source http://tex.stackexchange.com/a/32690
\newcolumntype{R}[2]{%
    >{\adjustbox{angle=#1,lap=\width-(#2)}\bgroup}%
    l%
    <{\egroup}%
}
\newcommand*\rot{\multicolumn{1}{R{45}{1em}}}% no optional argument here, please!

\fancyfoot{} % delete all existing footer formating
\fancyfoot[LE,RO]{\thepage} 

%%%%%%%%%%%%%%%%%%%%%%%%%%%%%%%%%%%%%%%%%%%%%%%%%%%%%%%%%%%%%%%%%%%%%%%%%%%%%%
% ---- Namen der Links im Dokument
%\addto\captionsngerman{\renewcommand{\figurename}{\small{\textbf{Abb.}}}}
%\addto\captionsngerman{\renewcommand{\tablename}{Tab.}}
%\addto\captionsngerman{\captionsetup{figurewithin = section}}
%\addto\captionsngerman{\captionsetup{font=small, labelfont=bf}}

%
%---------------------------------------------------      
%----- Settings for word separation  
%---------------------------------------------------      
% Help for separation (from package babel, section 22)):
% In german package the following hints are additionally available:
% "- = an explicit hyphen sign, allowing hyphenation in the rest of the word
% "| = disable ligature at this position. (e.g., Schaf"|fell)
% "~ = for a compound word mark without a breakpoint (e.g., bergauf und "~ab)
% "= = for a compound word mark with a breakpoint, allowing hyphenation in the composing words
% "" = like "-, but producing no hyphen sign (e.g., und/""oder)
%
% Describe separation hints here:
\hyphenation{
% Pro-to-koll-in-stan-zen
% Ma-na-ge-ment  Netz-werk-ele-men-ten
% Netz-werk Netz-werk-re-ser-vie-rung
% Netz-werk-adap-ter Fein-ju-stier-ung
% Da-ten-strom-spe-zi-fi-ka-tion Pa-ket-rumpf
% Kon-troll-in-stanz
}

\usepackage{ragged2e}  % for '\RaggedRight' macro (allows hyphenation)
\newcolumntype{Y}{>{\RaggedRight\arraybackslash}X}

%%%%%%%%%%%%%%%%%%%%%%%%%%%%%%%%%%%%%%%%%%%%%%%%%%%%%%
% The content part of the document starts here! %%
%%%%%%%%%%%%%%%%%%%%%%%%%%%%%%%%%%%%%%%%%%%%%%%%%%%%%%

\begin{document}
\frontmatter 

%\pagenumbering{alph} % even though, these page numbers are not visible there are necessary to have unique page numbers 
% ---------------------------------------------------
% ----- Title page of the template
% ----- for Bachelor-, Master thesis and class papers
% ---------------------------------------------------
%  Created by C. Müller-Birn on 2012-08-17, CC-BY-SA 3.0.
%  Freie Universität Berlin, Institute of Computer Science, Human Centered Computing. 
%
\begin{titlepage}

\title{\includegraphics[width=0.6\textwidth]{Resources/FU_logo.pdf}\\
{\small Master thesis, Institute of Computer Science, Freie Universität Berlin}\\
{\small Biorobotics Lab}\\
[6ex]
{\LARGE Temporal Network Analysis\\ of Honeybee Interaction Networks\\ } \\
{\normalsize based on Spatial Proximity }}

\author{
{\emph{\normalsize Alexa Schlegel}}\\
{\normalsize Matriculation number: 4292909}\\
{\normalsize alexa.schlegel@fu-berlin.de}\\ 
[15ex]   
{\normalsize Supervisor: Prof. Dr. Tim Landgraf}\\
{\normalsize Second Supervisor: Prof. Dr. ???}\\
}
\vspace{6ex}
\date{\normalsize Berlin, \today}
 
\maketitle  

\end{titlepage}
% ---------------------------------------------------
% ----- Declaration of the template
% ----- for Bachelor-, Master thesis and class papers
% ---------------------------------------------------
%  Created by C. Müller-Birn on 2012-08-17, CC-BY-SA 3.0.
%  Freie Universität Berlin, Institute of Computer Science, Human Centered Computing. 
%
\pagestyle{empty}

\subsection*{Eidesstattliche Erklärung}

Ich versichere hiermit an Eides Statt, dass diese Arbeit von niemand anderem als meiner Person verfasst worden ist. Alle verwendeten Hilfsmittel wie Berichte, Bücher, Internetseiten oder ähnliches sind im Literaturverzeichnis angegeben, Zitate aus fremden Arbeiten sind als solche kenntlich gemacht. Die Arbeit wurde bisher in gleicher oder ähnlicher Form keiner anderen Prüfungskommission vorgelegt und auch nicht veröffentlicht.
\par\bigskip  
\noindent Berlin, den \today

\vspace{1.2cm}

\noindent Alexa Schlegel

\cleardoublepage
%%*******************************************************
% Dedication
%*******************************************************
\thispagestyle{empty}
%\phantomsection 

\pdfbookmark[1]{Dedication}{Dedication}

\vspace*{3cm}

\begin{center}
    Everything we hear is an opinion, not a fact. \\ Everything we see is a perspective, not the truth.  \\ \medskip
    --- Marcus Aurelius 
\end{center}

\medskip

\begin{center}
    Dedicated to my parents and my sister.
\end{center}

%---------------------------------------------------
%----- Abstracts in English and German   
%---------------------------------------------------

% ---------------------------------------------------
% ----- Abstract (English)
% ---------------------------------------------------
%  Freie Universität Berlin, Institute of Computer Science, Human Centered Computing. 
%
\pagestyle{empty}
\providecommand{\keywords}[1]{\textbf{\textit{Keywords---}} #1}

\subsection*{Abstract}
The BeesBook system automatically tracks individual honey bees inside a hive over their entire life and provides a high-resolution dataset of bee movements of a single colony.
This thesis focuses on the inference of interaction networks, by implementing a network pipeline.
Spatial proximity is using as an indicator for interactions between bees.
Social network analysis methods were applied to investigate the static and dynamic properties of the resulting social networks of honey bees on a global, intermediate and local level.
The resulting networks were characterized by a low hierarchical structure and a high density.
The global structure of the colony seems to be stable over time.
The local structure is highly dynamic, as bees change communities as they age.
Communities in the honey bee network represent age groups with a high spatial fidelity.
The findings are in line with established state of research that a colonies organization is shaped by the age-based task division of individuals.
The results of the analysis validate the implemented pipeline and inferred networks and consequently provide an excellent foundation for future work focusing more on temporal network analysis aspects.

\vspace{5mm}
\keywords{social insects, spatial proximity network, social network, interaction network, honey bee, behavioural tracking, Apis mellifera, community detection, social network analysis}

\cleardoublepage

% ---------------------------------------------------
% ----- Abstract (German) of the template
% ----- for Bachelor-, Master thesis and class papers
% ---------------------------------------------------
%  Created by C. Müller-Birn on 2012-08-17, CC-BY-SA 3.0.
%  Freie Universität Berlin, Institute of Computer Science, Human Centered Computing. 
%
\pagestyle{empty}

\subsection*{Zusammenfassung}
TODO

\cleardoublepage  
                                          
%---------------------------------------------------
%----- Table of content   
%---------------------------------------------------
\tableofcontents
%\setcounter{tocdepth}{3}   % reduce the included sections in the table of content

%---------------------------------------------------
%----- Main part
%---------------------------------------------------
\mainmatter

\pagestyle{fancy} 

\chapter{Introduction}
\label{ch:intro}

Complex insect societies are formed by thousands of individuals, which continously move and interact inside a dark nest. Honey bee colonies are thus organized complex social systems, which form a collective intelligence. Observing individual honey bees is therefore vital for understanding collective behavior, decision making and organisation of task within the colony.

The Biorobotics Lab of Freie Universität Berlin developed technologies to track all individuals of a complete honey bee (apis mellifera) colony. Conventional approaches usually focus on a small subset of the hive life, whether this regards time, space, or animal identity~\cite{wario2015automatic}[TODO: change sentence]. All individuals of a colony have been marked on their thorax using circular tags (figure~\ref{fig:markers}). Spatial information for each bee is then recorded for a peroid of nine weeks.


\begin{figure}[htb]
	\centering
	\includegraphics[width=1.0\textwidth]{Figures/markers}
	\caption{Tagged bees inside the hive.}
	\label{fig:markers}
\end{figure}

\section{Motivation}

Most of the studies analysing behaviour of insects colonies only use a small amount of individually labeled animals, a short observation period, and usually manually detect interaction between animals looking at the videos. data~\cite{quevillon2015social}[TODO: more references]. Here it is done in a more inclusive way, all animals, long term observation, automatic detection of individuals.

\section{Research Goal}

Starting off with creating worker-worker interaction networks using spatial proximity as a proxy for interactions between bees.

Answering the following questions:

(1) Is it possible to infer networks with the provided data dataset? (challenges and limitations)\\

(2) Welche Eigenschaften haben diese Netzwerke? Was sind das für Netzwerke? (Netzwerkklasse: ziemlich dichtes (auf keinen fall sparse) Netzwerk, ungerichtetes, gewichtetes Netzwerk (höhres Gewicht, engere Benziehung). no power law degree distribution, daher keine Hubs, sehr geringer Diameter, daher small-world property, small clustering coefficient and giant component.\\

(3) Gibt es in diesen Netzwerken Communities?\\

(4) Wie sind diese charakterisiert? (machen die Sinn) spiegeln die das wieder was man bisher über Bienen weiß? (Alter und Aufenthaltsort)\\

(5) Wie entwicklen sich die Communities über die Zeit?\\
über einen Zeitraum von X tagen (y tage) bleiben die zwei Gruppen erhalten und auch altersverteilung bleibt signifikant verschieden.\\
-> stabilität nachgewiesen?\\
für XY einfach plotten\\

(6) Wie ändern sich die Mitglieder der Communities über die Zeit?\\


\section{Methodology}

Network Science Approach.

\section{Outline}
[TODO]
\chapter{Theretical Background}


\section{Social Network Analysis}

\section{Temporal Networks}

\section{Community Detection}

\section{Community Tracking}
\chapter{Related Work}
\label{ch:relatedwork}

Relevant for my work are studies using a network analysis approach focusing on interaction networks\footnote{Studies using worker-task, worker-nestarea, nestarea-nestarea or other bipartite networks are excluded.} to investigate the behavior of social insects, especially honey bees.
Therefore, I conducted a literature review in the field of network analysis of social insect\footnote{Animals belonging to social insects are: ants, bees, wasps, and termites.} colonies.
I mainly reviewed studies mentioned in the survey papers of \textcite{Pinter-Wollman2014}, \textcite[chapter~15]{krause2014animal} and \textcite{charbonneau2013social}.


The most relevant studies were classified by (1) type of analysis: temporal or static analysis (using automated or manual tracking over a long or short term); and (2) studied species: honey bees or other social insects.
Additionally, I inspected their shortcomings regarding time, space, and the number of tracked individuals, and thus, examined the following characteristics: duration of the study, observation period, sampling resolution, the number of colonies and marked individuals, space limitations and whether they integrated age cohorts.
Table~\ref{tab:studies} (Appendix~\ref{ch:appendix}) summarizes the extracted studies and their characteristics. Also, I summarized the used software tools for network analysis.


In the field of static analysis of honey bee networks, a few studies exist~\cite{baracchi2014socio,naug2008structure,scholl2011olfactory,naug2007experimentally}, but most of the papers relate to other social insects like ants, wasps and bumblebees~\cite{greenwald2015ant,pinter2011effect,otterstatter2007contact,quevillon2015social,naug2009structure,formica2012fitness,waters2012information,sendova2010emergency}.
Studies focusing on temporal aspects exist only for ants~\cite{mersch2013tracking,blonder2011time,jeanson2012long}, but, to the best of my knowledge, not for honey bee colonies.


The work by \textcite{kimura2011new} presents an automatic tracking system for honeybees, but the system is due to memory and storage limitations not usable for long-term observations, neither did they use network analysis methods for data analysis.

\section{Static Network Analysis of Honey Bee Colonies}

The most advanced work studying honey bees using a network science approach is by \textcite{baracchi2014socio}.
Using colored numbered discs for individually marked bees, they reveal a highly compartmentalized structure inside the honey bee colony.
Depending on the age, bees occupy separate areas of the comb and perform different tasks. Also, there is limited contact between age groups.
The frequency of interactions between bees is used as weights for edges in an undirected worker-worker interaction network. The body length of a bee defines the radius of spatial proximity.
Baracchi and Cini make use of the node level measures strength (weighted degree), closeness and eigenvector centrality to investigate the networks.
Furthermore, they perform a cluster analysis using the dissimilarity measures ’average linkage between groups’ and ’squared Euclidian distance among network values.'
The main drawback is that they marked only 211 bees from three predefined age cohorts out of one colony with 4000 individuals and observed only one side of the observation hive for ten hours by capturing with a low resolution of one frame per minute. [TODO: explain drawback of clustering in a better way]

%%%%%%%%%%%%%%%%%%%%%%%%%%%%%%%%%%%%%%%%%%%%%%%%%%%%%%%%%%%%%%%%%%%%%%%%%%%%%%%

\textcite{scholl2011olfactory} investigate the mechanism behind the emergence of organizational immunity by using unweighted, undirected physical contact and trophallaxis networks.
In their case, the observation is limited to one hour per day, with three days of observation spread over three weeks.
Besides looking at the interactions between three predefined age groups, no other methods regarding networks are used.

%%%%%%%%%%%%%%%%%%%%%%%%%%%%%%%%%%%%%%%%%%%%%%%%%%%%%%%%%%%%%%%%%%%%%%%%%%%%%%%

\textcite{naug2008structure} inspects the network structure of weighted, directed trophallaxis networks using four age cohorts and evaluates the changes in transmission dynamics produced by experimental manipulation.
The data set is limited to one hour and only first- and second-order trophallaxis interactions are considered. The food transfer from the forager to a worker bee is called first level interaction, the food transfer from that worker bee to other bees is called second-order. The study does not capture other levels of trophallaxis.


%%%%%%%%%%%%%%%%%%%%%%%%%%%%%%%%%%%%%%%%%%%%%%%%%%%%%%%%%%%%%%%%%%%%%%%%%%%%%%%
%%%%%%%%%%%%%%%%%%%%%%%%%%%%%%%%%%%%%%%%%%%%%%%%%%%%%%%%%%%%%%%%%%%%%%%%%%%%%%%
\section{Temporal Network Analysis of Insect Colonies}
%%%%%%%%%%%%%%%%%%%%%%%%%%%%%%%%%%%%%%%%%%%%%%%%%%%%%%%%%%%%%%%%%%%%%%%%%%%%%%%
%%%%%%%%%%%%%%%%%%%%%%%%%%%%%%%%%%%%%%%%%%%%%%%%%%%%%%%%%%%%%%%%%%%%%%%%%%%%%%%

Regarding the used methods, the study of~\textcite{mersch2013tracking} is very close to my work.
They automatically tracked all individuals of six ant colonies over a period of 41 days using a resolution of two frames per second.
For each observation day, the authors extracted time-aggregated weighted contact networks per colony, using antennation as the physical contact event.
They applied the Infomap community detection algorithm to each daily network and thus revealed three distinct and robust groups.
Each group represents a functional behavioral unit, with ants changing groups as they age.
The six ant colonies, they studied, contained 122 to 192 individuals, which is relatively small compared to the size of honey bee colonies used in the static analysis approaches.
Except for community detection, they did not use any other network science methods to investigate the network properties.

%%%%%%%%%%%%%%%%%%%%%%%%%%%%%%%%%%%%%%%%%%%%%%%%%%%%%%%%%%%%%%%%%%%%%%%%%%%%%%%

Another work, using automatic tracking, is by\textcite{jeanson2012long}.
It focuses on the investigation of the temporal stability of spatial proximity networks in four ant colonies.
Here, proximity is defined as $\frac{4}{3}$ of an ant’s body length.
For each week of three weeks of observation, they generate weighted time-aggregated networks per colony,  using the total duration of interaction as the edge weights.
They investigated the strength, betweenness and closeness centrality and found out that the networks are stable over time, without the queen contributing to the network structure.
Individuals with long lasting interactions seem to have a reduced tendency to move, while mobile ants interact homogeneously with their nestmates.
Nevertheless, the size of the observed colonies ranges from 55 to 58 individuals, which is again, compared to bee colonies, rather small.

%%%%%%%%%%%%%%%%%%%%%%%%%%%%%%%%%%%%%%%%%%%%%%%%%%%%%%%%%%%%%%%%%%%%%%%%%%%%%%%

The only study not only using time-aggregated but time-ordered (dynamic) networks is by \textcite{blonder2011time}.
They marked all individuals of four ant colonies with colored paint and filmed each colony for 30 minutes on two days being three weeks apart.
The interaction events, physical contact of an ant's antenna with an ant's body, were manually extracted by watching the videos. Edges are therefore time-stamped interactions between individuals.
They show how temporal and spatial dynamics of individual interactions provide upper bounds to rates of colony-level information flow and how this flow scales with individual mobility and group size.
This very specialized study on dynamics in information flow also observed colonies with 6 to only 90 individuals.
\chapter{Approach and Implementation}
\label{ch:approach}

textwidth: \printinunitsof{in}\prntlen{\textwidth}

linewidth: \printinunitsof{in}\prntlen{\linewidth}

In this chapter the basic work flow is described in detail. The process is mainly drive by an exploratory approach, but follows primarily Farines and Whiteheads~\cite{farine2015constructing} primary steps and key considerations for social network analysis to non-human animal data. The adapted and resulting process is visualized in figure~\ref{fig:process}.

The dataset was first analysed regarding data quality and to form an understanding of the dataset and behaviour of bees in general. Those findings were used to define nodes and infer associations/edges to build the network, respectively derive parameters for the network-generating-pipeline. The static and temporal networks are analysed using network scienc tools and methods (e.g. XXX). For testing hypothesis the networks are combined with attributed data (positions and age information). Each step is explained within the following sections.

\begin{figure}[htb]
	\centering
	\includegraphics[width=1.0\textwidth]{Figures/WorkProcess}
	\caption{Steps of the Research Approach}
	\label{fig:process}
\end{figure}


\section{The Dataset}
\label{sec:dataset}
The basis of the dataset are video files, that capture tagged honey bees of one colony in a two sided observation hive.
Each individual of the colony, including about 3200 bees, were tagged with 12-bit markers. Four cameras were used to film the hive, the setup of the cameras is illustrated in figure~\ref{fig:cams} and an example of tagged bees is shown in figure~\ref{fig:markers}.

The recording season lastet nine weeks (63 days), around the clock, from 19.07.2016 until 19.09.2016, with some interruptions due to maintainance and technical failures, this is shown in figure~\ref{fig:period}.
The recording resolution of each camera is three frames per second, aiming for $1024$ frames (about $5.7$ minutes) for a video files.For each frame, bee detections were extracted by using an image analysis pipeline.

\begin{figure}[htb]
	\centering
	\includegraphics[width=1.0\textwidth]{Figures/setupCams}
	\caption{Camera Setup in 2016: (1) Top View:  vertical hive with two cameras for each side, overlapping in the middle. (2) Front View: left and right camera setup, the red dot indicated the origin $(0,0)$ of the camera.}
	\label{fig:cams}
\end{figure}

The resulting detection data is stored in a binary file format. A python library called \emph{bb-binary}\footnote{\url{https://github.com/BioroboticsLab/bb_binary}; Last accesed: 2106-02-16, 04:28PM} provides easy access to the binary files. Each file in bb\_binary file format corresponds to a video file of a single camera.
The size of the complete dataset for 2016 is $470$~GB, about $7.5$~GB of binary data per day.

In 2016 exactly $3.191$ bees were tagged. The tagging period is 67 days long. The tagging started on 2016-06-28 (22 days before the recording started) and lasted until 2016-09-02, so 17 days before the recording ended. The your bees were tagged and then added to the hive. The overall tagging frequency is shown in figure~\ref{fig:tagging}. The hatching day for each bee is known. On the day when the recording started about half of the tags were used up. 

\begin{figure}[htb]
	\centering
	\includegraphics[width=0.5\textwidth]{Figures/foo}
	\caption{Recording Season with maintainance and failures}
	\label{fig:period}
\end{figure}

\begin{figure}[htb]
	\centering
	\includegraphics[width=1.0\textwidth]{Figures/tagging_period}
	\caption[Tagging Frequency]{Tagging frequency of bees: The bees were primarily tagged during the week. The weekend (sometimes also mondays or fridays) no bees were tagged. On average 48 bees were tagged each day, considering only tagging days, the average is about 91 ($\pm50$) bees (median 118, mode 128).}
	\label{fig:tagging}
\end{figure}

\subsection{Structure of the Dataset}
The data is organised in \emph{frame container}, wich corresponds to a video file of a single camera. A frame container holds all \emph{frames} for that specific video.
Each frame has a list of all bees detected by the image analysis pipeline.

A \emph{detection} has the following attributes, which are relevant to this project:

\begin{itemize}
\item \textbf{xpos}: $x$ coordinate of bee with respect to the image in pixel
\item \textbf{ypos}: $y$ coordinate of bee with respect to the image in pixel
\item \textbf{radius}: of the tag
\item \textbf{decodedId}: decoded 12-bit id, the bit probabilities are discretised to 0-255
\end{itemize}

Besides further information, the frame container specifies the camera and a frame is also attributed with a timestamp. The data can be accessed iterating on frame container (file) or on frame level, in both cases using timestamps for start and end. The data scheme is illustrated in figure~\ref{fig:scheme}.
The complete data scheme can be found on github\footnote{\url{https://github.com/BioroboticsLab/bb_binary/blob/master/bb_binary/bb_binary_schema.capnp}; Last accessed: 2106-02-16, 04:46PM}. 


\subsection{ID Confidence and Tracking Quality}
Each bit of the decoded 12-bit ID represents a probability between $0$ and $255$. That means when using a high ID confidence\footnote{The confidence of an ID is calculated as follows. TODO} results in less data, but with a higher accuracy, a low confidence on more data, but rather uncertain and errors. A good tradeoff between data quality and amount of data should be chosen. Figure~\ref{fig:tradeoff} shows the proportion of wrong detections depending on the confidence level for all four cameras. For a confidece level of $0.9$ wrong detection are about $4\%$, but the amount of data is still at $70\%$ (todo choose values according to figure).

\begin{figure}[htb]
	\centering
	\includegraphics[width=1.0\textwidth]{Figures/structure}
	\caption{Structure of the data scheme}
	\label{fig:scheme}
\end{figure}

\begin{figure}[htb]
	\centering
	\includegraphics[width=0.5\textwidth]{Figures/foo}
	\caption[Tradeoff: Confidence level, data quality and amount of data]{On 21.07.2016, about half of the bee tags were used up. This day was chosen to determine the effects of the ID confidence level on data quality and amount of remaining data. For each camera a ten minute test dataset was chosen (12:00-12:10) with a sample size of $10.000$. For each detected bee, the age was determined, negative ages were counted as wrong detection (dann mal zwei rechnen weil die falschen die rechts drin sind sieht man ja nicht, desswegen wurde die Haelfte gewaehlt.).}
	\label{fig:tradeoff}
\end{figure}

Some statistics about the tracking quality, gaps of size one, two, three and so on. This is relevant for later on. Damn!

\subsection{Some Statistics}
TODO
speed of bees\\
presence and absence of bees (counted and as duration)\\

\subsection{Implications}
TODO


\section{Inferring Networks}

\subsection{Network Pipeline}
\subsection{Thresholding Edges}
\subsection{Runtime and Complexity}

\section{Static and Temporal Analysis}	







%%%%%%%%%%%%%%%%%%%%%%%%%%%%%%%%%%%%%%%%%%%%%%%%%%%%%%%%%%%%%%%%%%%%%%%%%%%%%%%
%%%%%%%%%%%%%%%%%%%%%%%%%%%%%%%%%%%%%%%%%%%%%%%%%%%%%%%%%%%%%%%%%%%%%%%%%%%%%%%
%%%%%%%%%%%%%%%%%%%%%%%%%%%%%%%%%%%%%%%%%%%%%%%%%%%%%%%%%%%%%%%%%%%%%%%%%%%%%%%
%%%%%%%%%%%%%%%%%%%%%%%%%%%%%%%%%%%%%%%%%%%%%%%%%%%%%%%%%%%%%%%%%%%%%%%%%%%%%%%
\chapter{Results of Network Analysis}
\label{ch:results}
%%%%%%%%%%%%%%%%%%%%%%%%%%%%%%%%%%%%%%%%%%%%%%%%%%%%%%%%%%%%%%%%%%%%%%%%%%%%%%%
%%%%%%%%%%%%%%%%%%%%%%%%%%%%%%%%%%%%%%%%%%%%%%%%%%%%%%%%%%%%%%%%%%%%%%%%%%%%%%%
%%%%%%%%%%%%%%%%%%%%%%%%%%%%%%%%%%%%%%%%%%%%%%%%%%%%%%%%%%%%%%%%%%%%%%%%%%%%%%%
%%%%%%%%%%%%%%%%%%%%%%%%%%%%%%%%%%%%%%%%%%%%%%%%%%%%%%%%%%%%%%%%%%%%%%%%%%%%%%%
This chapter summarizes the analysis results of the temporal, spatial proximity network of honey bees, consisting of three consecutive time-aggregated snapshots.\\
The first section describes my results related to static aspects of the networks on three levels.
First I examine the networks' global structure and derive properties of the overall colony (global level). Second I study the characteristics of individual bees (local level), and it's relation to detection frequency and age.
Additionally, I investigate the intermediate level of the colonies social organization by detecting communities and inspecting their practical meaning.\\
The second section focuses on the temporal network aspects of all three snapshots.
I investigate the stability of local and global properties, as well as the stability of functional groups of bees concerning age and spatial distribution. Furthermore, the dynamics of individual bees regarding their group membership over time is examined.
The last section of this chapter summarizes the main results and discusses the findings.

%%%%%%%%%%%%%%%%%%%%%%%%%%%%%%%%%%%%%%%%%%%%%%%%%%%%%%%%%%%%%%%%%%%%%%%%%%%%%%%
%%%%%%%%%%%%%%%%%%%%%%%%%%%%%%%%%%%%%%%%%%%%%%%%%%%%%%%%%%%%%%%%%%%%%%%%%%%%%%%
\section{Static Analysis}
%%%%%%%%%%%%%%%%%%%%%%%%%%%%%%%%%%%%%%%%%%%%%%%%%%%%%%%%%%%%%%%%%%%%%%%%%%%%%%%
%%%%%%%%%%%%%%%%%%%%%%%%%%%%%%%%%%%%%%%%%%%%%%%%%%%%%%%%%%%%%%%%%%%%%%%%%%%%%%%

I analyzed a temporal network, consisting of three time-aggregated snapshots; these are referred to below as snapshot~1~($N=922$), snapshot~2~($N=978$) and snapshot~3~($N=922$). 
The snapshots are aggregated for ten hours (108,000 frames) starting at 8~a.m. and lasting until 6~p.m, see table~\ref{tab:networks} for details about the added bees per day,  figure~\ref{fig:ages} for the age distributions. Figure~\ref{fig:network-matching} shows the proportion of intersecting bees between each snapshot. This figure illustrates the stability of the network concerning its size. 

\begin{table}[htb]
\small
\centering
\caption[Sampling period]{\textbf{Sampling period} Overview of the chosen aggregated daily snapshots including the number of added bees and the time they were added to the hive.}
\vspace*{5mm}
\begin{tabularx}{\textwidth}{ccccccc}
\toprule
{} & 20.08.16 & 21.08.16 & 22.08.16 & 23.08.16 & 24.08.16 \\
\midrule
Snapshot ID & 1 & - & 2 & - & 3 & \\
Number of added bees & 0 & 0 & 110 & 60 & 0 \\
Time added & - & - & 2~p.m. & 6~p.m. & - \\
\bottomrule
\end{tabularx}
\label{tab:networks}
\end{table}

\begin{figure}[htb]
	\centering
	\includegraphics[width=.8\textwidth]{Figures/network_matching}
	\caption[Number of bees per snapshot]{\textbf{Number of bees per snapshot} This figure show the amount of bees for each snapshot and the proportion of intersecting bees between snapshots.}
	\label{fig:network-matching}
\end{figure}

Each snapshot consists of one large component.
Table~\ref{tab:stats} summarizes basic network properties.
For all, the density $D$ is over 50\%.
The diameter $\langle d_{\texttt{max}} \rangle$ is three and the average shortest path length $\langle d \rangle$ is below two.
The global clustering coefficient $C_\Delta$ of all snapshots is higher than compared to an Erdös-Renyi random graph, averaged over 100 runs using the same number of nodes and edges.
The high clustering coefficient and the small diameter suggest a small-world network type.
On average, each bee is connected to at least 50\% of all other bees in the network.

Figure~\ref{fig:fVSd} shows a positive correlation between the frequency of interactions and the total duration of interactions (averaged).
 The weight of edges is the frequency of interactions.
The edge weight distribution is shown in figure~\ref{fig:edgeWdist}.
Most edges have a low weight; only a few edges have a high weight.
It seems that bees do not prefer individuals bees for interaction.

\begin{table}[htb]
\centering
\caption[Global network properties]{\textbf{Global network properties} $N$ is the number of nodes, $L$ the number of edges, $D$ is the diameter, $\langle d_{\texttt{max}} \rangle$ is the average path length, $C_\Delta$ the global clustering coefficient, $C_{\Delta}^\texttt{rand}$ is the global clustering coefficient for randomized graph, $\langle k \rangle$ the average degree and $\langle s \rangle$ represents the average strength, as introduced in section~\ref{sec:definitions}.}
\label{tab:stats}
\vspace*{5mm}
\begin{tabularx}{\textwidth}{lccccccccc}
\toprule
{} &  $N$ &   $L$ &  $D$ &  $\langle d_{\texttt{max}} \rangle$ &  $\langle d \rangle$ &   $C_\Delta$ & $\langle k \rangle$ &  $\langle s \rangle$ \\
\midrule
Snapshot 1 & 922 & 291179 & 0.69 & 3 & 1.32 &  0.79 & 631.62 & 5680.17 \\
Random 1  & 922 & 291179 & 0.69 & 2 & 1.31 &  0.69 & 631.62 & - \\ \midrule
Snapshot 2 & 978 & 256066 & 0.54 & 3 & 1.46 &  0.72 & 523.65 & 3977.94 \\
Random 2  & 978 & 256066 & 0.54 & 2 & 1.46 &  0.54 & 523.65 & - \\ \midrule
Snapshot 3 & 922 & 259421 & 0.61 & 3 & 1.39 &  0.75 & 562.74 & 4205.99 \\
Random 3  & 922 & 259421 & 0.61 & 2 & 1.39 &  0.61 & 562.74 & - \\
\bottomrule
\end{tabularx}
\end{table}

\begin{figure}[htb]
	\centering
	\begin{subfigure}[b]{0.49\textwidth}
	\centering
	\includegraphics[width=1.0\textwidth]{Figures/n3-freqVSduration}
	\caption[Type of edge weights]{Type of edge weights}
	\label{fig:fVSd}
	\end{subfigure} 
	\begin{subfigure}[b]{0.49\textwidth}
	\centering
	\includegraphics[width=1.0\textwidth]{Figures/n3-edgeWeightDist.pdf}
	\caption[Edge weight distribution]{Edge weight distribution}
	\label{fig:edgeWdist}
	\end{subfigure}
	\caption[Edge wights]{\textbf{Edge wights} }
	\label{fig:edges}
\end{figure}


Figure~\ref{fig:n3ageDist} shows the age distribution of the investigated snapshot. This distribution does not seem to follow any known distribution. It corresponds to the artificial tagging of bees. Consequently, bees of certain age groups are simply not present. The detection frequency of an individual bee is negatively correlated with its age (figure~\ref{fig:n3detfVSage}).


\begin{figure}[htb]
	\centering
	\begin{subfigure}[b]{0.33\textwidth}
	\centering
	\includegraphics[width=1.0\textwidth]{Figures/n3_detFvsAge}
	\caption[]{}
	\label{fig:n3detfVSage}
	\end{subfigure} 
	\begin{subfigure}[b]{0.66\textwidth}
	\centering
	\includegraphics[width=1.0\textwidth]{Figures/n3_ages.pdf}
	\caption[]{}
	\label{fig:n3ageDist}
	\end{subfigure}
	\caption[X]{\textbf{X}}
	\label{fig:ageDetF}
\end{figure}

%%%%%%%%%%%%%%%%%%%%%%%%%%%%%%%%%%%%%%%%%%%%%%%%%%%%%%%%%%%%%%%%%%%%%%%%%%%%%%%
\subsection{Degree, Strength and Local Clustering Coefficient}
%%%%%%%%%%%%%%%%%%%%%%%%%%%%%%%%%%%%%%%%%%%%%%%%%%%%%%%%%%%%%%%%%%%%%%%%%%%%%%%

bimodal degree distribution\\
type of network: no scale free\\
todo plot in relation to age of bees\\
todo plot in relation to detection frequency\\

\begin{figure}[!htb]
	\centering
	\begin{subfigure}[b]{1.0\textwidth}
	\centering
	\includegraphics[width=1.0\textwidth]{Figures/n3-stat-degreeStrLCC}
	\caption[Distributions]{\textbf{Distributions}}
	\label{fig:n3-d-s-cc}
	\end{subfigure}
	\caption[Degree, strength and local clustering coefficient]{\textbf{Degree, strength and local clustering coefficient} xxx}
	\label{fig:n3-degreeStrLCC}
\end{figure}

%%%%%%%%%%%%%%%%%%%%%%%%%%%%%%%%%%%%%%%%%%%%%%%%%%%%%%%%%%%%%%%%%%%%%%%%%%%%%%%
\subsection{Betweenness and Closeness Centrality}
%%%%%%%%%%%%%%%%%%%%%%%%%%%%%%%%%%%%%%%%%%%%%%%%%%%%%%%%%%%%%%%%%%%%%%%%%%%%%%%
[TODO]\\
in relation to age and detection frequency\\
closeness\\
betweenness\\

%%%%%%%%%%%%%%%%%%%%%%%%%%%%%%%%%%%%%%%%%%%%%%%%%%%%%%%%%%%%%%%%%%%%%%%%%%%%%%%
\subsection{Communities}
%%%%%%%%%%%%%%%%%%%%%%%%%%%%%%%%%%%%%%%%%%%%%%%%%%%%%%%%%%%%%%%%%%%%%%%%%%%%%%%
The leading eigenvector community detection algorithms revealed two communities, about the same size, the walktrap algorithm instead three communities (see table~\ref{tab:n3-communities}).
The communities correspond to separate age groups and are located in different regions on the comb (see figure~\ref{fig:n3-communities}). The younger communities are situated in the center and the old communities closer to the hive exit. The middle-aged community (only for walktrap) is located between and on the periphery. Table~\ref{tab:n3-pvalues2} shows the $p$-values for the two sampel KS-test.

\begin{table}
\centering
\caption[Communities per algorithm]{\textbf{Communities per algorithm} Communities marked with * contain the queen. Age and standard deviation (SD) are measured in days. The queen and bees with a negative age (10 bees).}
\label{tab:n3-communities}
\vspace*{5mm}
\begin{tabular}{lcrrrrr}
	\toprule
	{}  & Community ID & Members & Proportion & Age & SD\\
	\midrule  
	\quad LE  & CY & $*381$  & 41.78\% & $13.15$ & $\pm13.50$ \\
	          & CO & $531$   & 58.22\% & $28.70$ & $\pm11.67$ \\
    \midrule 
	\quad WT & CY & $*229$  & 25.11\% & $6.55$  & $\pm10.36$\\
			 & CM & $298$  & 32.68\% & $25.08$ & $\pm11.97$\\
			 & CO & $385$  & 42.21\% & $29.29$ & $\pm11.44$\\
	\bottomrule
\end{tabular}
\end{table}
\begin{table}[htb]
\small
\centering
\caption[Kolmogorov-Smirnov test]{\textbf{Kolmogorov-Smirnov test} $p$-values for leading eigenvector (LE) and walktrap (WT)}
\label{tab:n3-pvalues2}
\vspace*{5mm}
\begin{tabular}{crrrrr}
	\toprule
	 Communities & LE p-value & WT p-value\\
	\midrule 
    CY, CO & 5.10e-66 & 5.51e-67\\
    CY, CM &          & 1.10e-95\\
    CM, CO &          & 1.98e-05\\ 
	\bottomrule
\end{tabular}
\end{table}

\begin{figure}[!htb]
	\centering
	\begin{subfigure}[b]{1.0\textwidth}
	\centering
	\includegraphics[width=1.0\textwidth]{Figures/le_network3}
	%\vspace{1pt}
	\end{subfigure} 
	\begin{subfigure}[b]{1.0\textwidth}
	\centering
	\includegraphics[width=1.0\textwidth]{Figures/n3-ageDistribution-LE}
	\caption[Leading eigenvector communities]{Leading eigenvector communities}
	\label{fig:n3ageLE}
	\end{subfigure}
	\begin{subfigure}[b]{1.0\textwidth}
	\vspace{5mm}
	\centering
	\includegraphics[width=1.0\textwidth]{Figures/wt_network3}
	\end{subfigure}
	\begin{subfigure}[b]{1.0\textwidth}
	\centering
	\includegraphics[width=1.0\textwidth]{Figures/n3-ageDistribution-WT}
	\caption[Walktrap communities]{Walktrap communities}
	\label{fig:n3ageWT}
	\end{subfigure}
	\caption[Age and spatial distribution of communities]{\textbf{Age and spatial distribution of communities} \emph{Green} represents the young community occupying the center are of the comb and \emph{orange} the old community, which is situated closer to the hive access. For walktrap the middle-aged community is depicted in \emph{gray} and is located inbetween.}
	\label{fig:n3-communities}
\end{figure}
\newpage
\section{Temporal Analysis}

same edge weight distribution\\
bess were added to the colony can be seen in table X, histogramm can be found in appendix for all three networks\\
same correlation between age and detection frequency for all three networks (see appendix), the older the less oftern detected\\


same degree distribution, can be seen in figure X\\
same strength distribution and same local clustering coefficient distribution (figure are in appendix)\\
figure degee distribution\\
maybe: relation to age and detection frequency also in appendix\\


same centrality distribution betweenness is in appendix\\
figure for closeness distribution\\
maybe relation to age and detection frequency also in appendix\\


%%%%%%%%%%%%%%%%%%%%%%%%%%%%%%%%%%%%%%%%%%%%%%%%%%%%%%%%%%%%%%%%%%%%%%%%%%%%%%%
\subsection{Stable Communities}
%%%%%%%%%%%%%%%%%%%%%%%%%%%%%%%%%%%%%%%%%%%%%%%%%%%%%%%%%%%%%%%%%%%%%%%%%%%%%%%
Table~\ref{tab:communities} lists the exact number of bees per community for each algorithm and snapshot.
For each snapshot, the leading eigenvector detected two communities with about the same number of bees.
The first communities CY(1,2,3) contain the queen and on average younger bees than the second communities CO(1,2,3).\\
In comparison, walktrap identified three communities, but two for the first snapshot.
Again the first communities CY(1,2,3) consist of the queen and on average younger bees than the second CM(2,3) and third communities CO(1,2,3).
The bees in CM2 and CM3 are on average younger than the bees in CO2 and CO3.
Figure~\ref{fig:ageDistribution} depicts the age distribution for each community and snapshot.

A two-sample Kolmogorov–Smirnov test showed that the age distributions are significantly different ($p< 0.001$) for both algorithms. However, the $p$-values for the walktrap communities CM2, CO2, and CM3, CO3 are lower.

CY(1,2,3) are located in the center of the comb, CO(1,2,3) closer to the hive access and CM(2,3) are situated in between. This spatial segregation of communities is similar in all three snapshots. For further reference see heat maps in~\ref{fig:communitiesPerNetworkWT} and~\ref{fig:communitiesPerNetworkLE}.


Functional groups of honey bees seem to differ in their respective age and occupy different areas of the comb.


\begin{table}
\small
\centering
\caption[Overview about communities]{\textbf{Overview about communities per snapshot} Communities marked with * contain the queen. Age and standard deviation (SD) are measured in days. For each network the queen and bees with a negative age are excluded: snapshot 1 - 12 bees, snapshot 2 - 119 bees, snapshot 3 - 10 bees.}
\label{tab:communities}
\vspace*{5mm}
\begin{tabular}{lcrrrrr}
	\toprule
	{}  & ID & Members & Proportion & Age & SD\\
	\midrule

	Leading eigenvector &&&&&\\
	\midrule 
	\quad Snapshot 1  & CY1 & $*430$  & 47.25\% & $17.12$ & $\pm10.97$ \\
	                 & CO1 & $480$   & 52.75\% & $27.24$ & $\pm10.96$ \\
	\midrule   							
	\quad Snapshot 2  & CY2 & $*392$  & 45.63\% & $20.24$ & $\pm12.01$ \\
	                 & CO2 & $467$   & 54.37\% & $28.10$ & $\pm10.88$ \\
	\midrule  
	\quad Snapshot 3  & CY3 & $*381$  & 41.78\% & $13.15$ & $\pm13.50$ \\
	                 & CO3 & $531$   & 58.22\% & $28.70$ & $\pm11.67$ \\
    \midrule

    Walktrap &&&&&\\
    \midrule 
	\quad Snapshot 1 & CY1 & $*427$ & 46.92\% & $17.07$ & $\pm10.92$\\
	                & CO1 & $482$  & 52.97\% & $27.23$ & $\pm11.00$\\
	\midrule
	\quad Snapshot 2 & CY2 & $*263$ & 30.62\% & $18.23$ & $\pm11.46$\\
				    & CM2 & $305$  & 35.51\% & $25.20$ & $\pm11.47$\\
				    & CO2 & $291$  & 33.88\% & $29.47$ & $\pm10.06$\\            
	\midrule
	\quad Snapshot 3 & CY3 & $*229$  & 25.11\% & $6.55$  & $\pm10.36$\\
					& CM3 & $298$  & 32.68\% & $25.08$ & $\pm11.97$\\
					& CO3 & $385$  & 42.21\% & $29.29$ & $\pm11.44$\\
	\bottomrule
\end{tabular}
\end{table}
\begin{table}
\small
\centering
\caption[Kolmogorov-Smirnov test]{\textbf{Kolmogorov-Smirnov test} $p$-values for leading eigenvector (LE) and walktrap (WT) for each snapshot and its communities.}
\label{tab:pvalues2}
\vspace*{5mm}
\begin{tabular}{lcrrrrr}
	\toprule

	 & & LE p-value & WT p-value\\
	\midrule 
	\quad Snapshot 1     & CY1, CO1 & $2.18\times10^{-33}$ & $1.52\times10^{-32}$ \\
	\midrule   							
	\quad Snapshot 2     & CY2, CO2 & $2.99\times10^{-20}$ & $2.3\times10^{-32}$ \\
					    & CY2, CM2 &          & $4.72\times10^{-10}$\\
					    & CM2, CO2 &          & $1.00\times10^{-04}$\\
	\midrule  
	\quad Snapshot 3     & CY3, CO3 & $5.10\times10^{-66}$ & $5.51\times10^{-67}$\\
					    & CY3, CM3 &          & $1.10\times10^{-95}$\\
						& CM3, CO3 &          & $1.98\times10^{-05}$\\ 
	\bottomrule
\end{tabular}
\end{table}


%%%%%%%%%%%%%%%%%%%%%%%%%%%%%%%%%%%%%%%%%%%%%%%%%%%%%%%%%%%%%%%%%%%%%%%%%%%%%%%
\subsection{Dynamic of Community Members}
%%%%%%%%%%%%%%%%%%%%%%%%%%%%%%%%%%%%%%%%%%%%%%%%%%%%%%%%%%%%%%%%%%%%%%%%%%%%%%%
Figure~\ref{fig:membersLE} (leading eigenvector) and figure~\ref{fig:membersWT} (walktrap) show the flow of  community members between consecutive snapshots.
For leading eigenvector communities, the majority of the bees stay in their age group, and a small fraction of bees switches to older communities.
Only a few bees change to younger communities.
The new middle-aged communities CM2 and CM3, detected by walktrap, consist partly of young (CY1) and old (CO1) bees. The switching behavior of individuals between communities is similar to leading eigenvector.

Individual bees change communities as they age.

\begin{figure}[htb]
	\centering
	\begin{subfigure}[b]{1.0\textwidth}
	\centering
	\includegraphics[width=.8\textwidth]{Figures/le_matching}
	\caption[Leading eigenvector communities]{Leading eigenvector communities}
	\label{fig:membersLE}
	\vspace*{5mm}
	\end{subfigure} 
	\begin{subfigure}[b]{1.0\textwidth}
	\centering
	\includegraphics[width=.8\textwidth]{Figures/wt_matching}
	\caption[Walktrap communities]{Walktrap communities}
	\label{fig:membersWT}
	\vspace*{5mm}
	\end{subfigure}
	\caption[Dynamic community members]{\textbf{Dynamic community members} 
	Each column represents a time step, the colored rectangles represent the communities for each time step, and the height of the rectangles corresponds to the number of its community members, as referenced by the number. \emph{Green} indicates the community containing young bees and the queen, \emph{gray} represents the community containing middle-aged bees (only for walktrap), and \emph{orange} the community containing old bees. This figure shows that the major part of the members eighter stay in the same aged community or switch to an older group.}
	\label{fig:members}
\end{figure}

%%%%%%%%%%%%%%%%%%%%%%%%%%%%%%%%%%%%%%%%%%%%%%%%%%%%%%%%%%%%%%%%%%%%%%%%%%%%%%%
%%%%%%%%%%%%%%%%%%%%%%%%%%%%%%%%%%%%%%%%%%%%%%%%%%%%%%%%%%%%%%%%%%%%%%%%%%%%%%%
\section{Discussion of Results}
%%%%%%%%%%%%%%%%%%%%%%%%%%%%%%%%%%%%%%%%%%%%%%%%%%%%%%%%%%%%%%%%%%%%%%%%%%%%%%%
%%%%%%%%%%%%%%%%%%%%%%%%%%%%%%%%%%%%%%%%%%%%%%%%%%%%%%%%%%%%%%%%%%%%%%%%%%%%%%%

following part summarizes the presented results in relation to the research goals, listed in section~\ref{sec:intro:goals}\\
discusses the results per goal\\
implications towards goal 1 (inferring temporal networks)\\


%%%%%%%%%%%%%%%%%%%%%%%%%%%%%%%%%%%%%%%%%%%%%%%%%%%%%%%%%%%%%%%%%%%%%%%%%%%%%%%
\subsection{Network Topology and Characteristics of the Honey Bee}
%%%%%%%%%%%%%%%%%%%%%%%%%%%%%%%%%%%%%%%%%%%%%%%%%%%%%%%%%%%%%%%%%%%%%%%%%%%%%%%
\emph{What kind of worker-worker interaction networks emerge and how are they structured?
What is their topology?
What properties are characteristic and how do they differ from randomly generated networks?}

The investigated honey bee spatial proximity networks are characterized by a high density (69\%, 54\%, 61\%).
Apparently, bees encounter many nestmates during the ten hours of data aggregation, either they are very active, or the comb is very crowded, and so the probability that two bees are in proximity is very likely.\\
Comparing to the ant contact networks of \textcite{mersch2013tracking} ($D=72\%\pm5.3$), the values are similar.
As opposed to \textcite{baracchi2014socio} ($D=0.15$) the density is higher, probably due to their lower observation resolution of one frame per minute.

The small diameter ($d_{\texttt{max}}=3$) of my investigated networks and the small average shortest path of $1.4$ in combination with a high global clustering coefficient (0.79, 0.72, 0.75) are characteristic for a class of networks known as small world networks.
This type of networks allows for rapid and efficient communication between bees.

\textcite{charbonneau2013social} state that many biological networks, including insect colonies, are thought to approximate scale-free networks, and for many biological networks the scale-free property has been shown, but for social insect networks there is no clear answer yet. The author's reasons that this is because investigated social insect colonies are often small and therefore the methods to detect scale-free phenomena are limited. They do not further specify the type of social insect networks, whether they mean, interaction networks base on spatial proximity, physical contacts of food transfer.
The size of the network I explored is large compared to present studies (compare~\ref{ch:relatedwork}).
The degree distribution of the investigated spatial proximity network of honey bees does not follow a power-law. Consequently, hubs are absent and accordingly a non-hierarchical structure is typical for this network.
This result corresponds to the decentralized structure of a honey bee colony, and the absence of a central authority described by~\textcite{seeley1989honey}.

I observed a correlation between the detection frequency of a bee, its age, and its corresponding network measure value. Older bees are detected less often than younger bees and therefore differ regarding their network measures.
\textcite{baracchi2014socio} also assumed that the time bees spend outside the hive, affects their connectedness within the interaction network and, hence, findings might be trivial.
The age-based task division of bees in a colony observed by \textcite{seeley1989social}, namely old bees are foragers, the middle-aged bees relate to several tasks but mainly they store resources, and young bees are primarily nursing, might be an explanation.
I observed bimodal degree, strength, closeness and betweenness distributions and a right skewed lcc distribution. Those findings could imply two functional groups of bees, related to the age-based division of labour. The first group is older than 45 days and might correspond to foragers spending a lot of time outside the hive, and the second group might correspond to in-hive workers and therefore younger bees.
The study by \textcite{baracchi2014socio} investigated tree predetermindes age cohorts regarding the network property strength. They also conclude that young bees are more connected then old bees.

%%%%%%%%%%%%%%%%%%%%%%%%%%%%%%%%%%%%%%%%%%%%%%%%%%%%%%%%%%%%%%%%%%%%%%%%%%%%%%%
\subsection{Characterization of Community Structure}
%%%%%%%%%%%%%%%%%%%%%%%%%%%%%%%%%%%%%%%%%%%%%%%%%%%%%%%%%%%%%%%%%%%%%%%%%%%%%%%
\emph{Does the network display a meaningful community structure?
How are the identified communities characterized?
Do they reflect already known colony behavior concerning age and spatial distribution?}

According to the definition of communities in section~\ref{subsec:bg:communities}, I found two to three groups, depending on the used algorithm.
The algorithms (leading eigenvector and walktrap) detected communities, despite the high density and without thresholding edges with low values, as opposed to~\textcite{mersch2013tracking}. They had to artificially reduce the network's density to 25\% for beeing able to apply the infomap algorithm.

Similar to the findings of \textcite{baracchi2014socio}, I also explored the spatial fidelity of three groups. The young bees are located close to the brood (upper center of the comb),  the old bees are situated closer to the hive exit, and (3) the middle-aged bees are placed between the two groups and around the brood, where the cells for honey storage are positioned.

It is surprising that the results align with \textcite{baracchi2014socio}, although they did not use a community detection algorithm. The authors conducted a hierarchical clustering method based on the network measures strength, eigenvector and betweenness centrality of individual bees.
The two approaches discovered the same functional groups of the bee colony, on the one hand by node level network measures (hierarchical clustering) and on the other hand by a higher than expected density of nodes (community detection).
That acknowledges the existence of the age-based division of labor in honey bee colonies as well as the higher communication frequency within groups than between groups. Nevertheless, the low modularity score indicates that the segregation of groups is not that obvious and strict; therefore much interaction between groups exists.

[TODO pus somewhere]
This finding aligns with ~\textcite{seeley1982adaptive} and \textcite{johnson2008within}: workers change tasks over the course of their lifetime, starting as nurses in the nest and ending as foragers outside.
\textcite{johnson2008within} observed two within-nest bees: young bees brood care tasks, middle-aged bees specialized on nectar processing and nest maintenance.
\textcite{seeley1982adaptive} disticts four age subcasts among workers: cell cleaning, broodnest, food storage, forager.

%%%%%%%%%%%%%%%%%%%%%%%%%%%%%%%%%%%%%%%%%%%%%%%%%%%%%%%%%%%%%%%%%%%%%%%%%%%%%%%
\subsection{Dynamics of Community Members}
%%%%%%%%%%%%%%%%%%%%%%%%%%%%%%%%%%%%%%%%%%%%%%%%%%%%%%%%%%%%%%%%%%%%%%%%%%%%%%%
\emph{How do these communities develop over time?
Are they stable regarding their properties?
How do members move between communities?}

I inspected three snapshots over a period of five days and found out that the detected communities are stable over time. Age-division and spatial fidality can be observed in all the snapshot.
Bees from younger communities move to older communities as they age. Only a few bees change from older to younger communities.
This finding aligns with ~\textcite{seeley1982adaptive} and \textcite{johnson2008within}: workers change tasks over the course of their lifetime, starting as nurses in the nest and ending as foragers outside.

\textcite{mersch2013tracking} revealed that the behavioural maturation of ants is a slow and noisy process. Instead of investigating the transistion of individuals daywise, they grouped 41 days in four periods. For each period they asigned each ant to a community if it was found in this community 70\% of the time.
It seems that honey bee transitions are in contrast to ants fast and smoother.
\chapter{Conclusion}
\label{ch:conclusion}

inferred time-aggregated networks out of high resolution tracking data of one honey bee colony\\
weighted undirected spatial proximity networks at three consecutive timesteps over a period of five days\\
analysis regarding network topology, community structures and development of community members\\

the small world characteristic allows for efficient communication within the bee colony\\
it is not a scale-free networks, as opposed to other real-wordl networks\\
the topology of the network is non-hierarchical, therefor no central bees in the sense of degree, strength, and centrality measure like closeness and betweenness exists\\
older bees are generally less detected than younger bees\\
position in the network also depends on the age\\
detected communities relate to age based groups with a spatial fidality towards regions of the comb, explanation: temporal polyethism\\
also bees move to older functional groups as they age\\


stable global non-hierarchical structure of the colony over time (topology, distribution of measures, communities)\\
but dynamic local structure nodes/bees change communities as they age\\
is aligned with previous studies and confirms research results of others (no central authority, dezentralized, temporal polyethism)\\


verifies my definition of the networks and the network pipeline with the chosen network parameters (maximal distance and minimum contact duration)\\
therefore analysis results validate the first goal: inferrence of time-aggregated networks seems to be ok\\
network pipeline is suitable for further research, good foundation, maybe with some modifications, depending on the aimed goals, fine tuning\\


\section{Limitations}
regarding my methods, concerning implementation\\
dataset: quality, kind of a bit complex preprocessing (syncing cameras, removing invalid data, confidence), but still dead bess and almost dead bees in network\\
definition of interaction - minimum contact duration maybe was not the best idea\\
type of network spatial proximity (only proxy for interaction, no real interaction) -> maybe better contact (angle of bees similar to Mersch)/food-exchange network (interaction events), too much noise in the data\\


age distribution has some gaps (tagging not on weekends)-> not sure about it effects\\
context, missing domain knowledge, working in an interdisziplinary teams\\

\section{Future Work}
(1) recommendations for further studies, (2) recommendation for change\\
each recommendation should directly trace a direct conclusion\\

investigate: correlation of density and time, upper bound? or 100\% density after some X frames?\\
are communities robust regarding pipeline parameters maximum distance, minimum contact duration, window size (yes they are, tested for some values, but not systhematically)\\
pipelie parameters (maximum distance, minimum contact duration, window size) effects on network properties (degree, strength,lcc, edgeweights) ... tested but not systematically only some values and only window size up to one hour, should be done more systematically\\
compare the networks to random geometric graph~\cite{rgg2002} and model a walking bee as a random walker\\
so far nodes are attributed with only age and detection frequency, add other stuff, e.g. average speed, total distance traveled\\
compute centrality measures per community\\
weighted edges, should also inpect weighted versions of node measures (this was a bit to complicated, already implemented weighted version in networkX and igraoh do favour weights over number of hops, should look at~\textcite{opsahl2010node}, but depending on what you want to investigate you have to choose the $\alpha$)\\
compare analysis results to NON artifical bees networks\\
investigate different granularities (size of time window)\\
study longer time period not only 5 days\\
normalized edge weight (SRI - Simple Ratio Index, SRI measures
the proportion of times two individuals were seen
together out of the total number of times those individ-
uals were observed, by \textcite{croft2008exploring}) or normalized degree and stuff by frequency of detection\\
apply network reduction algorithm to reduce density and noise, by applying (no simple edge thresholding), disparity filter~\textcite{serrano2009extracting}

\vspace{2cm}
more data does not mean better analysis, one have to ask good questions, develop specialized hypothesis, a lot of options\\
explorative analysis approach difficult with data from animals I do not know, maybe easier with 'human' data (missing domain data)\\
alienation: automatic tracking -> no relation to the data or how the data was collected or animals\\
just applying network method to new research fields, sometimes just replicates known facts, Ending with: Zitat Krause (Buch)
%---------------------------------------------------
%----- Bibliography
%---------------------------------------------------
\printbibliography

%---------------------------------------------------
%----- Directories   
%---------------------------------------------------

\addcontentsline{toc}{chapter}{\listfigurename}
\listoffigures
\clearpage %\cleardoublepage %for openright
\addcontentsline{toc}{chapter}{\listtablename}
\listoftables
\clearpage %\cleardoublepage %for openright
%\lstlistoflistings

%---------------------------------------------------
%----- Appendix   
%---------------------------------------------------
\appendix
\chapter{Appendix Stuff}
\label{ch:appendix}

\begin{table}
\centering
\caption[XXX]{\textbf{XXX} \url{https://docs.google.com/spreadsheets/d/1eKuPU-XmqwrHkS_5-TgS8UnO5O-Hwe1kyRIpareywP4/edit?usp=sharing}}
\label{tab:studies}
\vspace*{5mm}
\begin{tabular}{ccc}
	\toprule
	{}  & TODO & TODO \\
	\midrule

	x & x & x\\
	x & x & x\\
	\bottomrule
\end{tabular}
\end{table}

\begin{table}
\centering
\caption[Network measures of studies]{\textbf{Network measures of studies} \url{https://docs.google.com/spreadsheets/d/1eKuPU-XmqwrHkS_5-TgS8UnO5O-Hwe1kyRIpareywP4/edit?usp=sharing}}
\label{tab:studies-measures}
\vspace*{5mm}
\begin{tabular}{ccc}
	\toprule
	{}  & TODO & TODO \\
	\midrule

	x & x & x\\
	x & x & x\\
	\bottomrule
\end{tabular}
\end{table}

\begin{table}
\centering
\caption[Network types of studies]{\textbf{Network types of studies} \url{https://docs.google.com/spreadsheets/d/1eKuPU-XmqwrHkS_5-TgS8UnO5O-Hwe1kyRIpareywP4/edit?usp=sharing}}
\label{tab:studies-nwtype}
\vspace*{5mm}
\begin{tabular}{ccc}
	\toprule
	{}  & TODO & TODO \\
	\midrule

	x & x & x\\
	x & x & x\\
	\bottomrule
\end{tabular}
\end{table}

\begin{figure}[htb]
	\centering
	\includegraphics[width=1.0\textwidth]{Figures/study-measures}
	\caption[XXX]{\textbf{XXX} XXX}
	\label{fig:study-measures}
\end{figure}

\begin{figure}[htb]
	\centering
	\includegraphics[width=1.0\textwidth]{Figures/study-nwtype}
	\caption[XXX]{\textbf{XXX} XXX}
	\label{fig:study-nwtype}
\end{figure}

\begin{figure}[htb]
	\centering
	\includegraphics[width=1.0\textwidth]{Figures/study-study}
	\caption[XXX]{\textbf{XXX} XXX}
	\label{fig:study-study}
\end{figure}


\begin{figure}[htb]
	\centering
	\includegraphics[width=1.0\textwidth]{Figures/tagging_period}
	\caption[Tagging frequency]{\textbf{Tagging frequency} The bees were primarily tagged during the week. On average 48 bees were tagged each day, considering only tagging days, the average is about 91. [TODO: combine with other image or make nicer!]}
	\label{fig:tagging-period}
\end{figure}

\begin{figure}[htb]
	\centering
	\includegraphics[width=1.0\textwidth]{Figures/recording}
	\caption[Recording season with maintainance and failures]{\textbf{Recording season with maintainance and failures} \emph{Green} indicates recording went without any big interruption; \emph{Yellow} indicates maintainance work or technical failures of one or all cameras. This is calculated using the expected number of files produced by each camera per hour. [TODO, reduzieren auf eine Info pro Tag (keine stuendliche aufloesung), kombinieren mit anzahl der getaggten bienen pro tag, und welchen Zeitraum hab ich nun verwendet], ausserdem Zeit von links nach rechts!, evtl. kein Datum, sonder Tage durchnummerieren}
	\label{fig:observation-period}
\end{figure}



\section{Network Analysis}
\label{app-analysis}

\begin{figure}[htb]
	\centering
	\begin{subfigure}[b]{1.0\textwidth}
		\centering
		\includegraphics[width=\textwidth]{Figures/le_network1}
		\caption[Network 1]{Network 1}
		\label{fig:le1}
		\vspace*{5mm}
	\end{subfigure}
	%\vspace{1cm} 
	\begin{subfigure}[b]{1.0\textwidth}
		\includegraphics[width=\textwidth]{Figures/le_network2}
		\caption[Network 2]{Network 2}
		\label{fig:le2}
		\vspace*{5mm}
	\end{subfigure}
	%\vspace{1cm} 
	\begin{subfigure}[b]{1.0\textwidth}
		\includegraphics[width=\textwidth]{Figures/le_network3}
		\caption[Network 3]{Network 3}
		\label{fig:le3}
		\vspace*{5mm}
	\end{subfigure}
	\caption[Communities per network - leading eigenvector]{\textbf{Communities per network - leading eigenvector} The \emph{green} colour represents the younger community, containing the queen. The \emph{orange} color represents the older community. The hive exit on side A is on the bottom right and on side B on the bottom left. The data is aggredated for the complete timeframe of ten hours.}
	\label{fig:communitiesPerNetworkLE}
\end{figure}

\begin{figure}[htb]
	\centering
	\begin{subfigure}[b]{1.0\textwidth}
		\centering
		\includegraphics[width=\textwidth]{Figures/wt_network1}
		\caption[Network 1]{Network 1}
		\label{fig:wt1}
		\vspace*{5mm}
	\end{subfigure}
	%\vspace{1cm} 
	\begin{subfigure}[b]{1.0\textwidth}
		\includegraphics[width=\textwidth]{Figures/wt_network2}
		\caption[Network 2]{Network 2}
		\label{fig:wt2}
		\vspace*{5mm}
	\end{subfigure}
	%\vspace{1cm} 
	\begin{subfigure}[b]{1.0\textwidth}
		\includegraphics[width=\textwidth]{Figures/wt_network3}
		\caption[Network 3]{Network 3}
		\label{fig:wt3}
		\vspace*{5mm}
	\end{subfigure}
	\caption[Communities per network - walktrap]{\textbf{Communities per network - walktrap} The \emph{green} colour represents the younger community, containing the queen. The \emph{orange} color represents the older community. The \emph{gray} represents the middle-age community. The hive exit on side A is on the bottom right and on side B on the bottom left. The data is aggredated for the complete timeframe of ten hours.}
	\label{fig:communitiesPerNetworkWT}
\end{figure}

\begin{figure}[htb]
	\centering
	\begin{subfigure}[b]{1.0\textwidth}
		\centering
		\includegraphics[width=1.0\textwidth]{Figures/ageDistribution-LE}
		\caption[Leading eigenvector]{Leading eigenvector}
		\label{fig:ageLE}
		\vspace*{5mm}
	\end{subfigure}
	%\vspace{1cm} 
	\begin{subfigure}[b]{1.0\textwidth}	
		\centering
		\includegraphics[width=1.0\textwidth]{Figures/ageDistribution-WT}
		\caption[Walktrap]{Walktrap}
		\label{fig:ageWT}
		\vspace*{5mm}
	\end{subfigure}
	%\vspace{1cm}
	\caption[Age distribution for each community and network] {\textbf{Age distribution for each community and network} The \emph{green} bar is the community containing the queen. The queens age is not included in the statistic. The \emph{orange} bars coresspond to the second community, containing older bees. The \emph{gray} bars is a third community only revealed by walktrap and contains middle-aged bees.}
	\label{fig:ageDistribution}
\end{figure}

\begin{figure}[!t]
	\centering
	\begin{subfigure}[b]{1.0\textwidth}
	\centering
	\includegraphics[width=0.92\textwidth]{Figures/stat-degreeDist}
	\caption[Degree distribution]{Degree distribution}
	\label{fig:statDegreeDist}
	\end{subfigure} 
	\begin{subfigure}[b]{1.0\textwidth}
	\centering
	\includegraphics[width=0.92\textwidth]{Figures/stat-strengthDist}
	\caption[Strength distribution]{Strength distribution}
	\label{fig:statStrengthDist}
	\end{subfigure}
	\begin{subfigure}[b]{1.0\textwidth}
	\centering
	\includegraphics[width=0.92\textwidth]{Figures/stat-lccDist}
	\caption[Local clustering coefficient]{Local clustering coefficient}
	\label{fig:statlccDist}
	\end{subfigure}
	\begin{subfigure}[b]{1.0\textwidth}
	\centering
	\includegraphics[width=0.92\textwidth]{Figures/stat-edgeWeightDist}
	\caption[Edge weight distribution]{Edge weight distribution}
	\label{fig:statEdgeWeightDist}
	\end{subfigure}
	\caption[Degree, strength and edge weight distribution]{\textbf{Degree, strength and edge weight distribution} for all three networks.}
	\label{fig:distributions}
\end{figure}

\begin{figure}[htb]
	\centering
	\includegraphics[width=1.0\textwidth]{Figures/ages}
	\caption[Age distribution per network]{\textbf{Age distribution per network} The width of a bar corresponds to two days. For each network bees with a negative age and the queen were removed (11, 10, and 9 bees).}
	\label{fig:ages}
\end{figure}
\backmatter

\end{document}