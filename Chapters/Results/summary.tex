%%%%%%%%%%%%%%%%%%%%%%%%%%%%%%%%%%%%%%%%%%%%%%%%%%%%%%%%%%%%%%%%%%%%%%%%%%%%%%%
%%%%%%%%%%%%%%%%%%%%%%%%%%%%%%%%%%%%%%%%%%%%%%%%%%%%%%%%%%%%%%%%%%%%%%%%%%%%%%%
\section{Discussion of Results}
%%%%%%%%%%%%%%%%%%%%%%%%%%%%%%%%%%%%%%%%%%%%%%%%%%%%%%%%%%%%%%%%%%%%%%%%%%%%%%%
%%%%%%%%%%%%%%%%%%%%%%%%%%%%%%%%%%%%%%%%%%%%%%%%%%%%%%%%%%%%%%%%%%%%%%%%%%%%%%%

following part summarizes the presented results in relation to the research goals, listed in section~\ref{sec:intro:goals}\\
discusses the results per goal\\
implications towards goal 1 (inferring temporal networks)\\


%%%%%%%%%%%%%%%%%%%%%%%%%%%%%%%%%%%%%%%%%%%%%%%%%%%%%%%%%%%%%%%%%%%%%%%%%%%%%%%
\subsection{Network Topology and Characteristic Properties}
%%%%%%%%%%%%%%%%%%%%%%%%%%%%%%%%%%%%%%%%%%%%%%%%%%%%%%%%%%%%%%%%%%%%%%%%%%%%%%%
\emph{Goal 2:
What kind of worker-worker interaction networks emerge and how are they structured?
What is their topology?
What properties are characteristic and how do they differ from randomly generated networks?}

A) high density 69\%, 54\%, 61\%\\
cumulated data for 10 hours\\
compared to \textcite{mersch2013tracking} it is similar 72\%$\pm5.3$, but they reduced density to 25\%, ants\\
compared to \textcite{baracchi2014socio} (0.15) the density is very high, but they only had 1 frame per minute and 10 hours of data\\
also one should not compare networks created by different techniques~\cite{castles2014social}\\

B) small world type and no scale free\\
average shortest path $1.4$,
gcc higher than random,
diameter 3, random 2\\
short path/way for communication and transfer of something\\
rapid communication between nodes\\
many biological networks, including insect colonies are thought to approximate scale-free networks, degree follows power law and hubs, also belong to small world class networks~\textcite{holme2013temporalbook} (!)\\

C) no hierarcical structure, dezentralized, no hubs, no central nodes\\
no central authority, dezentralized controle~\textcite{seeley1989honey}\\

D) correlation: detection frequency, age, network measures\\
foragers outside, affects their connectedness~\textcite{baracchi2014socio}\\
division of labor: old (foragers), middle-aged (storers), young (nursing, according to \textcite{seeley1989social}\\
age based task division, division of labout\\
temporal polyethism (Lindauer/Rosch/Sagakami)\\

E) bimodal distribution of network measures imply 2 groups of bees\\
old group > 45, high lcc, rest low, rest of the bees: low lcc, rest high\\
resuls from detection frequency\\
no studies showing distributions, just \cite{baracchi2014socio}, 3 age groups, no exact age and network measure distributions\\


%%%%%%%%%%%%%%%%%%%%%%%%%%%%%%%%%%%%%%%%%%%%%%%%%%%%%%%%%%%%%%%%%%%%%%%%%%%%%%%
\subsection{Characterization of Community Structure}
%%%%%%%%%%%%%%%%%%%%%%%%%%%%%%%%%%%%%%%%%%%%%%%%%%%%%%%%%%%%%%%%%%%%%%%%%%%%%%%
\emph{Goal 3:
Does the network display a meaningful community structure?
How are the identified communities characterized?
Do they reflect already known colony behavior concerning age and spatial distribution?}

according to the definition of communities (we call a community a group of nodes that have a higher likelihood of connecting to each other than to nodes from other communities, leading eigenvector and walktrap optimize modularity), found 2 to 3 groups depending on algorithm\\
found communities, despite high density, without thesholding edges with low values, as opposed to~\textcite{mersch2013tracking}\\
spacial fidelity (fidelity ist echt ein komisches Wort): young communitiy located in upper center of comb (brood), old community close to the hive exit, middle-aged are placed inbetween and around the young group (maybe pollen,nectar,honey?)\\
similar to the results of \textcite{mersch2013tracking} location of ant communities also close to brood, inbetween and close to exit, but ant colonies insted of honey bees, method both community detection, but they used infomap, I used LE and WT\\
also very similar to results of \textcite{baracchi2014socio}: foraging bees close to exit (old bees), nurses close to brood and middle-aged bees inbetween\\
but they used some hierarchical clustering (clustered based on network measures: strength, eigenvector and betweenness with average linkage, squared euclidian distance) and no community detection algorithm\\
interaction frequency vergleichen mit~\textcite{scholl2011olfactory}, young, middle, old\\


%%%%%%%%%%%%%%%%%%%%%%%%%%%%%%%%%%%%%%%%%%%%%%%%%%%%%%%%%%%%%%%%%%%%%%%%%%%%%%%
\subsection{Dynamics of Community Members}
%%%%%%%%%%%%%%%%%%%%%%%%%%%%%%%%%%%%%%%%%%%%%%%%%%%%%%%%%%%%%%%%%%%%%%%%%%%%%%%
\emph{Goal 4:
How do these communities develop over time?
Are they stable regarding their properties?
How do members move between communities?}

per snapshot 2-3 age groups\\
per snapshot groups in different region of the comb: center (brood), close to exit (foragers?), inbetween and outer parts (storing stuff?) (inspected images of the comb)\\
bees from younger communities move to older communities as they age, not the other way around (only very few older bees change back to younger groups)

in honey bees workers change tasks over the course of their lifetime, starting as nurses in the nest and ending as foragers outside~\cite{seeley1982adaptive,johnson2008within}\\
\textcite{johnson2008within} observed two within-nest bees: young bees brood care tasks, middle-aged bees specialized on nectar processing and nest maintenance\\
\textcite{seeley1982adaptive} 5 female casts: queen (reproductive cast), 4 age subcasts among workers: cell cleaning, broodnest, food storage, forager

difference method: transitions for each day over a three day period (spread over 5 days), \textcite{mersch2013tracking} grouped 11/10 days together and then investigated transitions of ants, 70\% threshold, age related  behavioural maturation is a slow and noisy process in ants with important individual variations~\cite{mersch2013tracking}\\
honey bees instead kind of fast and not that noisy\\

%%%%%%%%%%%%%%%%%%%%%%%%%%%%%%%%%%%%%%%%%%%%%%%%%%%%%%%%%%%%%%%%%%%%%%%%%%%%%%%
\subsection{Implications}
%%%%%%%%%%%%%%%%%%%%%%%%%%%%%%%%%%%%%%%%%%%%%%%%%%%%%%%%%%%%%%%%%%%%%%%%%%%%%%%
1. - stable global structure of the colony over time\\
a) stable network properties: distribution of network measures (degree, strength, lcc, $C_B$ and $C_C$ stays the same over the three snapshots\\
b) communities stay the same over the snapshots (age groups, spatial groups)

2. - dynamic local structure (node level, individual bee): bees change communities as they age\\

as expected from honey bee colonies, confirms research results of others\\
verifies my definition of the networks (network pipeline) and chosen network parameters, validates the first goal: inferrence of temporal networks\\