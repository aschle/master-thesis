%%%%%%%%%%%%%%%%%%%%%%%%%%%%%%%%%%%%%%%%%%%%%%%%%%%%%%%%%%%%%%%%%%%%%%%%%%%%%%%
%%%%%%%%%%%%%%%%%%%%%%%%%%%%%%%%%%%%%%%%%%%%%%%%%%%%%%%%%%%%%%%%%%%%%%%%%%%%%%%
\section{Discussion of Results}
%%%%%%%%%%%%%%%%%%%%%%%%%%%%%%%%%%%%%%%%%%%%%%%%%%%%%%%%%%%%%%%%%%%%%%%%%%%%%%%
%%%%%%%%%%%%%%%%%%%%%%%%%%%%%%%%%%%%%%%%%%%%%%%%%%%%%%%%%%%%%%%%%%%%%%%%%%%%%%%

following part summarizes the presented results in relation to the research goals, listed in section~\ref{sec:intro:goals}\\
discusses the results per goal\\
implications towards goal 1 (inferring temporal networks)\\


%%%%%%%%%%%%%%%%%%%%%%%%%%%%%%%%%%%%%%%%%%%%%%%%%%%%%%%%%%%%%%%%%%%%%%%%%%%%%%%
\subsection{Network Topology and Characteristics of the Honey Bee}
%%%%%%%%%%%%%%%%%%%%%%%%%%%%%%%%%%%%%%%%%%%%%%%%%%%%%%%%%%%%%%%%%%%%%%%%%%%%%%%
\emph{What kind of worker-worker interaction networks emerge and how are they structured?
What is their topology?
What properties are characteristic and how do they differ from randomly generated networks?}

The investigated honey bee spatial proximity networks are characterized by a high density (69\%, 54\%, 61\%).
Apparently, bees encounter many nestmates during the ten hours of data aggregation, either they are very active, or the comb is very crowded, and so the probability that two bees are in proximity is very likely.\\
Comparing to the ant contact networks of \textcite{mersch2013tracking} ($D=72\%\pm5.3$), the values are similar.
As opposed to \textcite{baracchi2014socio} ($D=0.15$) the density is higher, probably due to their lower observation resolution of one frame per minute.

The small diameter ($d_{\texttt{max}}=3$) of my investigated networks and the small average shortest path of $1.4$ in combination with a high global clustering coefficient (0.79, 0.72, 0.75) are characteristic for a class of networks known as small world networks.
This type of networks allows for rapid and efficient communication between bees.

\textcite{charbonneau2013social} state that many biological networks, including insect colonies, are thought to approximate scale-free networks, and for many biological networks the scale-free property has been shown, but for social insect networks there is no clear answer yet. The author's reasons that this is because investigated social insect colonies are often small and therefore the methods to detect scale-free phenomena are limited. They do not further specify the type of social insect networks, whether they mean, interaction networks base on spatial proximity, physical contacts of food transfer.
The size of the network I explored is large compared to present studies (compare~\ref{ch:relatedwork}).
The degree distribution of the investigated spatial proximity network of honey bees does not follow a power-law. Consequently, hubs are absent and accordingly a non-hierarchical structure is typical for this network.
This result corresponds to the decentralized structure of a honey bee colony, and the absence of a central authority described by~\textcite{seeley1989honey}.

I observed a correlation between the detection frequency of a bee, its age, and its corresponding network measure value. Older bees are detected less often than younger bees and therefore differ regarding their network measures.
\textcite{baracchi2014socio} also assumed that the time bees spend outside the hive, affects their connectedness within the interaction network and, hence, findings might be trivial.
The age-based task division of bees in a colony observed by \textcite{seeley1989social}, namely old bees are foragers, the middle-aged bees relate to several tasks but mainly they store resources, and young bees are primarily nursing, might be an explanation.
I observed bimodal degree, strength, closeness and betweenness distributions and a right skewed lcc distribution. Those findings could imply two functional groups of bees, related to the age-based division of labour. The first group is older than 45 days and might correspond to foragers spending a lot of time outside the hive, and the second group might correspond to in-hive workers and therefore younger bees.
The study by \textcite{baracchi2014socio} investigated tree predetermindes age cohorts regarding the network property strength. They also conclude that young bees are more connected then old bees.

%%%%%%%%%%%%%%%%%%%%%%%%%%%%%%%%%%%%%%%%%%%%%%%%%%%%%%%%%%%%%%%%%%%%%%%%%%%%%%%
\subsection{Characterization of Community Structure}
%%%%%%%%%%%%%%%%%%%%%%%%%%%%%%%%%%%%%%%%%%%%%%%%%%%%%%%%%%%%%%%%%%%%%%%%%%%%%%%
\emph{Does the network display a meaningful community structure?
How are the identified communities characterized?
Do they reflect already known colony behavior concerning age and spatial distribution?}

According to the definition of communities in section~\ref{subsec:bg:communities}, I found two to three groups, depending on the used algorithm.
The algorithms (leading eigenvector and walktrap) detected communities, despite the high density and without thresholding edges with low values, as opposed to~\textcite{mersch2013tracking}. They had to artificially reduce the network's density to 25\% for beeing able to apply the infomap algorithm.

Similar to the findings of \textcite{baracchi2014socio}, I also explored the spatial fidelity of three groups. The young bees are located close to the brood (upper center of the comb),  the old bees are situated closer to the hive exit, and (3) the middle-aged bees are placed between the two groups and around the brood, where the cells for honey storage are positioned.

It is surprising that the results align with \textcite{baracchi2014socio}, although they did not use a community detection algorithm. The authors conducted a hierarchical clustering method based on the network measures strength, eigenvector and betweenness centrality of individual bees.
The two approaches discovered the same functional groups of the bee colony, on the one hand by node level network measures (hierarchical clustering) and on the other hand by a higher than expected density of nodes (community detection).
That acknowledges the existence of the age-based division of labor in honey bee colonies as well as the higher communication frequency within groups than between groups. Nevertheless, the low modularity score indicates that the segregation of groups is not that obvious and strict; therefore much interaction between groups exists.

[TODO pus somewhere]
This finding aligns with ~\textcite{seeley1982adaptive} and \textcite{johnson2008within}: workers change tasks over the course of their lifetime, starting as nurses in the nest and ending as foragers outside.
\textcite{johnson2008within} observed two within-nest bees: young bees brood care tasks, middle-aged bees specialized on nectar processing and nest maintenance.
\textcite{seeley1982adaptive} disticts four age subcasts among workers: cell cleaning, broodnest, food storage, forager.

%%%%%%%%%%%%%%%%%%%%%%%%%%%%%%%%%%%%%%%%%%%%%%%%%%%%%%%%%%%%%%%%%%%%%%%%%%%%%%%
\subsection{Dynamics of Community Members}
%%%%%%%%%%%%%%%%%%%%%%%%%%%%%%%%%%%%%%%%%%%%%%%%%%%%%%%%%%%%%%%%%%%%%%%%%%%%%%%
\emph{How do these communities develop over time?
Are they stable regarding their properties?
How do members move between communities?}

I inspected three snapshots over a period of five days and found out that the detected communities are stable over time. Age-division and spatial fidality can be observed in all the snapshot.
Bees from younger communities move to older communities as they age. Only a few bees change from older to younger communities.
This finding aligns with ~\textcite{seeley1982adaptive} and \textcite{johnson2008within}: workers change tasks over the course of their lifetime, starting as nurses in the nest and ending as foragers outside.

\textcite{mersch2013tracking} revealed that the behavioural maturation of ants is a slow and noisy process. Instead of investigating the transistion of individuals daywise, they grouped 41 days in four periods. For each period they asigned each ant to a community if it was found in this community 70\% of the time.
It seems that honey bee transitions are in contrast to ants fast and smoother.