%%%%%%%%%%%%%%%%%%%%%%%%%%%%%%%%%%%%%%%%%%%%%%%%%%%%%%%%%%%%%%%%%%%%%%%%%%%%%%%
%%%%%%%%%%%%%%%%%%%%%%%%%%%%%%%%%%%%%%%%%%%%%%%%%%%%%%%%%%%%%%%%%%%%%%%%%%%%%%%
\section{Static Perspectives of Honey Bee Networks}
%%%%%%%%%%%%%%%%%%%%%%%%%%%%%%%%%%%%%%%%%%%%%%%%%%%%%%%%%%%%%%%%%%%%%%%%%%%%%%%
%%%%%%%%%%%%%%%%%%%%%%%%%%%%%%%%%%%%%%%%%%%%%%%%%%%%%%%%%%%%%%%%%%%%%%%%%%%%%%%
The networks are examindes on the levels.
First of all I examine the networks' global structure and derive properties of the overall colony (global level). Second I study the characteristics of individual bees (local level), and it's relation to detection frequency and age.
Additionally, I investigate the intermediate level of the colonies social organization by detecting communities and inspecting their practical meaning.\\

I analyzed a temporal network, consisting of three time-aggregated snapshots; these are referred to below as snapshot~1~($N=922$), snapshot~2~($N=978$) and snapshot~3~($N=922$). 
The snapshots are aggregated for ten hours (108,000 frames) starting at 8~a.m. and lasting until 6~p.m, see table~\ref{tab:networks} for details about the added bees per day. Figure~\ref{fig:network-matching} shows the proportion of intersecting bees between consecutive snapshot. This figure illustrates the stability of the network concerning its size. 

\begin{table}[htb]
\small
\centering
\caption[Sampling period]{\textbf{Sampling period} Overview of the chosen aggregated daily snapshots including the number of added bees and the time they were added to the hive.}
\vspace*{5mm}
\begin{tabularx}{\textwidth}{ccccccc}
\toprule
{} & 20.08.16 & 21.08.16 & 22.08.16 & 23.08.16 & 24.08.16 \\
\midrule
Snapshot ID & 1 & - & 2 & - & 3 & \\
Number of added bees & 0 & 0 & 110 & 60 & 0 \\
Time added & - & - & 2~p.m. & 6~p.m. & - \\
\bottomrule
\end{tabularx}
\label{tab:networks}
\end{table}

\begin{figure}[htb]
	\centering
	\includegraphics[width=1.0\textwidth]{Figures/network_matching}
	\caption[Number of bees per snapshot]{\textbf{Number of bees per snapshot} [TODO: bessere Beschreibung]This figure shows the amount of bees for each snapshot and the proportion of intersecting.}
	\label{fig:network-matching}
\end{figure}



%%%%%%%%%%%%%%%%%%%%%%%%%%%%%%%%%%%%%%%%%%%%%%%%%%%%%%%%%%%%%%%%%%%%%%%%%%%%%%%
\subsection{Properties of the Bee Colony}
\label{subsec:colony}
%%%%%%%%%%%%%%%%%%%%%%%%%%%%%%%%%%%%%%%%%%%%%%%%%%%%%%%%%%%%%%%%%%%%%%%%%%%%%%%
Each snapshot consists of one component.
The density $D$ is over~50\% for all snapshots (69\%; 54\%; 61\%).
The diameter $\langle d_{\texttt{max}} \rangle$ is~$3$ and the average shortest path length $\langle d \rangle$ is between~$1$ and~$2$.
The global clustering coefficient~(gcc) $c_\Delta$ of all snapshots is higher than compared to an Erd\H{o}s-R\'{e}nyi random graph, averaged over~100 runs using the same number of nodes and links.
On average, each bee is connected to at least~50\% of the colony~(68\%; 52\%; 61\%).
During the ten-hour observation period, a bee interacts over 4,000 times~(5,680; 3,978; 4,206) on average.
Table~\ref{tab:stats} summarizes the basic network properties for each snapshot and lists the values of its corresponding random graph.

For further analysis, I select snapshot~3 because no young bees were added to the colony during that day and, unlike snapshot~1, bees below the age of five days were part of the colony~(Figure~\ref{fig:agesAll}).
Figure~\ref{fig:n3ageDist} shows the age distribution of snapshot~3.
This distribution corresponds to the artificial tagging of the bees.
Consequently, bees of certain age groups are simply not present.
The detection frequency of an individual bee is negatively correlated with its age~(Figure~\ref{fig:n3detfVSage}).
The link weight distribution is shown in Figure~\ref{fig:edgeWdist}.
Most links have a low weight; only a few links have a high weight.
The logarithmized frequency distribution of the link weights appears to be an exponential decay function with an exponent of 0.015. The fitted plot is shown in Figure~\ref{fig:123edgeslog}.

\begin{table}[htb]
\centering
\caption[Global network properties]{\textbf{Global network properties} $N$ is the number of nodes, $L$ the number of edges, $D$ is the diameter, $\langle d_{\texttt{max}} \rangle$ is the average path length, $C_\Delta$ the global clustering coefficient, $C_{\Delta}^\texttt{rand}$ is the global clustering coefficient for randomized graph, $\langle k \rangle$ the average degree and $\langle s \rangle$ represents the average strength, as introduced in section~\ref{sec:definitions}.}
\label{tab:stats}
\vspace*{5mm}
\begin{tabularx}{\textwidth}{lccccccccc}
\toprule
{} &  $N$ &   $L$ &  $D$ &  $\langle d_{\texttt{max}} \rangle$ &  $\langle d \rangle$ &   $C_\Delta$ & $\langle k \rangle$ &  $\langle s \rangle$ \\
\midrule
Snapshot 1 & 922 & 291179 & 0.69 & 3 & 1.32 &  0.79 & 631.62 & 5680.17 \\
Random 1  & 922 & 291179 & 0.69 & 2 & 1.31 &  0.69 & 631.62 & - \\ \midrule
Snapshot 2 & 978 & 256066 & 0.54 & 3 & 1.46 &  0.72 & 523.65 & 3977.94 \\
Random 2  & 978 & 256066 & 0.54 & 2 & 1.46 &  0.54 & 523.65 & - \\ \midrule
Snapshot 3 & 922 & 259421 & 0.61 & 3 & 1.39 &  0.75 & 562.74 & 4205.99 \\
Random 3  & 922 & 259421 & 0.61 & 2 & 1.39 &  0.61 & 562.74 & - \\
\bottomrule
\end{tabularx}
\end{table}

\begin{figure}[bp]
	\centering
	\begin{subfigure}[b]{1\textwidth}
	\centering
	\includegraphics[width=1.0\textwidth]{Figures/n3_detFvsAge}
	\caption[Correlation]{Correlation of detection frequency and age}
	\label{fig:n3detfVSage}
	\vspace{10mm}
	\end{subfigure} 
	\begin{subfigure}[b]{1\textwidth}
	\centering
	\includegraphics[width=1.0\textwidth]{Figures/n3_ages.pdf}
	\caption[Age distribution]{Age distribution}
	\label{fig:n3ageDist}
	\vspace{10mm}
	\end{subfigure}	\begin{subfigure}[b]{1\textwidth}
	\centering
	\includegraphics[width=1.0\textwidth]{Figures/n3-edgeWeightDist.pdf}
	\caption[Link weight distribution]{Link weight distribution (log-log plot)}
	\label{fig:edgeWdist}
	\end{subfigure}
	
	
	\caption[Age distribution, correlation with detection frequency and link weight distribution of snapshot~3]{\textbf{Age distribution, correlation with detection frequency and link weight distribution of snapshot~3} (a) Detection frequency and the age of a honeybee seem to be negatively correlated. (b) The age of bees ranges from $1$ to $60$ days, but some age groups are missing. (c) The link weight as a log-log plot.}
	\label{fig:ageDetF}
\end{figure}
% Characteristics of Individual Bees


Degree of a bee, the number of other bees this bee interacts with.
Strength


Degree, Strength and Local Clustering Coefficient and \\
Betweenness and Closeness Centrality\\
\\
bimodal degree distribution\\
type of network: no scale free\\
todo plot in relation to age of bees\\
todo plot in relation to detection frequency\\

\begin{figure}[!htb]
	\centering
	\begin{subfigure}[b]{1.0\textwidth}
	\centering
	\includegraphics[width=1.0\textwidth]{Figures/n3-stat-degreeStrLCC}
	\caption[Distribution of degree, strength and LCC]{\textbf{Distribution of degree, strength and LCC}}
	\label{fig:n3-d-s-cc}
	\end{subfigure}
	\caption[Degree, strength and local clustering coefficient (LCC)]{\textbf{Degree, strength and local clustering coefficient (LCC)} xxx}
	\label{fig:n3-degreeStrLCC}
\end{figure}

[TODO]\\
in relation to age and detection frequency\\
closeness\\
betweenness\\
%%%%%%%%%%%%%%%%%%%%%%%%%%%%%%%%%%%%%%%%%%%%%%%%%%%%%%%%%%%%%%%%%%%%%%%%%%%%%%%
\subsection{Functional Groups within the Colony}
%%%%%%%%%%%%%%%%%%%%%%%%%%%%%%%%%%%%%%%%%%%%%%%%%%%%%%%%%%%%%%%%%%%%%%%%%%%%%%%
The leading eigenvector community detection algorithms (LE) revealed two communities with a similar size. The walktrap algorithm (WT) discovered three communities instead, also evenly distributed. Table~\ref{tab:n3-communities} lists the precise number of members per community and algorithm.

The communities correspond to different age groups. The young community is 13.15 days old (6.55 days for WT), and the old community is 28.70 days old (29.29 days for WT). The third middle-aged community of WT is 25.08 days old. The age distribution for each algorithm is represented in figure~\ref{fig:n3ageLE} and~\ref{fig:n3ageWT}. The two sample Kolmogorov-Smirnov test confirmed that the age distributions per community are significantly different. The corresponding $p$-values are listed in table~\ref{tab:n3-pvalues2}.

Each community occupies a different region of the comb.
Figure~\ref{fig:n3-communities} shows that the young communities spend the most time in the comb center and the old communities closer to the hive exit. The middle-aged community is positioned between the young and old community and in the periphery of the comb.

\begin{table}
\centering
\caption[Communities per algorithm]{\textbf{Communities per algorithm} Communities marked with * contain the queen. Age and standard deviation (SD) are measured in days. The queen and bees with a negative age (10 bees).}
\label{tab:n3-communities}
\vspace*{5mm}
\begin{tabular}{lcrrrrr}
	\toprule
	{}  & Community ID & Members & Proportion & Age & SD\\
	\midrule  
	\quad LE  & CY & $*381$  & 41.78\% & $13.15$ & $\pm13.50$ \\
	          & CO & $531$   & 58.22\% & $28.70$ & $\pm11.67$ \\
    \midrule 
	\quad WT & CY & $*229$  & 25.11\% & $6.55$  & $\pm10.36$\\
			 & CM & $298$  & 32.68\% & $25.08$ & $\pm11.97$\\
			 & CO & $385$  & 42.21\% & $29.29$ & $\pm11.44$\\
	\bottomrule
\end{tabular}
\end{table}
\begin{table}[htb]
\small
\centering
\caption[Kolmogorov-Smirnov test]{\textbf{Kolmogorov-Smirnov test} $p$-values for leading eigenvector (LE) and walktrap (WT)}
\label{tab:n3-pvalues2}
\vspace*{5mm}
\begin{tabular}{crrrrr}
	\toprule
	 Communities & LE p-value & WT p-value\\
	\midrule 
    CY, CO & 5.10e-66 & 5.51e-67\\
    CY, CM &          & 1.10e-95\\
    CM, CO &          & 1.98e-05\\ 
	\bottomrule
\end{tabular}
\end{table}

\begin{figure}[!htb]
	\centering
	\begin{subfigure}[b]{1.0\textwidth}
	\centering
	\includegraphics[width=1.0\textwidth]{Figures/le_network3}
	\vspace{1pt}
	\end{subfigure} 
	\begin{subfigure}[b]{1.0\textwidth}
	\centering
	\includegraphics[width=1.0\textwidth]{Figures/n3-ageDistribution-LE}
	\caption[Leading eigenvector communities]{Leading eigenvector communities}
	\vspace{5mm}
	\label{fig:n3ageLE}
	\end{subfigure}
	\begin{subfigure}[b]{1.0\textwidth}
	\vspace{1pt}
	\centering
	\includegraphics[width=1.0\textwidth]{Figures/wt_network3}
	\vspace*{1pt}
	\end{subfigure}
	\begin{subfigure}[b]{1.0\textwidth}
	\centering
	\includegraphics[width=1.0\textwidth]{Figures/n3-ageDistribution-WT}
	\caption[Walktrap communities]{Walktrap communities}
	\label{fig:n3ageWT}
	\end{subfigure}
	\caption[Age and spatial distribution of communities]{\textbf{Age and spatial distribution of communities} \emph{Green} represents the young community occupying the center area of the comb and \emph{orange} the old community, which is situated closer to the hive access. For walktrap the \emph{gray} middle-aged community is positioned between the other to and in the periphery of the comb.}
	\label{fig:n3-communities}
\end{figure}