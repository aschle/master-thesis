\chapter{Related Work}
\label{ch:relatedwork}

Relevant for my work are studies also using a network science approach focusing on interaction networks\footnote{Studies using worker-task, worker-nestarea, nestarea-nestarea or other bipartite networks are excluded.} to investigate the behavior of honey bees.
Therefore, I conducted a literature review in the field of network analysis of social insect\footnote{Animals belonging to social insects are: ants, dees, wasps and termites.} colonies.
Besides other sources, I mainly considered the studies mentioned in the survey papers of~\textcite{Pinter-Wollman2014}, ~\textcite[chapter~15]{krause2014animal} and~\textcite{charbonneau2013social}

I classified the studies by (A) type of analysis: temporal or static analysis (using automated or manual tracking over a long or short term); and (B) studied species: honey bees or other social insects.
Additionally, I reviewed their shortcomings regarding time, space, and individuals and inspected, thus, the following characteristics: duration of the study, observation period, sampling resolution, the number of colonies and marked individuals, space limitations and whether they integrated cohorts related to age. The studies and their characteristics are summarized in table~\ref{tab:studies} (appendix~\ref{ch:appendix}).

A lot of work already exists in the field of static analysis of social insect networks~\cite{greenwald2015ant,pinter2011effect,otterstatter2007contact,quevillon2015social,naug2009structure,formica2012fitness,waters2012information}, but only a few studies related to honey bees~\cite{baracchi2014socio,naug2008structure,scholl2011olfactory}.
Studies focusing on temporal aspects only exists for ants~\cite{mersch2013tracking,blonder2011time,jeanson2012long}, but, to the best of my knowledge, not for honey bees.
There is one work by~\cite{kimura2011new}, introducing an automatic tracking system for honeybees, but the system is, due to memory and storage limitations, not usable for long term observations, neighter did they approach network science methods for data analysis.

Static/bees\\
Focus~\textcite{baracchi2014socio}, spatial proximity, explain limitations\\
Only mention shortly:\\
\textcite{naug2008structure} trophallaxis, transfer duration and amount, too specialized\\
\textcite{scholl2011olfactory} only interaction between age groups, no network measures\\

Temporal/insects\\
Focus: \textcite{mersch2013tracking}, time-aggregated, communities, but no other network aspects, age\\
\textcite{jeanson2012long}, weighted time-aggregated networks, spatial proximity, TODO find limitations\\
Mention shortly:\\
\textcite{blonder2011time}, time-aggregated and time-ordered,\\




Goal: Temporal social network analysis of honey bees.