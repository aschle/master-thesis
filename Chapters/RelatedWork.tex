\chapter{Related Work}
\label{ch:relatedwork}

Relevant for my work are studies also using a network science approach focusing on interaction networks\footnote{Studies using worker-task, worker-nest area, nest area-nest area or other bipartite networks are excluded.} to investigate the behavior of honey bees.\\
Therefore, I conducted a literature review in the field of network analysis of social insect\footnote{Animals belonging to social insects are: ants, bees, wasps, and termites.} colonies.
Besides other sources, I mainly considered the studies mentioned in the survey papers of~\textcite{Pinter-Wollman2014}, ~\textcite[chapter~15]{krause2014animal} and~\textcite{charbonneau2013social}

I classified the studies by (A) type of analysis: temporal or static analysis (using automated or manual tracking over a long or short term); and (B) studied species: honey bees or other social insects.
Additionally, I reviewed their shortcomings regarding time, space, and the number of individuals and, thus, inspected the following characteristics: duration of the study, observation period, sampling resolution, the number of colonies and marked individuals, space limitations and whether they integrated cohorts related to age. The studies and their characteristics, also what software they used for network analysis are summarized in table~\ref{tab:studies} (appendix~\ref{ch:appendix}).

A lot of work already exists in the field of static analysis of social insect networks~\cite{greenwald2015ant,pinter2011effect,otterstatter2007contact,quevillon2015social,naug2009structure,formica2012fitness,waters2012information,sendova2010emergency}, and a few studies related to honey bees~\cite{baracchi2014socio,naug2008structure,scholl2011olfactory,naug2007experimentally}.
Studies focusing on temporal aspects only exists for ants~\cite{mersch2013tracking,blonder2011time,jeanson2012long}, but, to the best of my knowledge, not for honey bees.

The work by~\textcite{kimura2011new}, introducing an automatic tracking system for honeybees, but the system is, due to memory and storage limitations, not usable for long-term observations, neither did they use network science methods for data analysis.

\section{Static Network Analysis of honey bee colonies}
The most advanced work studying honey bees using a network science approach is by~\textcite{baracchi2014socio}. Using colored numbered discs for individually marking bees, they revealed a highly compartmentalized structure inside the honey bee colony. Depending on the age, bees occupy different areas of the comb and correspond to different tasks. Also, there is limited contact within age groups.\\
They used the frequency of interactions between bees as weights for edges in an undirected worker-worker interaction network, using the minor body length of a bee to define the radius of spatial proximity.
They use the node level measures strength (weighted degree), closeness and eigenvector centrality to investigate the networks.\\
A cluster analysis using as dissimilarity measures 'average linkage between groups' and 'squared Euclidian distance among network values' (net displacement, ward's linkage method with squared Euclidean distance as a measure of similarity).

The main drawback is that they marked only 211 from three predefined age cohorts out of one colony with 4000 individuals and observed only one side of the observation hive for 10 hours, capturing at a low resolution of one frame per minute. [TODO: explain drawback of clustering in a better way]

\textcite{scholl2011olfactory} investigate the mechanism for the emergence of organizational immunity by using unweighted, undirected physical contact and trophallaxis network. The observation is limited to one hour of video data for three days and spread over three weeks. Besides looking at the interaction between three predefined age groups no other methods regarding networks are used.

\textcite{naug2008structure} looks at the network structure of weighted,  directed trophallaxis networks using four age cohorts and the changes in transmission dynamics produced by experimental manipulation. The data is limited to one hour and only first and second order interactions are considered.

%%%%%%%%%%%%%%%%%%%%%%%%%%%%%%%%%%%%%%%%%%%%%%%%%%%%%%%%%%%%%%%%%%%%%%%%%%%%%%%
%%%%%%%%%%%%%%%%%%%%%%%%%%%%%%%%%%%%%%%%%%%%%%%%%%%%%%%%%%%%%%%%%%%%%%%%%%%%%%%
\section{Temporal Network Analysis of insect colonies}
%%%%%%%%%%%%%%%%%%%%%%%%%%%%%%%%%%%%%%%%%%%%%%%%%%%%%%%%%%%%%%%%%%%%%%%%%%%%%%%
%%%%%%%%%%%%%%%%%%%%%%%%%%%%%%%%%%%%%%%%%%%%%%%%%%%%%%%%%%%%%%%%%%%%%%%%%%%%%%%

Regarding the used methods, the study of~\textcite{mersch2013tracking} is very close to my work.
They automatically tracked all individuals of six ant colonies over a period of 41 days using a resolution of two frames per second.
For each observation day, the authors extracted time-aggregated weighted contact networks per colony, using antennation as the physical contact event.
They applied the Infomap community detection algorithm to each daily network and thus revealed three distinct and robust groups.
Each group represents a functional behavioral unit, with ants changing groups as they age.
The six ant colonies, they studied, contained 122 to 192 individuals, which is relatively small compared to the size of honey bee colonies used in the static analysis approaches.
Except for community detection, they did not use any other network science methods to investigate the network properties.

%%%%%%%%%%%%%%%%%%%%%%%%%%%%%%%%%%%%%%%%%%%%%%%%%%%%%%%%%%%%%%%%%%%%%%%%%%%%%%%

Another work, using automatic tracking, is by\textcite{jeanson2012long}.
It focuses on the investigation of the temporal stability of spatial proximity networks in four ant colonies.
Here, proximity is defined as $\frac{4}{3}$ of an ant’s body length.
For each week of three weeks of observation, they generate weighted time-aggregated networks per colony,  using the total duration of association as the edge weights.
They investigated the strength, betweenness and closeness centrality and found out that the networks are stable over time, without the queen contributing to the network structure.
Individuals with long lasting associations seem to have a reduced tendency to move, while mobile ants interact homogeneously with their nestmates.
Nevertheless, the size of the observed colonies ranges from 55 to 58 individuals, which is again, compared to bee colonies, rather small.

%%%%%%%%%%%%%%%%%%%%%%%%%%%%%%%%%%%%%%%%%%%%%%%%%%%%%%%%%%%%%%%%%%%%%%%%%%%%%%%

The only study not only using time-aggregated but, time-ordered (dynamic) networks is by\textcite{blonder2011time}.
They color painted all individuals of four ant colonies and filmed each colony for 30 minutes on two days, being three weeks apart.
The interaction events, physical contact of an ant's antenna with an ant's body, were manually extracted by watching the videos. Edges are therefore time-stamped interactions between individuals.
They show how temporal and spatial dynamics of individual interactions provide upper bounds to rates of colony-level information flow and how this flow scales with individual mobility and group size.
This very specialized study on dynamics in information flow also observed colonies with 6 to only 90 individuals.