%%%%%%%%%%%%%%%%%%%%%%%%%%%%%%%%%%%%%%%%%%%%%%%%%%%%%%%%%%%%%%%%%%%%%%%%%%%%%%%
%%%%%%%%%%%%%%%%%%%%%%%%%%%%%%%%%%%%%%%%%%%%%%%%%%%%%%%%%%%%%%%%%%%%%%%%%%%%%%%
%%%%%%%%%%%%%%%%%%%%%%%%%%%%%%%%%%%%%%%%%%%%%%%%%%%%%%%%%%%%%%%%%%%%%%%%%%%%%%%
%%%%%%%%%%%%%%%%%%%%%%%%%%%%%%%%%%%%%%%%%%%%%%%%%%%%%%%%%%%%%%%%%%%%%%%%%%%%%%%
\chapter{Approach and Implementation}
\label{ch:approach}
%%%%%%%%%%%%%%%%%%%%%%%%%%%%%%%%%%%%%%%%%%%%%%%%%%%%%%%%%%%%%%%%%%%%%%%%%%%%%%%
%%%%%%%%%%%%%%%%%%%%%%%%%%%%%%%%%%%%%%%%%%%%%%%%%%%%%%%%%%%%%%%%%%%%%%%%%%%%%%%
%%%%%%%%%%%%%%%%%%%%%%%%%%%%%%%%%%%%%%%%%%%%%%%%%%%%%%%%%%%%%%%%%%%%%%%%%%%%%%%
%%%%%%%%%%%%%%%%%%%%%%%%%%%%%%%%%%%%%%%%%%%%%%%%%%%%%%%%%%%%%%%%%%%%%%%%%%%%%%%

This chapter describes the workflow and implementation I applied to reach my research goals.
The first section describes the given data set and the approach to infer networks.
Generally speaking, this first step of network interference, was primarily driven by a combination of an exploratory data analysis and an iterative pipeline development processes.
It serves as a prerequisite for the further thesis.
The second section explains the methods I used to analyze the resulting networks regarding network properties, communities, and its development.

%%%%%%%%%%%%%%%%%%%%%%%%%%%%%%%%%%%%%%%%%%%%%%%%%%%%%%%%%%%%%%%%%%%%%%%%%%%%%%%
%%%%%%%%%%%%%%%%%%%%%%%%%%%%%%%%%%%%%%%%%%%%%%%%%%%%%%%%%%%%%%%%%%%%%%%%%%%%%%%
% INFERRING NETWORKS APPROACH
\section{Inferring Networks}
\label{sec:infNetworks}
%%%%%%%%%%%%%%%%%%%%%%%%%%%%%%%%%%%%%%%%%%%%%%%%%%%%%%%%%%%%%%%%%%%%%%%%%%%%%%%
%%%%%%%%%%%%%%%%%%%%%%%%%%%%%%%%%%%%%%%%%%%%%%%%%%%%%%%%%%%%%%%%%%%%%%%%%%%%%%%

To yield the first set of functional and non-functional requirements concerning the pipeline, I conducted (1) a data analysis of the given tracking data, and (2) a literature review,  already mentioned in chapter~\ref{ch:relatedwork}.
(1) supported the forming of a general understanding of the given dataset, its structure, characteristics and estimation of its quality.
The purpose of (2) was to get an overview of the possible types of networks and common methods and approaches in this field of research.
The results of (1) and (2) are then used to decide for a type of network and its node and edge definitions.
Furthermore, pipeline parameters are derived, and I decided for steps in the procedure of network extraction.

This pipeline is developed, tested and then refined in an iterative process.
The first prototype was implemented in regards to the first collected requirements and then iteratively tested and evaluated.
Test evaluation results consequently lead to new or changing functional requirements.
The evaluation is conducted by investigating the pipeline parameters' effects on network properties and checking the validity and quality of the networks by investigating the age of bees in the resulting network.





%%%%%%%%%%%%%%%%%%%%%%%%%%%%%%%%%%%%%%%%%%%%%%%%%%%%%%%%%%%%%%%%%%%%%%%%%%%%%%%
%%%%%%%%%%%%%%%%%%%%%%%%%%%%%%%%%%%%%%%%%%%%%%%%%%%%%%%%%%%%%%%%%%%%%%%%%%%%%%%
\subsubsection{Data Scheme}
\label{subsec:datascheme}
%%%%%%%%%%%%%%%%%%%%%%%%%%%%%%%%%%%%%%%%%%%%%%%%%%%%%%%%%%%%%%%%%%%%%%%%%%%%%%%
%%%%%%%%%%%%%%%%%%%%%%%%%%%%%%%%%%%%%%%%%%%%%%%%%%%%%%%%%%%%%%%%%%%%%%%%%%%%%%%

 



%%%%%%%%%%%%%%%%%%%%%%%%%%%%%%%%%%%%%%%%%%%%%%%%%%%%%%%%%%%%%%%%%%%%%%%%%%%%%%%
%%%%%%%%%%%%%%%%%%%%%%%%%%%%%%%%%%%%%%%%%%%%%%%%%%%%%%%%%%%%%%%%%%%%%%%%%%%%%%%
\subsection{Defining the Network Pipeline}
%%%%%%%%%%%%%%%%%%%%%%%%%%%%%%%%%%%%%%%%%%%%%%%%%%%%%%%%%%%%%%%%%%%%%%%%%%%%%%%
%%%%%%%%%%%%%%%%%%%%%%%%%%%%%%%%%%%%%%%%%%%%%%%%%%%%%%%%%%%%%%%%%%%%%%%%%%%%%%%

This section describes the pipeline for generating spatial proximity networks out of honey bee tracking data. The network pipeline takes as input a path to the data  and a set of parameter described before and outputs a graph in graphML file format. The pipeline is parallelized on frame level, that means, each process gets a portion (frames for a timeinterval of five minutes) of the data and extracts interactions/edges. The main process accumulates everything and creates a network.\\

The pipeline consist of the following steps:

\begin{enumerate}
\item \textbf{Prefilter detections}\\
All detections below the chosen level of confidence level are filtered out.

\item \textbf{Simple stitching}\\
Each side of the hive consists of two cameras. 	The $x$-coordinates of each detection (of the right	cameras) is moved further to the right, also adding an offset of $2\times \texttt{maximum distance}$. So the left and the right detection of each side of the hive are move into one reference system.

\item \textbf{Syncronize Cameras}\\
For each side of the hive the cameras need to be syncronized. In the normal case the difference between consecutive frames should be about $0.332$~seconds, due to technical problem this value can be lower ($0.003$ ) and higher ($2.932$) at certain times. Cameras 3 and 2 and cameras 1 and 0 are matched, frames without a match are dropped (shorter number of frames, matchen, threshold $0.33/2$, minimum).

\item \textbf{Discard Detections with certain IDs}\\
All detections whos ID is in a list are keept, other detections are discarded.

\item \textbf{Extract close pairs}\\
For each side of the hive, all close pairs according to the maximum distance parameter are calculated and then joined together using a KDE-tree.

\item \textbf{Combine data of to sides of the hive}\\
Per frame the data gets combined.

\item \textbf{Generate time series of bee pairs}\\
The data structure (frames and detection) is transformed to time series of bee pairs.

\item \textbf{Correct pair time series.}\\
The time series of bees are corrected by filling in the gaps of length \texttt{gap size}.

\item \textbf{Extract interactions}\\
The edges and its attributes (frequency and duration) are extracted from the time series of bees using the minimum contact duration parameter. A sequence of at least X ones counts as one interaction. The frequency of those series adn the total duration (number of ones) are the attributes.

\end{enumerate}

%%%%%%%%%%%%%%%%%%%%%%%%%%%%%%%%%%%%%%%%%%%%%%%%%%%%%%%%%%%%%%%%%%%%%%%%%%%%%%%
%%%%%%%%%%%%%%%%%%%%%%%%%%%%%%%%%%%%%%%%%%%%%%%%%%%%%%%%%%%%%%%%%%%%%%%%%%%%%%%
\subsection{Specifying the Network and its Parameters}
%%%%%%%%%%%%%%%%%%%%%%%%%%%%%%%%%%%%%%%%%%%%%%%%%%%%%%%%%%%%%%%%%%%%%%%%%%%%%%%
%%%%%%%%%%%%%%%%%%%%%%%%%%%%%%%%%%%%%%%%%%%%%%%%%%%%%%%%%%%%%%%%%%%%%%%%%%%%%%%
As this work constitues the first step towards network analysis using this tracking data I chose to infer time-aggregated spatial proximity network. The Accordingly, the interactions are undirected but weighted.
A node in the network is a bee, identified by an ID.
The network consists only of bees that interact with other bees at least once, during the specified time interval.\\
Two bees are associated (spatially close to each other), if their distance is smaller than a \emph{maximum distance}.
Using only this criterion leads to many interactions, resulting in a very dense network because an interaction could only last for 0.33 seconds.
Therefore, an additional parameter the \emph{minimum contact duration} is introduced.
It specifies the minimum time two bees have to spend close to each other to be called associated.

Edges are assigned two attributes.
The first one is the frequency of contacts, meaning how often they share a close position. The second parameter refers to the total duration of contact, so the total time they spend nearby.

%%%%%%%%%%%%%%%%%%%%%%%%%%%%%%%%%%%%%%%%%%%%%%%%%%%%%%%%%%%%%%%%%%%%%%%%%%%%%%%
\subsubsection{Pipeline Parameters}
%%%%%%%%%%%%%%%%%%%%%%%%%%%%%%%%%%%%%%%%%%%%%%%%%%%%%%%%%%%%%%%%%%%%%%%%%%%%%%%
The network pipeline takes two types of parameters: one for specifying the resulting network and how spacial proximity is defined and one relates to the data set.

\begin{description}
\item[Maximum distance] level of closeness between to individual bees~(in pixel)
\item[Minimum contact duration] the number of frames two individuals need to spend close by in order to count it as an interaction~(in frames)
\item [Start timestamp] starting point of the network aggregation~(as UTC string)
\item [Window size] size of time window for aggregating the network~(in minutes)

\vspace{5mm}

\item[Confidence] level of confidence, as described in section~\nameref{subsec:confidence}~(in percent)
\item[Valid IDs] list of valid ids within a specified time interval, as described in section~\nameref{subsubsec:dataset:filter}~(in csv file format)
\item[Gap Size] this is used to corect the time series of bee pairs~(in frames)
\item[Number of CPUs] number of used CPUs for parallelization
\item[Year] calculate bee IDs and stitching of camera images according to the observation period~(2015 or 2016)
\end{description}

%%%%%%%%%%%%%%%%%%%%%%%%%%%%%%%%%%%%%%%%%%%%%%%%%%%%%%%%%%%%%%%%%%%%%%%%%%%%%%%
\subsubsection{Chosen Parameter Values for Network Analysis}
%%%%%%%%%%%%%%%%%%%%%%%%%%%%%%%%%%%%%%%%%%%%%%%%%%%%%%%%%%%%%%%%%%%%%%%%%%%%%%%
\begin{table}[tbp]
\small
\centering
\caption[Parameters chosen for network analysis]{\textbf{Parameters chosen for network analysis} The maximum distance corresponds to the length of a bee body and the minimum contact duration is about one second. The networks are aggregated for ten hours.\\
}
\label{tab:chosenparams}

\begin{tabular}{rrl}
	\toprule
	\textbf{Parameter} & \textbf{Value} & \textbf{Unit} \\ \midrule
	Maximum distance & 212 & px \\
	Minimum contact duration & 3 & frames \\
	Window size & 600 & minutes \\ \midrule
	Confidence & 95 & percent \\
	Gap size & 2 & frames \\
	\bottomrule
\end{tabular}

\end{table}

For further network analysis, I chose three days: 20., 22., and 24. August.
Those days were chosen because a wide range of age groups was present at this time. The hive especially contained older bees which are likely to be foragers. Besides, no data is missing on those days.

The values are chosen according to biological constraints and similar to other studies, for better comparability.
I chose the length of a bee body, according to \textcite{baracchi2014socio}, as the maximum distance between two bees (figure~\ref{fig:contactRadius}). The average bee length of $212$px ($\pm 16$px)  was determinded by manually measuring the length of all bees ($n=337$) of four camera images using the tool ImageJ\footnote{\url{http://imagej.net/Welcome}; Last accessed:
 22.02.2016}.
The minimum contact duration is set to three frames (one second). This corresponds to~\textcite{mersch2013tracking}, they as also exclude interactions below one second.
To keep about 50\% of the data the confidence is set to $95\%$.
The gap size is set to two frames. This value corresponds to the median gap length in the time series of pairs.

\begin{figure}[htb]
	\centering
	\begin{subfigure}[b]{0.45\textwidth}
		\includegraphics[width=\textwidth]{Figures/sizeTagBee}
		\caption[Body length of a bee]{Body length of a bee}
		\label{fig:size}
	\end{subfigure}
	\hspace{0.08\textwidth}
	\begin{subfigure}[b]{0.45\textwidth}
		\centering
		\includegraphics[width=\textwidth]{Figures/radius}
		\caption[Contact radius]{Contact radius}
		\label{fig:radius}
	\end{subfigure}
	\caption{Distance Between Bees: A length of a bee is chosen as the maximal  distance between bees.}
	\label{fig:contactRadius}
\end{figure}
\subsection{Summary and Results}
[TODO]\\

Networks in general, yes but\\
complex preprocessing of the data essential\\
confidence: reduce amount of data\\
hight confidence: less data, a bit better quality, but more gaps\\
low confidence: a lot of data, bad quality but less gaps\\
stitching cameras: remove duplicates\\
syncronize cameras\\
pre filter detections by ids, whos detection frequency is low\\

Temporal networks, yes but\\
time--aggregated\\
daily networks, because biological useful, other window would be random\\
due to maintainance and technical, camera failures: no period without interruptions\\



% NETWORK ANALYSIS APPROACH
%%%%%%%%%%%%%%%%%%%%%%%%%%%%%%%%%%%%%%%%%%%%%%%%%%%%%%%%%%%%%%%%%%%%%%%%%%%%%%%
%%%%%%%%%%%%%%%%%%%%%%%%%%%%%%%%%%%%%%%%%%%%%%%%%%%%%%%%%%%%%%%%%%%%%%%%%%%%%%%
\section{Methods for Analyzing Spatial Proximity Networks}
%%%%%%%%%%%%%%%%%%%%%%%%%%%%%%%%%%%%%%%%%%%%%%%%%%%%%%%%%%%%%%%%%%%%%%%%%%%%%%%
%%%%%%%%%%%%%%%%%%%%%%%%%%%%%%%%%%%%%%%%%%%%%%%%%%%%%%%%%%%%%%%%%%%%%%%%%%%%%%%
[TODO überarbeiten]\\
This section explains the what measures I used to investigate the properties of my temporal networks and justifies my choice. Also I explain how I chose a community detection algorithm and which one I picked. Explains method to examine age and spatial segregation of communities and how I study the development of communities.

%%%%%%%%%%%%%%%%%%%%%%%%%%%%%%%%%%%%%%%%%%%%%%%%%%%%%%%%%%%%%%%%%%%%%%%%%%%%%%%
\subsection{Investigating the Topology and Network Characteristics}
\label{subsec:APmeasures}
%%%%%%%%%%%%%%%%%%%%%%%%%%%%%%%%%%%%%%%%%%%%%%%%%%%%%%%%%%%%%%%%%%%%%%%%%%%%%%%
[TODO: überarbeiten]\\
Table~\ref{tab:studies-measures} (or figure~\ref{fig:study-measures} summarized the used network analysis methods in the reviewed studies mentioned in chapter~\ref{ch:relatedwork}. The table includes global level measures, node level measures and other network analysis methods the authors used in their studies.
I chose the measures for my own analysis, because of XY.
[TODO: do I need to explain, why I used this and not that?]
Therefore, I am going to analyse the global network properties and local node level properties listed in table~\ref{tab:netprop}.
The node level metrics are investigated also in relation to the bees age.
The global network properties are compared to an Erdos-Renyi random network, by averaging over 100 runs [TODO cite?].\\

The degree $k$ of a bee represents the number of other bees this focal animal interacts.
Bees with a high number of interaction partners, therefore, have a high degree. Bees with a low number of interaction partners consequently a low degree.\\
The strength $s$ of a bee is the total number of all its interactions. A high strength indicates that this bee has either a high number of interaction partners (with a low interaction frequency, low edge weight) or interaction partners, with a high interaction frequency (high edge weight).\\
The local clustering coefficient (lcc) $c$ of a bee indicates how close its interaction partners are to being a clique\footnote{A clique is a complete subgraph.}. A high lcc indicates that its interaction partners all interact with each other. A low lcc shows the absence of those interactions.

The betweenness of a bee measure how many shortest paths go through a bee, meaning how many information would flow through a bee or how many foods is transferred. A bee with a high betweenness would be central or important for the network in the sense of information flow. Removing this bee would lead to the breakdown of information or food flow and would negatively affect the robustness of the network.

The closeness of a bee measures how fast it can reach all others in the network. A high closeness indicated a very short path to every other bee. A low closeness consequently a long path to all other bees. Regarding information flow, a bee with high closeness can spread information to all other bees very fast.

\begin{table}
\small
\centering
\caption[Measures used for analysis]{\textbf{Measures used for analysis} Each measure is explained in Chapter~\ref{sec:definitions}}
\vspace*{5mm}
\begin{tabularx}{\textwidth}{p{0.5\linewidth}p{0.5\linewidth}}
\toprule
\textbf{Global level measures} & \textbf{Node level measures}\\
\midrule
Number of nodes $N$ and links $E$ & Degree $k$ \\
Average degree $\langle k \rangle$ &  Strength $s$\\
Average strength $\langle s \rangle$ &   Local clustering coefficient $c$\\
Density $D$ & Closeness centrality $C_C$ \\
Diameter $d_{max}$ & Betweenness centrality $C_B$\\
Number of components & \\
Global clustering coefficient $c_{\Delta}$ &  \\
Average shortest path length $\langle d \rangle$ & \\
Link weights $w$ & \\

\bottomrule
\end{tabularx}
\label{tab:netprop}
\end{table}


%%%%%%%%%%%%%%%%%%%%%%%%%%%%%%%%%%%%%%%%%%%%%%%%%%%%%%%%%%%%%%%%%%%%%%%%%%%%%%%
\subsection{Detecting Communities}
\label{subsec:APcommunityDet}
%%%%%%%%%%%%%%%%%%%%%%%%%%%%%%%%%%%%%%%%%%%%%%%%%%%%%%%%%%%%%%%%%%%%%%%%%%%%%%%
[TODO: überarbeiten]\\
(1) check reviwed studies, (2) check comparative analysis, (3) check algos by myselfe.
The reviewed studies only include two examples of community and cluster analysis.
\textcite{mersch2013tracking} used the infomap~\cite{rosvall2009map,rosvall2007information} algorithm. As they explain this algorithm only works for sparse networks, it is not applicable in my case. \textcite{baracchi2014socio} use a clustering algorithm. [TODO explain and why not want to use] I want to perform community detection insted of cluster analysis. [TODO: difference?]
There are comparative analysis of community detection algorithms, e.g.~\cite{yang2016comparative, harenberg2014community}. They seem to be promising, but assume eighter a power law degree distribution or evaluate networks with a low density, which is not applicable here.

Thererfor, I tested all community detection algorithms implemented in python, to find an algorithm, which works well for my case of animal social networks. The three most common python libraries for network analysis were reviewed: NetworkX\footnote{\url{https://networkx.github.io/}; Last accessed: 16.03.2016, 6:36~p.m.}, igraph\footnote{\url{http://igraph.org/python/}; Last accessed: 16.03.2016, 6:38~p.m.}, and graph-tool\footnote{\url{https://graph-tool.skewed.de/}; Last accessed: 16:03.2016, 6:39~p.m.})

The algorithm needs to fulfill the following criteria:

\begin{itemize}
\item Support for large and very dense networks ($N>1000$, $D>50~\%$)
\item Support weighted edges
\item Fast runtime
\end{itemize}

Table~\ref{tab:algos} gives an overview about the twelve algorithms reviewed. Five algorithms did not terminate after 15~minutes and were therefore excluded from further investigations. Infomap and label propagation tend to partition all nodes into a single community, this is known especially in dense graphs~\cite{yang2016comparative, fortunato2010community}.
The Louvain algorithm is the same as multilevel, but takes longer producing almost the same communities and therefore was also excluded. Walktrap was tested for different step size parameters, as suggested in~\cite{pons2005computing}, the communities remained almost the same (only a few nodes switched communities). 

I had a closer look at fastgreedy, leading eigenvector, multilevel, and walktrap regarding the number of detected communities and community size for all three networks. Table~\ref{tab:algos4} shows the results. All algorithms found at least two communities. Except for leading eigenvector, there is a tendency that a third community exists.
I decided to use two algorithms for community detection: leading eigenvector and walktrap. \textcite{farine2015constructing} explains that leading eigenvector is often used with animal social networks and works well. Walktrap is chosen for also  examining the possible third community.

\begin{table}[htbp]
\small
\caption[Compairing community detection algorithms]{\textbf{Comparing community detection algorithms} Comparison of algorithms implemented in python. Criteria are the support of weighted links, runtime and number of communities. A runtime indicated by ``$-$'' means no termination after 15~minutes.\\
}
\label{tab:algos}

\begin{tabularx}{\textwidth}{lcccccccccccc}
\toprule
	 {} &
	 \rotatebox{90}{\textbf{Fastgreedy$^1$}} &
	 \rotatebox{90}{\textbf{Leading eigenvector$^1$}} &
	 \rotatebox{90}{Louvain$^2$} &
	 \rotatebox{90}{\textbf{Multilevel$^1$}} &
	 \rotatebox{90}{\textbf{Walktrap$^1$}} &
	 
	 \rotatebox{90}{Infomap$^1$} &
	 \rotatebox{90}{Label propagation$^1$} &
	 
	 \rotatebox{90}{Edge betweenness$^1$} &
	 \rotatebox{90}{K-clique communities$^2$\thinspace} &
	 \rotatebox{90}{Optimal modularity$^1$} &
	 \rotatebox{90}{Spinglass$^1$} &
	 \rotatebox{90}{Statistical inference$^3$} \\ \midrule
	 
	 
	 
	 Link weights & $\times$ & $\times$ & $\times$ & $\times$ & $\times$ & $\times$ & $\times$ & & $\times$ & $\times$ & $\times$ \\ \midrule
	 Runtime in sec & ~$3.6$ & ~$6.3$ & $11.7$ & ~$0.7$ & $19.4$ & $13.2$ & ~$0.2$ & $-$ & $-$ & $-$ & $-$ & $-$ \\ \midrule
	 Communities & $3$ & $2$ & $2$ & $3$ & $2$ & $1$ & $1$ & $-$ & $-$ & $-$ & $-$ & $-$ \\ \midrule
	 Size & 473 & 488 & 469 & 462 & 490 & 922 &  922 &  &  &  &  &  \\
	  & 434 & 434 & 453 & 427 & 431 &  &  &  &  &  &  &  \\
	  & 15 &  &  & 33 & (1) &  &  &  &  &  &  &  \\
	 \bottomrule
	 
\end{tabularx}
\begin{flushright}
\footnotesize{
$^1$ igraph, $^2$ NetworkX, $^3$ graph-tool\\
}
\end{flushright}

\end{table}

% \hdashline
% \midrule
% \bottomrule
\begin{table}[htbp]
\small
\centering
\caption[Number of community members per algorithm and snapshot]{\textbf{Number of community members per algorithm and snapshot} Four algorithms were tested and compared regarding the number of detected communities and the size of the communities.\\
}
\label{tab:algos4}

\begin{tabular}{lcccc}
\toprule
	 {} &
	 \rotatebox{90}{Fastgreedy} &
	 \rotatebox{90}{\textbf{Leading eigenvector}} &
	 \rotatebox{90}{Multilevel} &
	 \rotatebox{90}{\textbf{Walktrap}} \\ \midrule
	 
	  Snapshot 1
	  & 473 & 488 & 462 & 490 \\
	  & 434 & 434 & 427 & 431 \\
	  & 15 &   & 33 & (1) \\ \midrule
	  Snapshot 2
	  & 504 & 503 & 481 & 372 \\
	  & 467 & 475 & 439 & 311 \\
	  & 7 &   &  58 & 294 \\
	  & & & & (1) \\ \midrule
	  Snapshot 3
	  & 534 & 537 & 505 & 310 \\
	  & 388 & 385 & 415 & 390 \\
	  &  &   &  (2) & 231 \\
	 \bottomrule
\end{tabular}
\end{table}

% \hdashline
% \midrule
% \bottomrule

\paragraph{Age Distribution of Communities}
[TODO überarbeiten]\\
For each community I investigated the age distribution and the average age for. I also investigated whether the age division persists in each snapshot. A two sample Kolmogorov-Smirnov test was used to determine the statistically difference of the age distribution between communities.
Answer the question: Communities reflect different age groups?
For hypothesis (2) the data is stored as a csv file of birth dates of each bee. For testing if age goups are different the Kolmogorov Smirnov Test was used.\\

\paragraph{Spatial Distribution of Communities}
[TODO überarbeiten]\\
Communities occupy different areas of the comb (similar to~\cite{baracchi2014socio}). Do they stay at the same in each snapshot?
Answer the question: Communities reflect groups of bees working in different areas of the hive? The data which was used to test the hypothesis (1) is saved in a sqlite database for faster access, because using bb\_binary (parsing the data over and over again) was to slow.\\

%%%%%%%%%%%%%%%%%%%%%%%%%%%%%%%%%%%%%%%%%%%%%%%%%%%%%%%%%%%%%%%%%%%%%%%%%%%%%%%
\subsection{Evolving Communities}
\label{sec:bg:tracking}
%%%%%%%%%%%%%%%%%%%%%%%%%%%%%%%%%%%%%%%%%%%%%%%%%%%%%%%%%%%%%%%%%%%%%%%%%%%%%%%
[TODO: Change to intersection and flowcharts]
According to \textcite{aynaud2013communities} and  \textcite{brodka2014community} there are three main approaches for community detection in temporal networks (sometimes referred to as community tracking): (1) using a static community detection algorithm on several snapshots and then solving a matching problem, (2) using algorithms that are directly suited for temporal networks and (3) using incremental or online algorithms when processing data streams. For each of the three approaches, several methods already exist.
As community tracking is not the main focus of this work, I chose to apply the most natural method out of approach (1): detecting static communities for each snapshot and then matching those communities using set theory.


Two communities at successive time steps are matched if they share enough nodes.
The \emph{match value} between two communities $C$ and $D$ according to \textcite{hopcroft2004tracking} is defined as:

\begin{equation}
\label{eq:match}
\texttt{match}(C,D) = \texttt{min}\left( \frac{\textbar C\cap D \textbar}{\textbar C\textbar }, \frac{\textbar C\cap D \textbar}{\textbar D \textbar }\right)
\end{equation}


This value is between 0 and 1. A high match value occurs when two communities share many nodes and are of a similar size. Communities with the highest value are matched. The author suggests applying a threshold to more precisely define what ``share a lot of nodes'' means. Otherwise, a matching could occur between communities with only 0.1\% of overlapping nodes. I matched all communities, but excluded values below 3\%.


I calculated the match value between consecutive snapshots, to investigate the number of bees, which stay the same over time. Also, I calculated all match values of all communities per snapshot.

\subsection{Summary}
[maybe add some short summary]