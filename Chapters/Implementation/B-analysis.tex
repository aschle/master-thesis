%%%%%%%%%%%%%%%%%%%%%%%%%%%%%%%%%%%%%%%%%%%%%%%%%%%%%%%%%%%%%%%%%%%%%%%%%%%%%%%
%%%%%%%%%%%%%%%%%%%%%%%%%%%%%%%%%%%%%%%%%%%%%%%%%%%%%%%%%%%%%%%%%%%%%%%%%%%%%%%
\section{Methods for Analyzing Spatial Proximity Networks}
%%%%%%%%%%%%%%%%%%%%%%%%%%%%%%%%%%%%%%%%%%%%%%%%%%%%%%%%%%%%%%%%%%%%%%%%%%%%%%%
%%%%%%%%%%%%%%%%%%%%%%%%%%%%%%%%%%%%%%%%%%%%%%%%%%%%%%%%%%%%%%%%%%%%%%%%%%%%%%%
[TODO überarbeiten]\\
This section explains the what measures I used to investigate the properties of my temporal networks and justifies my choice. Also I explain how I chose a community detection algorithm and which one I picked. Explains method to examine age and spatial segregation of communities and how I study the development of communities.

%%%%%%%%%%%%%%%%%%%%%%%%%%%%%%%%%%%%%%%%%%%%%%%%%%%%%%%%%%%%%%%%%%%%%%%%%%%%%%%
\subsection{Investigating the Topology and Network Characteristics}
\label{subsec:APmeasures}
%%%%%%%%%%%%%%%%%%%%%%%%%%%%%%%%%%%%%%%%%%%%%%%%%%%%%%%%%%%%%%%%%%%%%%%%%%%%%%%
[TODO: überarbeiten]\\
Table~\ref{tab:studies-measures} (or figure~\ref{fig:study-measures} summarized the used network analysis methods in the reviewed studies mentioned in chapter~\ref{ch:relatedwork}. The table includes global level measures, node level measures and other network analysis methods the authors used in their studies.
I chose the measures for my own analysis, because of XY.
[TODO: do I need to explain, why I used this and not that?]
Therefore, I am going to analyse the global network properties and local node level properties listed in table~\ref{tab:netprop}.
The node level metrics are investigated also in relation to the bees age.
The global network properties are compared to an Erdos-Renyi random network, by averaging over 100 runs [TODO cite?].\\

The degree $k$ of a bee represents the number of other bees this focal animal interacts.
Bees with a high number of interaction partners, therefore, have a high degree. Bees with a low number of interaction partners consequently a low degree.\\
The strength $s$ of a bee is the total number of all its interactions. A high strength indicates that this bee has either a high number of interaction partners (with a low interaction frequency, low edge weight) or interaction partners, with a high interaction frequency (high edge weight).\\
The local clustering coefficient (lcc) $c$ of a bee indicates how close its interaction partners are to being a clique\footnote{A clique is a complete subgraph.}. A high lcc indicates that its interaction partners all interact with each other. A low lcc shows the absence of those interactions.

The betweenness of a bee measure how many shortest paths go through a bee, meaning how many information would flow through a bee or how many foods is transferred. A bee with a high betweenness would be central or important for the network in the sense of information flow. Removing this bee would lead to the breakdown of information or food flow and would negatively affect the robustness of the network.

The closeness of a bee measures how fast it can reach all others in the network. A high closeness indicated a very short path to every other bee. A low closeness consequently a long path to all other bees. Regarding information flow, a bee with high closeness can spread information to all other bees very fast.

\begin{table}
\small
\centering
\caption[Measures used for analysis]{\textbf{Measures used for analysis} Each measure is explained in Chapter~\ref{sec:definitions}}
\vspace*{5mm}
\begin{tabularx}{\textwidth}{p{0.5\linewidth}p{0.5\linewidth}}
\toprule
\textbf{Global level measures} & \textbf{Node level measures}\\
\midrule
Number of nodes $N$ and links $E$ & Degree $k$ \\
Average degree $\langle k \rangle$ &  Strength $s$\\
Average strength $\langle s \rangle$ &   Local clustering coefficient $c$\\
Density $D$ & Closeness centrality $C_C$ \\
Diameter $d_{max}$ & Betweenness centrality $C_B$\\
Number of components & \\
Global clustering coefficient $c_{\Delta}$ &  \\
Average shortest path length $\langle d \rangle$ & \\
Link weights $w$ & \\

\bottomrule
\end{tabularx}
\label{tab:netprop}
\end{table}


%%%%%%%%%%%%%%%%%%%%%%%%%%%%%%%%%%%%%%%%%%%%%%%%%%%%%%%%%%%%%%%%%%%%%%%%%%%%%%%
\subsection{Detecting Communities}
\label{subsec:APcommunityDet}
%%%%%%%%%%%%%%%%%%%%%%%%%%%%%%%%%%%%%%%%%%%%%%%%%%%%%%%%%%%%%%%%%%%%%%%%%%%%%%%
[TODO: überarbeiten]\\
(1) check reviwed studies, (2) check comparative analysis, (3) check algos by myselfe.
The reviewed studies only include two examples of community and cluster analysis.
\textcite{mersch2013tracking} used the infomap~\cite{rosvall2009map,rosvall2007information} algorithm. As they explain this algorithm only works for sparse networks, it is not applicable in my case. \textcite{baracchi2014socio} use a clustering algorithm. [TODO explain and why not want to use] I want to perform community detection insted of cluster analysis. [TODO: difference?]
There are comparative analysis of community detection algorithms, e.g.~\cite{yang2016comparative, harenberg2014community}. They seem to be promising, but assume eighter a power law degree distribution or evaluate networks with a low density, which is not applicable here.

Thererfor, I tested all community detection algorithms implemented in python, to find an algorithm, which works well for my case of animal social networks. The three most common python libraries for network analysis were reviewed: NetworkX\footnote{\url{https://networkx.github.io/}; Last accessed: 16.03.2016, 6:36~p.m.}, igraph\footnote{\url{http://igraph.org/python/}; Last accessed: 16.03.2016, 6:38~p.m.}, and graph-tool\footnote{\url{https://graph-tool.skewed.de/}; Last accessed: 16:03.2016, 6:39~p.m.})

The algorithm needs to fulfill the following criteria:

\begin{itemize}
\item Support for large and very dense networks ($N>1000$, $D>50~\%$)
\item Support weighted edges
\item Fast runtime
\end{itemize}

Table~\ref{tab:algos} gives an overview about the twelve algorithms reviewed. Five algorithms did not terminate after 15~minutes and were therefore excluded from further investigations. Infomap and label propagation tend to partition all nodes into a single community, this is known especially in dense graphs~\cite{yang2016comparative, fortunato2010community}.
The Louvain algorithm is the same as multilevel, but takes longer producing almost the same communities and therefore was also excluded. Walktrap was tested for different step size parameters, as suggested in~\cite{pons2005computing}, the communities remained almost the same (only a few nodes switched communities). 

I had a closer look at fastgreedy, leading eigenvector, multilevel, and walktrap regarding the number of detected communities and community size for all three networks. Table~\ref{tab:algos4} shows the results. All algorithms found at least two communities. Except for leading eigenvector, there is a tendency that a third community exists.
I decided to use two algorithms for community detection: leading eigenvector and walktrap. \textcite{farine2015constructing} explains that leading eigenvector is often used with animal social networks and works well. Walktrap is chosen for also  examining the possible third community.

\begin{table}[htbp]
\small
\caption[Compairing community detection algorithms]{\textbf{Comparing community detection algorithms} Comparison of algorithms implemented in python. Criteria are the support of weighted links, runtime and number of communities. A runtime indicated by ``$-$'' means no termination after 15~minutes.\\
}
\label{tab:algos}

\begin{tabularx}{\textwidth}{lcccccccccccc}
\toprule
	 {} &
	 \rotatebox{90}{\textbf{Fastgreedy$^1$}} &
	 \rotatebox{90}{\textbf{Leading eigenvector$^1$}} &
	 \rotatebox{90}{Louvain$^2$} &
	 \rotatebox{90}{\textbf{Multilevel$^1$}} &
	 \rotatebox{90}{\textbf{Walktrap$^1$}} &
	 
	 \rotatebox{90}{Infomap$^1$} &
	 \rotatebox{90}{Label propagation$^1$} &
	 
	 \rotatebox{90}{Edge betweenness$^1$} &
	 \rotatebox{90}{K-clique communities$^2$\thinspace} &
	 \rotatebox{90}{Optimal modularity$^1$} &
	 \rotatebox{90}{Spinglass$^1$} &
	 \rotatebox{90}{Statistical inference$^3$} \\ \midrule
	 
	 
	 
	 Link weights & $\times$ & $\times$ & $\times$ & $\times$ & $\times$ & $\times$ & $\times$ & & $\times$ & $\times$ & $\times$ \\ \midrule
	 Runtime in sec & ~$3.6$ & ~$6.3$ & $11.7$ & ~$0.7$ & $19.4$ & $13.2$ & ~$0.2$ & $-$ & $-$ & $-$ & $-$ & $-$ \\ \midrule
	 Communities & $3$ & $2$ & $2$ & $3$ & $2$ & $1$ & $1$ & $-$ & $-$ & $-$ & $-$ & $-$ \\ \midrule
	 Size & 473 & 488 & 469 & 462 & 490 & 922 &  922 &  &  &  &  &  \\
	  & 434 & 434 & 453 & 427 & 431 &  &  &  &  &  &  &  \\
	  & 15 &  &  & 33 & (1) &  &  &  &  &  &  &  \\
	 \bottomrule
	 
\end{tabularx}
\begin{flushright}
\footnotesize{
$^1$ igraph, $^2$ NetworkX, $^3$ graph-tool\\
}
\end{flushright}

\end{table}

% \hdashline
% \midrule
% \bottomrule
\begin{table}[htbp]
\small
\centering
\caption[Number of community members per algorithm and snapshot]{\textbf{Number of community members per algorithm and snapshot} Four algorithms were tested and compared regarding the number of detected communities and the size of the communities.\\
}
\label{tab:algos4}

\begin{tabular}{lcccc}
\toprule
	 {} &
	 \rotatebox{90}{Fastgreedy} &
	 \rotatebox{90}{\textbf{Leading eigenvector}} &
	 \rotatebox{90}{Multilevel} &
	 \rotatebox{90}{\textbf{Walktrap}} \\ \midrule
	 
	  Snapshot 1
	  & 473 & 488 & 462 & 490 \\
	  & 434 & 434 & 427 & 431 \\
	  & 15 &   & 33 & (1) \\ \midrule
	  Snapshot 2
	  & 504 & 503 & 481 & 372 \\
	  & 467 & 475 & 439 & 311 \\
	  & 7 &   &  58 & 294 \\
	  & & & & (1) \\ \midrule
	  Snapshot 3
	  & 534 & 537 & 505 & 310 \\
	  & 388 & 385 & 415 & 390 \\
	  &  &   &  (2) & 231 \\
	 \bottomrule
\end{tabular}
\end{table}

% \hdashline
% \midrule
% \bottomrule

\paragraph{Age Distribution of Communities}
[TODO überarbeiten]\\
For each community I investigated the age distribution and the average age for. I also investigated whether the age division persists in each snapshot. A two sample Kolmogorov-Smirnov test was used to determine the statistically difference of the age distribution between communities.
Answer the question: Communities reflect different age groups?
For hypothesis (2) the data is stored as a csv file of birth dates of each bee. For testing if age goups are different the Kolmogorov Smirnov Test was used.\\

\paragraph{Spatial Distribution of Communities}
[TODO überarbeiten]\\
Communities occupy different areas of the comb (similar to~\cite{baracchi2014socio}). Do they stay at the same in each snapshot?
Answer the question: Communities reflect groups of bees working in different areas of the hive? The data which was used to test the hypothesis (1) is saved in a sqlite database for faster access, because using bb\_binary (parsing the data over and over again) was to slow.\\

%%%%%%%%%%%%%%%%%%%%%%%%%%%%%%%%%%%%%%%%%%%%%%%%%%%%%%%%%%%%%%%%%%%%%%%%%%%%%%%
\subsection{Evolving Communities}
\label{sec:bg:tracking}
%%%%%%%%%%%%%%%%%%%%%%%%%%%%%%%%%%%%%%%%%%%%%%%%%%%%%%%%%%%%%%%%%%%%%%%%%%%%%%%
[TODO: Change to intersection and flowcharts]
According to \textcite{aynaud2013communities} and  \textcite{brodka2014community} there are three main approaches for community detection in temporal networks (sometimes referred to as community tracking): (1) using a static community detection algorithm on several snapshots and then solving a matching problem, (2) using algorithms that are directly suited for temporal networks and (3) using incremental or online algorithms when processing data streams. For each of the three approaches, several methods already exist.
As community tracking is not the main focus of this work, I chose to apply the most natural method out of approach (1): detecting static communities for each snapshot and then matching those communities using set theory.


Two communities at successive time steps are matched if they share enough nodes.
The \emph{match value} between two communities $C$ and $D$ according to \textcite{hopcroft2004tracking} is defined as:

\begin{equation}
\label{eq:match}
\texttt{match}(C,D) = \texttt{min}\left( \frac{\textbar C\cap D \textbar}{\textbar C\textbar }, \frac{\textbar C\cap D \textbar}{\textbar D \textbar }\right)
\end{equation}


This value is between 0 and 1. A high match value occurs when two communities share many nodes and are of a similar size. Communities with the highest value are matched. The author suggests applying a threshold to more precisely define what ``share a lot of nodes'' means. Otherwise, a matching could occur between communities with only 0.1\% of overlapping nodes. I matched all communities, but excluded values below 3\%.


I calculated the match value between consecutive snapshots, to investigate the number of bees, which stay the same over time. Also, I calculated all match values of all communities per snapshot.

\subsection{Summary}
[maybe add some short summary]