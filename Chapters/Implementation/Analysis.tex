\section{Static and Temporal Analysis}

Despite the possibility of generating networks of different granularity (resolution is minutes), here for further analysis daily networks (10h, two hours after sunrise until zwo hours before sunset) are aggregated.


\textcite{wey2008social}

\subsection{Static Network Analysis}
The following network properties were analysed for a static day and hour network.\\
TODO: list of properties. (similar to what others have done)
nodes, edges, density, diameter\\


\subsection{Temporal Analysis}
three day networks (2 days gap)\\
one network 2 weeks later\\

\subsection{Community Detection}
I tested all community detection algorithms implemented in python, to find an algorithm, which works well for my case of animal social networks. The three most common python libraries for network analysis were reviewed: NetworkX\footnote{\url{https://networkx.github.io/}; Last accessed: 16.03.2016, 6:36~p.m.}, igraph\footnote{\url{http://igraph.org/python/}; Last accessed: 16.03.2016, 6:38~p.m.}, and graph-tool\footnote{\url{https://graph-tool.skewed.de/}; Last accessed: 16:03.2016, 6:39~p.m.})

The algorithm needs to fulfill the following criteria:

\begin{itemize}
\item Support for large and very dense networks ($N>1000$, $D>50~\%$)
\item Support weighted edges
\item Fast runtime
\end{itemize}

Table~\ref{tab:algos} gives an overview about the twelve algorithms reviewed. Five algorithms did not terminate after 15~minutes and were therefore excluded from further investigations. Infomap and label propagation tend to partition all nodes into a single community, this is known especially in dense graphs~\cite{yang2016comparative, fortunato2010community}.
The Louvain algorithm is the same as multilevel, but takes longer producing almost the same communities and therefore was also excluded. Walktrap was tested for different step size parameters, as suggested in~\cite{pons2005computing}, the communities remained almost the same (only a few nodes switched communities). 

I had a closer look at fastgreedy, leading eigenvector, multilevel, and walktrap regarding the number of detected communities and community size for all three networks. Table~\ref{tab:algos4} shows the results. All algorithms found at least two communities. Except for leading eigenvector, there is a tendency that a third community exists.
I decided to use two algorithms for community detection: leading eigenvector and walktrap. \textcite{farine2015constructing} explains that leading eigenvector is often used with animal social networks and works well. Walktrap is chosen for also  examining the possible third community.

There are comparative analysis of community detection algorithms, e.g.~\cite{yang2016comparative, harenberg2014community}. They seem to be promising, but assume eighter a power law degree distribution or evaluate networks with a low density, which is not applicable here.

\begin{table}[htbp]
\small
\caption[Compairing community detection algorithms]{\textbf{Comparing community detection algorithms} Comparison of algorithms implemented in python. Criterias are the support of weighted edges, runtime and number of communities. A runtime indicated by $-$ mean no termination after 15~minutes.\\
}
\label{tab:algos}

\begin{tabularx}{\textwidth}{lcccccccccccc}
\toprule
	 {} &
	 \rotatebox{90}{\textbf{fastgreedy$^1$}} &
	 \rotatebox{90}{\textbf{leading eigenvector$^1$}} &
	 \rotatebox{90}{louvain$^2$} &
	 \rotatebox{90}{\textbf{multilevel$^1$}} &
	 \rotatebox{90}{\textbf{walktrap$^1$}} &
	 
	 \rotatebox{90}{infomap$^1$} &
	 \rotatebox{90}{label propagation$^1$} &
	 
	 \rotatebox{90}{edge betweenness$^1$} &
	 \rotatebox{90}{k-clique communities$^2$\thinspace} &
	 \rotatebox{90}{optimal modularity$^1$} &
	 \rotatebox{90}{spinglass$^1$} &
	 \rotatebox{90}{statistical inference$^3$} \\ \midrule
	 
	 
	 
	 Edge weights & $\times$ & $\times$ & $\times$ & $\times$ & $\times$ & $\times$ & $\times$ & & $\times$ & $\times$ & $\times$ \\ \midrule
	 Runtime in sec & ~$3.6$ & ~$6.3$ & $11.7$ & ~$0.7$ & $19.4$ & $13.2$ & ~$0.2$ & $-$ & $-$ & $-$ & $-$ & $-$ \\ \midrule
	 Communities & $3$ & $2$ & $2$ & $3$ & $2$ & $1$ & $1$ & $-$ & $-$ & $-$ & $-$ & $-$ \\ \midrule
	  & 473 & 488 & 469 & 462 & 490 & 922 &  922 &  &  &  &  &  \\
	  & 434 & 434 & 453 & 427 & 431 &  &  &  &  &  &  &  \\
	  & 15 &  &  & 33 & (1) &  &  &  &  &  &  &  \\
	 \bottomrule
	 
\end{tabularx}
\begin{flushright}
\footnotesize{
$^1$ igraph, $^2$ NetworkX, $^3$ graph-tool\\
}
\end{flushright}

\end{table}

% \hdashline
% \midrule
% \bottomrule
\begin{table}[htbp]
\centering
\caption[X]{\textbf{X} X\\
}
\label{tab:algos4}

\begin{tabular}{lcccc}
\toprule
	 {} &
	 fastgreedy &
	 leading eigenvector &
	 multilevel &
	 walktrap \\ \midrule
	 
	  Network 1
	  & 473 & 488 & 462 & 490 \\
	  & 434 & 434 & 427 & 431 \\
	  & 15 &   & 33 & (1) \\ \midrule
	  Network 2
	  & 504 & 503 & 481 & 372 \\
	  & 467 & 475 & 439 & 311 \\
	  & 7 &   &  58 & 294 \\
	  & & & & (1) \\ \midrule
	  Network 3
	  & 534 & 537 & 505 & 310 \\
	  & 388 & 385 & 415 & 390 \\
	  &  &   &  (2) & 231 \\
	 \bottomrule
	 
\end{tabular}

\end{table}

% \hdashline
% \midrule
% \bottomrule

\section{Attributed Data and Hypothesis Testing}
Hypothesis\\
(1) Communities reflect groups of bees working in different areas of the hive and\\
(2) Communities reflect different age groups\\

The data which was used to test the hypothesis (1) is saved in a sqlite database for faster access, because using bb\_binary (parsing the data over and over again) was to slow. For testing if lists of positions (spatial ditribution) are different the test XY was used [TODO: what to use here]

For hypothesis (2) the data is stored as a csv file of birth dates of each bee. For testing if age goups are different the Kolmogorov Smirnov Test was used.