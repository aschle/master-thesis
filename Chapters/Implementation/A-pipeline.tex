%%%%%%%%%%%%%%%%%%%%%%%%%%%%%%%%%%%%%%%%%%%%%%%%%%%%%%%%%%%%%%%%%%%%%%%%%%%%%%%
%%%%%%%%%%%%%%%%%%%%%%%%%%%%%%%%%%%%%%%%%%%%%%%%%%%%%%%%%%%%%%%%%%%%%%%%%%%%%%%
\subsection{Implementing the Network Pipeline}
%%%%%%%%%%%%%%%%%%%%%%%%%%%%%%%%%%%%%%%%%%%%%%%%%%%%%%%%%%%%%%%%%%%%%%%%%%%%%%%
%%%%%%%%%%%%%%%%%%%%%%%%%%%%%%%%%%%%%%%%%%%%%%%%%%%%%%%%%%%%%%%%%%%%%%%%%%%%%%%

This section describes the pipeline for generating spatial proximity networks out of honey bee tracking data.
The input for the network pipeline are the data and a set of parameters previously described.
It outputs a network in graphML file format.
I parallelized the pipeline on the frame level.
That means each process gets a fraction of the data and extracts interactions for a time interval of five minutes. The main process accumulates the resulting interactions and creates a final network.

The pipeline consist of the following steps:

\begin{enumerate}
\item \textbf{Prefilter detections}\\
All detections below the chosen level of confidence are filtered out.

\item \textbf{Simple stitching}\\
For each camera positioned on the right hand side of the hive (camera 1 and 3), the $x$-coordinates of each detection is offset to the right ($2\times$maximum distance), to combine all detections per side into one coordinate system.

\item \textbf{Syncronize cameras}\\
The cameras do not capture all images syncronously and sometime take more or less than $3.5$ images per second.
Therefore, the cameras per side need to be syncronized. In the normal case the difference between consecutive frames should be about $0.332$~seconds, due to technical problems this value can be lower ($0.003$ ) and higher ($2.932$) at certain times. The frames of cameras 3 and 2 and cameras 1 and 0 are matched, frames without a match are discarded.

\item \textbf{Discard detections with certain IDs}\\
All bee detections with an ID contained in a valid ID list are keept, all other detections are discarded.

\item \textbf{Extract close pairs}\\
For each side of the hive, all close bee pairs according to the maximum distance parameter are calculated using an KDE-tree and then combined.

\item \textbf{Generate time series of bee pairs}\\
The data structure (frames and pairs of bees) is transformed to time series of bee pairs.

\item \textbf{Correct pair time series.}\\
The time series of bees are corrected by filling in the gaps of less than a certain length (gap size parameter).

\item \textbf{Extract interactions}\\
The edges and edge attributes (frequency and duration) are extracted from the time series of bees using the minimum contact duration parameter.
A sequence of ones with a length of at least the minimum contact duration are evaluated as an interaction.
The frequency of those series and the total duration (number of ones) are the attributes.
\end{enumerate}


I used python to implement the network pipeline because the library to access the tracking data is only available in python.
The data analysis is done with pandas~\footnote{\url{http://pandas.pydata.org/}} and numpy~\footnote{\url{http://www.numpy.org/}} using the web-based environment Jupyter Notebook\footnote{\url{http://jupyter.org/}}.
The networks are saved in graphML format and are created using the python library networkX\footnote{\url{https://networkx.github.io/} Last accessed: 2017-02-17, 08:07PM} version 1.11. iGraph\footnote{\url{http://igraph.org/python/}} was used for community detection and for network analysis.
