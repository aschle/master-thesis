%%%%%%%%%%%%%%%%%%%%%%%%%%%%%%%%%%%%%%%%%%%%%%%%%%%%%%%%%%%%%%%%%%%%%%%%%%%%%%%
%%%%%%%%%%%%%%%%%%%%%%%%%%%%%%%%%%%%%%%%%%%%%%%%%%%%%%%%%%%%%%%%%%%%%%%%%%%%%%%
\subsection{Defining the Network Pipeline}
%%%%%%%%%%%%%%%%%%%%%%%%%%%%%%%%%%%%%%%%%%%%%%%%%%%%%%%%%%%%%%%%%%%%%%%%%%%%%%%
%%%%%%%%%%%%%%%%%%%%%%%%%%%%%%%%%%%%%%%%%%%%%%%%%%%%%%%%%%%%%%%%%%%%%%%%%%%%%%%

This section describes the pipeline for generating spatial proximity networks out of honey bee tracking data. The network pipeline takes as input a path to the data  and a set of parameter described before and outputs a graph in graphML file format. The pipeline is parallelized on frame level, that means, each process gets a portion (frames for a timeinterval of five minutes) of the data and extracts interactions/edges. The main process accumulates everything and creates a network.\\

The pipeline consist of the following steps:

\begin{enumerate}
\item \textbf{Prefilter detections}\\
All detections below the chosen level of confidence level are filtered out.

\item \textbf{Simple stitching}\\
Each side of the hive consists of two cameras. 	The $x$-coordinates of each detection (of the right	cameras) is moved further to the right, also adding an offset of $2\times \texttt{maximum distance}$. So the left and the right detection of each side of the hive are move into one reference system.

\item \textbf{Syncronize Cameras}\\
For each side of the hive the cameras need to be syncronized. In the normal case the difference between consecutive frames should be about $0.332$~seconds, due to technical problem this value can be lower ($0.003$ ) and higher ($2.932$) at certain times. Cameras 3 and 2 and cameras 1 and 0 are matched, frames without a match are dropped (shorter number of frames, matchen, threshold $0.33/2$, minimum).

\item \textbf{Discard Detections with certain IDs}\\
All detections whos ID is in a list are keept, other detections are discarded.

\item \textbf{Extract close pairs}\\
For each side of the hive, all close pairs according to the maximum distance parameter are calculated and then joined together using a KDE-tree.

\item \textbf{Combine data of to sides of the hive}\\
Per frame the data gets combined.

\item \textbf{Generate time series of bee pairs}\\
The data structure (frames and detection) is transformed to time series of bee pairs.

\item \textbf{Correct pair time series.}\\
The time series of bees are corrected by filling in the gaps of length \texttt{gap size}.

\item \textbf{Extract interactions}\\
The edges and its attributes (frequency and duration) are extracted from the time series of bees using the minimum contact duration parameter. A sequence of at least X ones counts as one interaction. The frequency of those series adn the total duration (number of ones) are the attributes.

\end{enumerate}
