\subsubsection{Pipeline Parameters}
weighted temporal (time-aggregated) spatial proximity interaction networks out of tracking data\\
1) parameters for resulting networks: minimum contact duration, start-timestamp, window size in min, maximal distance\\
2) parameters resulted from dataset characteristics: confidence, file with valid ID lists, gap size\\

[TODO überarbeiten]\\

For performing the network analysis, I chose the pipeline parameters as follows:

\begin{description}
\item[Confidence] As explained in section\ref{subsec:confidence}, the confidence is set to $95\%$.

\item[Maximum Distance] I chose the length of a bee body, according to \textcite{baracchi2014socio}, as the maximum distance between two bees (figure~\ref{fig:radius}). The average bee length of $212$px ($\pm 16$px)  was determinded by manually measuring the length of all bees ($n=337$) in four images (one for each camera, 21.07.2016, 03:00PM) using the tool ImageJ\footnote{\url{http://imagej.net/Welcome}; Last accessed: 22.02.2016}.

\item[Gap Size] The gap size is set to two frames. This value corresponds to the median gap length in the time series of pairs ($\texttt{mode}=1$, $\texttt{mean}=27$). [TODO: what dataset was used (95\% confidence, XXX\% cutOff, XXXpx maximal distance, date, camera)]

\item[Minimum Contact Duration] This is set to three frames (one second). This corresponds to~\textcite{mersch2013tracking}, they as well exclude interactions below one second. Looking at the frequency distribution of chains of ones ($1$, $11$, $111$, and so on) of the pair time series (after filling the gaps), then: $\texttt{mode}=1$, $\texttt{median}=2$ and $\texttt{mean}=4$. Three frames corresponds to $57\%$ of all chains, this seem to be reasonable. [TODO: what dataset was used (95\% confidence, XXX\% cutOff, XXXpx maximal distance, date, camera)] [TODO add time for throphalaxis min and max]

\end{description}

\begin{figure}[htb]
	\centering
	\begin{subfigure}[b]{0.4\textwidth}
		\centering
		\includegraphics[width=\textwidth]{Figures/radius}
		\caption[Contact Radius]{Contact Radius}
		\label{fig:radius}
	\end{subfigure}
	\begin{subfigure}[b]{0.4\textwidth}
		\includegraphics[width=\textwidth]{Figures/sizeTagBee}
		\caption[Bee and Tag Size]{Bee Length}
		\label{fig:size}
	\end{subfigure}
	\caption{Distance Between Bees: A length of a bee is chosen as the maximal  distance between bees.}
	\label{fig:contactRadius}
\end{figure}