%%%%%%%%%%%%%%%%%%%%%%%%%%%%%%%%%%%%%%%%%%%%%%%%%%%%%%%%%%%%%%%%%%%%%%%%%%%%%%%
%%%%%%%%%%%%%%%%%%%%%%%%%%%%%%%%%%%%%%%%%%%%%%%%%%%%%%%%%%%%%%%%%%%%%%%%%%%%%%%
\subsection{Specifying the Network and its Parameters}
%%%%%%%%%%%%%%%%%%%%%%%%%%%%%%%%%%%%%%%%%%%%%%%%%%%%%%%%%%%%%%%%%%%%%%%%%%%%%%%
%%%%%%%%%%%%%%%%%%%%%%%%%%%%%%%%%%%%%%%%%%%%%%%%%%%%%%%%%%%%%%%%%%%%%%%%%%%%%%%
[TODO] decisions and study table\\
type of network: time-aggregated (time-aggregated, time-ordered)\\
type of interaction: spatial proximity network (spatial, contact, food)\\
directed or undirected\\
weighted/unweighted: type of weights: frequency and duration\\

A node in the network is a bee.
They are distinguished by IDs.
The network consists only of bees that interact.
Two bees are associated (spatially close to each other), if their distance is smaller then a \emph{maximum distance}.
Using only this criterion leads to a many interactions, because an interaction could only last for $0.33$ seconds
So an additional parameter the \emph{minimum contact duration} is introduced, it is the minimum time they have to spend close to each other in order to be called associated.

Edges are assigned two attributes. The first one is the frequency of contacts, how often they share a close position. The second parameter is the total duration of contact, how many time frames in total they spend close by.

%%%%%%%%%%%%%%%%%%%%%%%%%%%%%%%%%%%%%%%%%%%%%%%%%%%%%%%%%%%%%%%%%%%%%%%%%%%%%%%
\subsubsection{Pipeline Parameters}
%%%%%%%%%%%%%%%%%%%%%%%%%%%%%%%%%%%%%%%%%%%%%%%%%%%%%%%%%%%%%%%%%%%%%%%%%%%%%%%
The network pipeline takes two types of parameters: one for specifying the resulting network and how spacial proximity is defined and one relates to the data set.

\begin{description}
\item[Maximum distance] level of closeness between to individual bees~(in pixel)
\item[Minimum contact duration] the number of frames two individuals need to spend close by in order to count it as an interaction~(in frames)
\item [Start timestamp] starting point of the network aggregation~(as UTC string)
\item [Window size] size of time window for aggregating the network~(in minutes)

\vspace{5mm}

\item[Confidence] level of confidence, as described in section~\nameref{subsec:confidence}~(in percent)
\item[Valid IDs] list of valid ids within a specified time interval, as described in section~\nameref{subsubsec:dataset:filter}~(in csv file format)
\item[Gap Size] this is used to corect the time series of bee pairs~(in frames)
\item[Number of CPUs] number of used CPUs for parallelization
\item[Year] calculate bee IDs and stitching of camera images according to the observation period~(2015 or 2016)
\end{description}

%%%%%%%%%%%%%%%%%%%%%%%%%%%%%%%%%%%%%%%%%%%%%%%%%%%%%%%%%%%%%%%%%%%%%%%%%%%%%%%
\subsubsection{Chosen Parameter Values for Network Analysis}
%%%%%%%%%%%%%%%%%%%%%%%%%%%%%%%%%%%%%%%%%%%%%%%%%%%%%%%%%%%%%%%%%%%%%%%%%%%%%%%
\begin{table}[tbp]
\small
\centering
\caption[Parameters chosen for network analysis]{\textbf{Parameters chosen for network analysis} The maximum distance corresponds to the length of a bee body and the minimum contact duration is about one second. The networks are aggregated for ten hours.\\
}
\label{tab:chosenparams}

\begin{tabular}{rrl}
	\toprule
	\textbf{Parameter} & \textbf{Value} & \textbf{Unit} \\ \midrule
	Maximum distance & 212 & px \\
	Minimum contact duration & 3 & frames \\
	Window size & 600 & minutes \\ \midrule
	Confidence & 95 & percent \\
	Gap size & 2 & frames \\
	\bottomrule
\end{tabular}

\end{table}

The values are chosen according to biological constraints and similar to other studies, for better comparability.
I chose the length of a bee body, according to \textcite{baracchi2014socio}, as the maximum distance between two bees (figure~\ref{fig:contactRadius}). The average bee length of $212$px ($\pm 16$px)  was determinded by manually measuring the length of all bees ($n=337$) of four camera images using the tool ImageJ\footnote{\url{http://imagej.net/Welcome}; Last accessed:
 22.02.2016}.
The minimum contact duration is set to three frames (one second). This corresponds to~\textcite{mersch2013tracking}, they as also exclude interactions below one second.
To keep about 50\% of the data the confidence is set to $95\%$.
The gap size is set to two frames. This value corresponds to the median gap length in the time series of pairs.

\begin{figure}[htb]
	\centering
	\begin{subfigure}[b]{0.45\textwidth}
		\includegraphics[width=\textwidth]{Figures/sizeTagBee}
		\caption[Body length of a bee]{Body length of a bee}
		\label{fig:size}
	\end{subfigure}
	\hspace{0.08\textwidth}
	\begin{subfigure}[b]{0.45\textwidth}
		\centering
		\includegraphics[width=\textwidth]{Figures/radius}
		\caption[Contact radius]{Contact radius}
		\label{fig:radius}
	\end{subfigure}
	\caption{Distance Between Bees: A length of a bee is chosen as the maximal  distance between bees.}
	\label{fig:contactRadius}
\end{figure}