\clearpage
%%%%%%%%%%%%%%%%%%%%%%%%%%%%%%%%%%%%%%%%%%%%%%%%%%%%%%%%%%%%%%%%%%%%%%%%%%%%%%%
%%%%%%%%%%%%%%%%%%%%%%%%%%%%%%%%%%%%%%%%%%%%%%%%%%%%%%%%%%%%%%%%%%%%%%%%%%%%%%%
\subsection{Summary and Results}
%%%%%%%%%%%%%%%%%%%%%%%%%%%%%%%%%%%%%%%%%%%%%%%%%%%%%%%%%%%%%%%%%%%%%%%%%%%%%%%
%%%%%%%%%%%%%%%%%%%%%%%%%%%%%%%%%%%%%%%%%%%%%%%%%%%%%%%%%%%%%%%%%%%%%%%%%%%%%%%
The goal, as mentioned in~\ref{sec:intro:goals}, was to answer the question whether it is possible to infer temporal networks with the provided honey bee tracking data and to work out challenges and limitations regarding the provided data set and to identify the parameters necessary for the pipeline.

\subsubsection{Parameters}
%%%%%%%%%%%%%%%%%%%%%%%%%%%%%%%%%%%%%%%%%%%%%%%%%%%%%%%%%%%%%%%%%%%%%%%%%%%%%%%
This analysis results in two types of pipeline parameters. The first category specifies the resulting network, concerning the definition of spatial proximity, duration of interaction and size of the aggregated time window. The second type represents parameters resulting out of the characteristics of the dataset.

\begin{enumerate}
\item \textbf{Network parameters}\\
maximum distance\\
minimum contact duration\\
window size
\item \textbf{Data parameters}\\
confidence\\
list of valid IDs\\
gap size
\end{enumerate}


\subsubsection{Limitations}
%%%%%%%%%%%%%%%%%%%%%%%%%%%%%%%%%%%%%%%%%%%%%%%%%%%%%%%%%%%%%%%%%%%%%%%%%%%%%%%
It is possible to infer networks, but a complex preprocessing of the dataset is essential with two major steps:

\begin{enumerate}
\item \textbf{Reduction of data}\\
Reduce the amount of data to obtain a reliable data set, by filtering out detections with a low confidence value or by IDs with a low detection frequency.
\item \textbf{Combine camera data}\\
This step consist of the time synchronization of each of the two cameras and the joining of the data per frame.
\end{enumerate}

A tradeoff between the remaining amount of data that can be used for network inference and the data's quality had to be found. A high confidence value reduces the amount of data and produces gaps, whereas the gap size parameter tries to fix this problem. 

It is also possible to infer time-aggregated networks, but with restrictions.
When limiting the window size for network aggregation to the biological rhythms of day and night\footnote{Any other window size entails the inclusion of the duration of biological processes related to honey bees, I would need to know beforehand. Alternatively, I would need to apply a method to infer an appropriate window size out of the given data, this it out of scope. [TODO: ref paper]}, then due to a large number of interruptions, only a small amount of useful analysis days remain.