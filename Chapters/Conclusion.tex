%%%%%%%%%%%%%%%%%%%%%%%%%%%%%%%%%%%%%%%%%%%%%%%%%%%%%%%%%%%%%%%%%%%%%%%%%%%%%%%
%%%%%%%%%%%%%%%%%%%%%%%%%%%%%%%%%%%%%%%%%%%%%%%%%%%%%%%%%%%%%%%%%%%%%%%%%%%%%%%
%%%%%%%%%%%%%%%%%%%%%%%%%%%%%%%%%%%%%%%%%%%%%%%%%%%%%%%%%%%%%%%%%%%%%%%%%%%%%%%
%%%%%%%%%%%%%%%%%%%%%%%%%%%%%%%%%%%%%%%%%%%%%%%%%%%%%%%%%%%%%%%%%%%%%%%%%%%%%%%
\chapter{Conclusion}
\label{ch:conclusion}
%%%%%%%%%%%%%%%%%%%%%%%%%%%%%%%%%%%%%%%%%%%%%%%%%%%%%%%%%%%%%%%%%%%%%%%%%%%%%%%
%%%%%%%%%%%%%%%%%%%%%%%%%%%%%%%%%%%%%%%%%%%%%%%%%%%%%%%%%%%%%%%%%%%%%%%%%%%%%%%
%%%%%%%%%%%%%%%%%%%%%%%%%%%%%%%%%%%%%%%%%%%%%%%%%%%%%%%%%%%%%%%%%%%%%%%%%%%%%%%
%%%%%%%%%%%%%%%%%%%%%%%%%%%%%%%%%%%%%%%%%%%%%%%%%%%%%%%%%%%%%%%%%%%%%%%%%%%%%%%

The purpose of this thesis was to implement a pipeline for the extraction of time-aggregated networks using the provided high-resolution honey bee tracking data.
Moreover, the resulting weighted undirected spatial proximity networks of three consecutive time steps were analyzed regarding their network topology, community structures and the development of community members, to investigate the characteristics of honey bee colonies.

As opposed to most real world networks, these honey bee interaction networks are not scale-free networks and are characterized by a non-hierarchical topology and decentralized structure.
The small world characteristic of those networks allows for efficient communication within the bee colony.
The frequency a bee is observed inside the hive drops with increasing age.
That directly relates to the bees position in the colony network.
The detected communities relate to age-based functional groups with a spatial fidelity towards different regions of the comb. Those regions relate to the distinct type of cells and therefore to distinct tasks bees allocate.
Individual bees dynamically change functional groups as they age.

The non-hierarchical global network structure of the honey bee colony is stable over time, but its local structure is highly dynamic as individual bees change communities as they age. Those findings are aligned with previous research results and directly relate to the absence of a central authority and the decentralized organization of honey bee colonies shaped by temporal polyethism.

These network analysis results verify the definition of interaction networks by the initially chosen set of parameters and the functionality of the network pipeline in general.
The network pipeline is suitable, and it provides an excellent foundation for further investigations.

%%%%%%%%%%%%%%%%%%%%%%%%%%%%%%%%%%%%%%%%%%%%%%%%%%%%%%%%%%%%%%%%%%%%%%%%%%%%%%%
%%%%%%%%%%%%%%%%%%%%%%%%%%%%%%%%%%%%%%%%%%%%%%%%%%%%%%%%%%%%%%%%%%%%%%%%%%%%%%%
\section{Limitations}
%%%%%%%%%%%%%%%%%%%%%%%%%%%%%%%%%%%%%%%%%%%%%%%%%%%%%%%%%%%%%%%%%%%%%%%%%%%%%%%
%%%%%%%%%%%%%%%%%%%%%%%%%%%%%%%%%%%%%%%%%%%%%%%%%%%%%%%%%%%%%%%%%%%%%%%%%%%%%%%
The following section outlines limitations concerning the accuracy and quality of the resulting networks and describes restrictions emerging from a high density.

Despite complex preprocessing the quality of the extracted networks could be higher. Although I filtered out erroneous detections before network generation, a few number of individuals remain in the extracted networks, that should not exist according to the studies tagging and hatching documentation. Besides, part of the network are bees, which die at some point during the aggregation period.\\
The prefiltering of detections and the synchronization of camera frames reduces the amount of data available for the extraction of interactions. The gap size parameter tries to fix the resulting gaps but does not solve the problem perfectly. I suppose that some interactions are shorter than they are in reality, which distorts the networks a bit. As more accurate tracking data will replace the data in the future, this was an excellent choice with an appropriate amount of effort.

Spatial proximity is an indicator for interaction but does not relate to actual interactions. The definition of spatial proximity by a maximal distance and a minimum contact duration is very loose and especially in honey bee colonies, where space is limited, leads to many edges and consequently to a high density of the network. The high noise weakens or blurred the real interactions between individuals.
My choice of aggregating the networks for ten hours supports this nosiness, resulting in a global state of the colony, rather than capturing finer granular dynamics.\\
In this context, the network property strength is the only measure I used, which profits by the aggregation, all other measures are less meaningful.
Due to the high density and size of the network, the methods I can apply concerning community detection are limited.
The selection of an algorithm for detecting communities is limited to algorithms finding only non-overlapping structures.

%%%%%%%%%%%%%%%%%%%%%%%%%%%%%%%%%%%%%%%%%%%%%%%%%%%%%%%%%%%%%%%%%%%%%%%%%%%%%%%
%%%%%%%%%%%%%%%%%%%%%%%%%%%%%%%%%%%%%%%%%%%%%%%%%%%%%%%%%%%%%%%%%%%%%%%%%%%%%%%
\section{Recommendations}
%%%%%%%%%%%%%%%%%%%%%%%%%%%%%%%%%%%%%%%%%%%%%%%%%%%%%%%%%%%%%%%%%%%%%%%%%%%%%%%
%%%%%%%%%%%%%%%%%%%%%%%%%%%%%%%%%%%%%%%%%%%%%%%%%%%%%%%%%%%%%%%%%%%%%%%%%%%%%%%
The following sections list recommendations for change regarding the applied methods. The focus lies on methods to reduce the noise within the network.


\paragraph{More Dynamic and Temporal Analysis}
%%%%%%%%%%%%%%%%%%%%%%%%%%%%%%%%%%%%%%%%%%%%%%%%%%%%%%%%%%%%%%%%%%%%%%%%%%%%%%%
By lowering the window size of the aggregated network and by investigating different granularities could allow for a more dynamic analysis of the networks. Instead of using time-aggregated networks, one could shift towards the use of time-ordered networks by using time-stamped interactions.

\paragraph{Focus on Important Interactions}
%%%%%%%%%%%%%%%%%%%%%%%%%%%%%%%%%%%%%%%%%%%%%%%%%%%%%%%%%%%%%%%%%%%%%%%%%%%%%%%
For reducing the edges to only meaningful interactions, I see three main approaches.
The space on the honeycomb is limited and crowded. Therefore fine tuning of pipeline parameters by reducing the size of the maximum distance and by increasing the number of frames for minimum contact duration is an option.
Instead of keeping the definition of spatial proximity I would recommend extracting contact events (e.g. by including an angle, so bees facing each other) or trophallaxis events for defining the edges, especially when using those networks to investigating more specific biological research questions.
Moreover, a simple global threshold for excluding edges below a certain value could be used.

\paragraph{Use the Potential of Weighted Edges}
%%%%%%%%%%%%%%%%%%%%%%%%%%%%%%%%%%%%%%%%%%%%%%%%%%%%%%%%%%%%%%%%%%%%%%%%%%%%%%%
Instead of applying a global threshold, a specific network reduction algorithms can extract the backbone structure of the network. The disparity measure characterizes the level of local heterogeneity of edges~\cite{barthelemy2003spatial}. \textcite{serrano2009extracting} uses this definition to proposes a disparity filter algorithm which seems promising but needs further investigation. 
For all network measures utilized in this work, weighted versions exist. The already implemented versions of weighted measures (e.g., closeness and betweenness) in iGraph and networkX do favor edge weights over the number of links and simply apply Dijkstra for calculating the shortest paths. \textcite{opsahl2010node} proposes weighted network measures by providing a generalized degree and shortest path algorithm. A tuning parameter has to be chosen, which defines whether to emphasize the number of links or weights of edges. This parameter has to be chosen according to the given research question.

\paragraph{Normalizing by the Detection Frequency of Individuals}
%%%%%%%%%%%%%%%%%%%%%%%%%%%%%%%%%%%%%%%%%%%%%%%%%%%%%%%%%%%%%%%%%%%%%%%%%%%%%%%
Depending on the topics of further research a normalization of the networks regarding the detection rate of individuals could be purposeful. 
I see two options, either by normalizing the edge weight by e.g. applying the simple ratio index (SRI); or by normalizing the particular node level measure by taking the detection frequency of that focal individual into account.

\paragraph{Random Geometric Graph}
%%%%%%%%%%%%%%%%%%%%%%%%%%%%%%%%%%%%%%%%%%%%%%%%%%%%%%%%%%%%%%%%%%%%%%%%%%%%%%%
Instead of comparing the network to an Erdös-Reniy random graph model a new model could be proposed.
Each frame can be modeled by a random geometric graph~\cite{rgg2002}, but with placing the nodes in each step not completely randomly, but by modeling the bee as a random walker. The direction of movement could be chosen randomly, but the distance of a step could be selected according to the average speed of bees. 


\section{Outlook}
I investigated the effects the pipeline parameters maximum distance, minimum contact duration and window size have on network properties like the number of nodes and edges, degree distribution and global clustering coefficient. I tested this for a few combination of values and but only for window sizes up to one hour; this could be investigated more systematically.
Similarly, the robustness of the detected communities regarding the pipeline parameters (maximum distance, minimum contact duration, and window size)  could be studied systematically. I only tested the robustness for some values but focused more on different algorithms.
The provided dataset facilitates the investigation of seasonal change in honey bee colonies using network analysis methods. Long-term dynamics offer a high potential for further studies.
It would also be interesting to compare my network analysis results of domesticated honey bees, with an artificial birth rate to the social networks of wild honey bees.

\section{Closing Remarks}
[TODO: redo or just delete]
In the first step, the automatic tracking of animals leads to more data, which covers a longer observation period and with a higher sampling resolution. The simultaneous observation of not only more individuals but also several colonies under distinct conditions becomes possible in an efficient way.
The availability of more data, which is not limited towards a specific study purpose, opens the space to investigate the data in an explorative way and to discover the unexpected.
The prerequisite for studying non-human animal data in an explorative way that fosters the framing of a novel biological hypothesis is either a personal, profound domain knowledge or the constant support of experts of that species.
A lot of valuable information beneficial for data analysis is gained during the process of manual data collection by observing the animals directly. An automatic observation process veils this part and therefore increases the abstraction level and encourages alienation between the researchers and observed animals.

Applying network analysis methods to novel datasets carries the risk of either simply describing animal network structures or leading to the restating of well-known facts. Framing important biological questions that benefit from network science methods or the development of new techniques in the field of network analysis with the help of this unique dataset should be the overall goal, but requires profound knowledge in both areas of research.