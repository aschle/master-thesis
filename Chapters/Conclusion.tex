%%%%%%%%%%%%%%%%%%%%%%%%%%%%%%%%%%%%%%%%%%%%%%%%%%%%%%%%%%%%%%%%%%%%%%%%%%%%%%%
%%%%%%%%%%%%%%%%%%%%%%%%%%%%%%%%%%%%%%%%%%%%%%%%%%%%%%%%%%%%%%%%%%%%%%%%%%%%%%%
%%%%%%%%%%%%%%%%%%%%%%%%%%%%%%%%%%%%%%%%%%%%%%%%%%%%%%%%%%%%%%%%%%%%%%%%%%%%%%%
%%%%%%%%%%%%%%%%%%%%%%%%%%%%%%%%%%%%%%%%%%%%%%%%%%%%%%%%%%%%%%%%%%%%%%%%%%%%%%%
\chapter{Conclusion}
\label{ch:conclusion}
%%%%%%%%%%%%%%%%%%%%%%%%%%%%%%%%%%%%%%%%%%%%%%%%%%%%%%%%%%%%%%%%%%%%%%%%%%%%%%%
%%%%%%%%%%%%%%%%%%%%%%%%%%%%%%%%%%%%%%%%%%%%%%%%%%%%%%%%%%%%%%%%%%%%%%%%%%%%%%%
%%%%%%%%%%%%%%%%%%%%%%%%%%%%%%%%%%%%%%%%%%%%%%%%%%%%%%%%%%%%%%%%%%%%%%%%%%%%%%%
%%%%%%%%%%%%%%%%%%%%%%%%%%%%%%%%%%%%%%%%%%%%%%%%%%%%%%%%%%%%%%%%%%%%%%%%%%%%%%%

The purpose of this thesis was to implement a pipeline for the extraction of time-aggregated networks using the provided high-resolution honey bee tracking data.
Moreover, the resulting weighted undirected spatial proximity networks of three consecutive time steps were analyzed regarding their network topology, community structures and the development of community members, to investigate the characteristics of honey bee colonies.

As opposed to most real world networks, these honey bee interaction networks are not scale-free networks and are characterized by a non-hierarchical topology and decentralized structure.
The small world characteristic of those networks allows for efficient communication within the bee colony.
The frequency a bee is observed inside the hive drops with increasing age.
That directly relates to the bees position in the colony network.
The detected communities relate to age-based functional groups with a spatial fidelity towards different regions of the comb. Those regions relate to the distinct type of cells and therefore to distinct tasks bees allocate.
Individual bees dynamically change functional groups as they age.

The non-hierarchical global network structure of the honey bee colony is stable over time, but its local structure is highly dynamic as individual bees change communities as they age. Those findings are aligned with previous research results and directly relate to the absence of a central authority and the decentralized organization of honey bee colonies shaped by temporal polyethism.

These network analysis results verify the definition of interaction networks by the initially chosen set of parameters and the functionality of the network pipeline in general.
The network pipeline is suitable, and it provides an excellent foundation for further investigations.

\section{Limitations}
(1) limitations by the dataset and implementation of the pipeline\\
(2) decisions taken concerning the type of spatial proximity networks\\
(3) regarding methods for network analysis and temporal aspects\\

(1) dataset and implementation\\
(a) low quality requires complex steps during preprocessing\\
filter out detections with a low confidence level and using a list of valid IDs per day\\
but still some individuals remain in the extracted networks, that should not exist\\
also bees who are going to die on that day remain in the network\\
(b) data reduction by prefiltering of detections and is also caused by the syncing of cameras (frames are missing)\\
try to solve this problem by the parameter fill gap size is not a perfect solution, because to 100 accurate, compromize, but was appropriate for first steps towards using network analysis methods\\
but data will be exchanged by tracking data in the future, least effort\\

(2) network type and temporal spects\\
(a) the definition of spatial proximity by only a maximal distance and a minimum contact duration is very loose,
generally spatial proximity is just an indicator for interaction, but does not relate to real interactions, but addds much noise to the networks\\
especially in hooney bee colonies, where space is limited anyway, this definition could be too vague\\
using contact events (e.g. by including an angle) or food-exchange events for defining edges, could be more appropritae for more specialized biological research questions\\
high density could also be an effect of an large maximum distance parameter or a too long chosen value for the minimum contact duration to long, too much noise\\

(3) analysis methods and temporal spects\\
aggregating the data for a large time window, results in a noisy network with a global state of the colony, rather than capturing more fine granular dynamics, as opposed to time-ordered networks with time stamped interactions\\
in this context, strength is the only measure I used, which holds more specific information, because its cumulating interaction informaion\\
The selection of an algorithm for detecting communities is limited to algorithms finding non-overlapping structures due to the high density and size of the networks.

\section{Future Work}
(1) recommendations for further studies, (2) recommendation for change\\
each recommendation should directly trace a direct conclusion\\

investigate: correlation of density and time, upper bound? or 100\% density after some X frames?\\
are communities robust regarding pipeline parameters maximum distance, minimum contact duration, window size (yes they are, tested for some values, but not systhematically)\\
pipelie parameters (maximum distance, minimum contact duration, window size) effects on network properties (degree, strength,lcc, edgeweights) ... tested but not systematically only some values and only window size up to one hour, should be done more systematically\\
compare the networks to random geometric graph~\cite{rgg2002} and model a walking bee as a random walker\\
so far nodes are attributed with only age and detection frequency, add other stuff, e.g. average speed, total distance traveled\\
compute centrality measures per community\\
weighted edges, should also inpect weighted versions of node measures (this was a bit to complicated, already implemented weighted version in networkX and igraoh do favour weights over number of hops, should look at~\textcite{opsahl2010node}, but depending on what you want to investigate you have to choose the $\alpha$)\\
compare analysis results to NON artifical bees networks\\
investigate different granularities (size of time window)\\
study longer time period not only 5 days\\
normalized edge weight (SRI - Simple Ratio Index, SRI measures
the proportion of times two individuals were seen
together out of the total number of times those individ-
uals were observed, by \textcite{croft2008exploring}) or normalized degree and stuff by frequency of detection\\
apply network reduction algorithm to reduce density and noise, by applying (no simple edge thresholding), disparity filter~\textcite{serrano2009extracting}

\vspace{2cm}
more data does not mean better analysis, one have to ask good questions, develop specialized hypothesis, a lot of options\\
explorative analysis approach difficult with data from animals I do not know, maybe easier with 'human' data (missing domain data), context, limited domain knowledge, working in an interdisziplinary teams promising\\
alienation: automatic tracking -> no relation to the data or how the data was collected or animals\\
just applying network method to new research fields, sometimes just replicates known facts, Ending with: Zitat Krause (Buch)