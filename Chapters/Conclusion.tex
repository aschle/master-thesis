%%%%%%%%%%%%%%%%%%%%%%%%%%%%%%%%%%%%%%%%%%%%%%%%%%%%%%%%%%%%%%%%%%%%%%%%%%%%%%%
%%%%%%%%%%%%%%%%%%%%%%%%%%%%%%%%%%%%%%%%%%%%%%%%%%%%%%%%%%%%%%%%%%%%%%%%%%%%%%%
%%%%%%%%%%%%%%%%%%%%%%%%%%%%%%%%%%%%%%%%%%%%%%%%%%%%%%%%%%%%%%%%%%%%%%%%%%%%%%%
%%%%%%%%%%%%%%%%%%%%%%%%%%%%%%%%%%%%%%%%%%%%%%%%%%%%%%%%%%%%%%%%%%%%%%%%%%%%%%%
\chapter{Conclusion}
\label{ch:conclusion}
%%%%%%%%%%%%%%%%%%%%%%%%%%%%%%%%%%%%%%%%%%%%%%%%%%%%%%%%%%%%%%%%%%%%%%%%%%%%%%%
%%%%%%%%%%%%%%%%%%%%%%%%%%%%%%%%%%%%%%%%%%%%%%%%%%%%%%%%%%%%%%%%%%%%%%%%%%%%%%%
%%%%%%%%%%%%%%%%%%%%%%%%%%%%%%%%%%%%%%%%%%%%%%%%%%%%%%%%%%%%%%%%%%%%%%%%%%%%%%%
%%%%%%%%%%%%%%%%%%%%%%%%%%%%%%%%%%%%%%%%%%%%%%%%%%%%%%%%%%%%%%%%%%%%%%%%%%%%%%%

The purpose of this thesis was to implement a pipeline for the extraction of time-aggregated networks using the provided high-resolution honey bee tracking data.
Moreover, the resulting weighted undirected spatial proximity networks of three consecutive time steps were analyzed regarding their network topology, community structures and the development of community members, to investigate the characteristics of honey bee colonies.

As opposed to most real world networks, these honey bee interaction networks are not scale-free networks and are characterized by a non-hierarchical topology and decentralized structure.
The small world characteristic of those networks allows for efficient communication within the bee colony.
The frequency a bee is observed inside the hive drops with increasing age.
That directly relates to the bees position in the colony network.
The detected communities relate to age-based functional groups with a spatial fidelity towards different regions of the comb. Those regions relate to the distinct type of cells and therefore to distinct tasks bees allocate.
Individual bees dynamically change functional groups as they age.

The non-hierarchical global network structure of the honey bee colony is stable over time, but its local structure is highly dynamic as individual bees change communities as they age. Those findings are aligned with previous research results and directly relate to the absence of a central authority and the decentralized organization of honey bee colonies shaped by temporal polyethism.

These network analysis results verify the definition of interaction networks by the initially chosen set of parameters and the functionality of the network pipeline in general.
The network pipeline is suitable, and it provides an excellent foundation for further investigations.

%%%%%%%%%%%%%%%%%%%%%%%%%%%%%%%%%%%%%%%%%%%%%%%%%%%%%%%%%%%%%%%%%%%%%%%%%%%%%%%
%%%%%%%%%%%%%%%%%%%%%%%%%%%%%%%%%%%%%%%%%%%%%%%%%%%%%%%%%%%%%%%%%%%%%%%%%%%%%%%
\section{Limitations}
%%%%%%%%%%%%%%%%%%%%%%%%%%%%%%%%%%%%%%%%%%%%%%%%%%%%%%%%%%%%%%%%%%%%%%%%%%%%%%%
%%%%%%%%%%%%%%%%%%%%%%%%%%%%%%%%%%%%%%%%%%%%%%%%%%%%%%%%%%%%%%%%%%%%%%%%%%%%%%%
(1) limitations by the dataset and implementation of the pipeline\\
(2) decisions taken concerning the type of spatial proximity networks\\
(3) regarding methods for network analysis and temporal aspects\\

(1) dataset and implementation\\
(a) low quality requires complex steps during preprocessing\\
filter out detections with a low confidence level and using a list of valid IDs per day\\
but still some individuals remain in the extracted networks, that should not exist\\
also bees who are going to die on that day remain in the network\\
(b) data reduction by prefiltering of detections and is also caused by the syncing of cameras (frames are missing)\\
try to solve this problem by the parameter fill gap size is not a perfect solution, because to 100 accurate, compromize, but was appropriate for first steps towards using network analysis methods\\
but data will be exchanged by tracking data in the future, least effort\\

(2) network type and temporal spects\\
(a) the definition of spatial proximity by only a maximal distance and a minimum contact duration is very loose,
generally spatial proximity is just an indicator for interaction, but does not relate to real interactions, but addds much noise to the networks\\
especially in hooney bee colonies, where space is limited anyway, this definition could be too vague\\
using contact events (e.g. by including an angle) or food-exchange events for defining edges, could be more appropritae for more specialized biological research questions\\
high density could also be an effect of an large maximum distance parameter or a too long chosen value for the minimum contact duration to long, too much noise\\

(3) analysis methods and temporal spects\\
aggregating the data for a large time window, results in a noisy network with a global state of the colony, rather than capturing more fine granular dynamics, as opposed to time-ordered networks with time stamped interactions\\
in this context, strength is the only measure I used, which holds more specific information, because its cumulating interaction informaion\\
The selection of an algorithm for detecting communities is limited to algorithms finding non-overlapping structures due to the high density and size of the networks.

%%%%%%%%%%%%%%%%%%%%%%%%%%%%%%%%%%%%%%%%%%%%%%%%%%%%%%%%%%%%%%%%%%%%%%%%%%%%%%%
%%%%%%%%%%%%%%%%%%%%%%%%%%%%%%%%%%%%%%%%%%%%%%%%%%%%%%%%%%%%%%%%%%%%%%%%%%%%%%%
\section{Future Work}
%%%%%%%%%%%%%%%%%%%%%%%%%%%%%%%%%%%%%%%%%%%%%%%%%%%%%%%%%%%%%%%%%%%%%%%%%%%%%%%
%%%%%%%%%%%%%%%%%%%%%%%%%%%%%%%%%%%%%%%%%%%%%%%%%%%%%%%%%%%%%%%%%%%%%%%%%%%%%%%
(A) recommendation for change\\
(B) recommendations for further studies\\
each recommendation should directly trace a direct conclusion\\


(A) things to change and do differently (methods)\\

compare the networks to a random geometric graph~\cite{rgg2002} and model a walking bee as a random walker\\
this could be a better model, than just comparing it to the simple Erdös-Reniy random graph\\

exploit the potential of weighted edges (frequency and duration) by:\\

(1) apply a network reduction algorithm to reduce density and noise, by applying (no simple edge thresholding), the proposed disparity filter by \textcite{serrano2009extracting}\\
the disparity measure characterizes the level of local heterogeneity~\cite{barthelemy2003spatial}

(2) inpecting the weighted versions of node measures\\
the already implemented version of weighted measures (e.g closeness and betweenness) in iGraph and networkX do favour edge weights over the number of hops, by applying Dijkstra for calculating the shortes paths. \textcite{opsahl2010node} proposes weighted network measures by providing a generalized degree and shortest path algorithm. A tuning parameter has to be set according to a reseach setting and data is defines weather to emphasis the importantce of number of links or weights more.

(3) depending on the topic of research a normalization of the networks regarding the detection rate of individuals could be purposeful\\
eighter by normalizing the edge weight by e.g. applying the simple ratio index (SRI)~\footnote{Dividing the edge weight by the proportion of times two individuals were seen
together out of the total number of times those individuals were observed.} 
or by normalizing the particular node level measure by taking the detection frequency of that focal individual into account\\


(B) things to investigate further\\

correlation of the networks density and size of time window for aggregation, is there an upper bound of density or does it reach 100\%

are the detected communities robust regarding the pipeline parameters maximum distance, minimum contact duration, and window size (yes they are, I tested this for some values, also with different algorithms), should be studied more systhematically\\

pipelie parameters (maximum distance, minimum contact duration, window size) and its effects on network properties (degree, strength, lcc, edgeweights, centrality measures), I tested this, but not systematically, only for a few combination of values and only window sizes up to one hour, should be done more systematically\\

so far the nodes in the networks are only attributed with the age and detection frequency of individuals, add other stuff, e.g. average speed of a bee and its total distance traveled, could be added for further analysis\\

compute centrality measures for each community, to see if they are different regarding those measures\\

compare analysis results to NON artifical bees networks\\

investigate different granularities (size of time window), for more dynamic analysis\\

long term studies, to investigate the seasonal change in honey bee colonies\\


\subsection{Closing Remarks}
In the first step, the automatic tracking of animals leads to more data, which covers a longer observation period and with a higher sampling resolution. The simultaneous observation of not only more individuals but also several colonies under distinct conditions becomes possible in an efficient way.
The availability of more data, which is not limited towards a specific study purpose, opens the space to investigate the data in an explorative way and to discover the unexpected.
The prerequisite for studying non-human animal data in an explorative way that fosters the framing of a novel biological hypothesis is either a personal, profound domain knowledge or the constant support of experts of that species.
A lot of valuable information beneficial for data analysis is gained during the process of manual data collection by observing the animals directly. An automatic observation process veils this part and therefore increases the abstraction level and encourages alienation between the researchers and observed animals.

Applying network analysis methods to novel datasets carries the risk of either simply describing animal network structures or leading to the restating of well-known facts. Framing important biological questions that benefit from network science methods or the development of new techniques in the field of network analysis with the help of this unique dataset should be the overall goal, but requires profound knowledge in both areas of research.