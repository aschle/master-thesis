\chapter{Conclusion and Future Work}
\label{ch:conclusion}

Summary based on 4 goals:\\
small world network (short distances, efficient communication between bees)\\
non-scale free networks (non hierarchical structure, no central bees)\\
correlation: detection frequency, age, network measures and kind of bimodal distributions, hint for functional groups\\
age based groups, spatial fidality, according to temporal polythingy\\
move to older functional groups as they age\\

1. - stable global structure of the colony over time\\
- stable network properties: distribution of network measures (degree, strength, lcc, $C_B$ and $C_C$ stays the same over the three snapshots\\
- communities stay the same over the snapshots (age groups, spatial groups)
2. - dynamic local structure (node level, individual bee): bees change communities as they age\\
->\\
as expected from honey bee colonies, confirms research results of others\\
verifies my definition of the networks and the network pipeline\\
and chosen network parameters (maximal distance and minimum contact duration)\\
therefore analysis results validate the first goal: inferrence of time-aggregated networks seems to be ok\\

network pipeline can be used for further research, maybe with some modifications, but depends on what one wants to investigate\\

more data does not mean better analysis\\
explorative analysis approach difficult with data from animals I do not know, maybe easier with 'human' data\\
difficult to frame a hypothesis\\
alienation: automatic tracking -> no relation to the data or animals\\

\section{Limitations}
regarding my methods, concerning implementation\\
dataset: quality, kind of a bit complex preprocessing (syncing cameras, removing invalid data, confidence)\\
definition of interaction - minimum contact duration maybe was not the best idea\\
type of network spatial proximity -> maybe better contact/food-exchange network (interaction events), too much noise in the data\\
context, working in an interdisziplinary teams, would easier to understand results and more efficient\\
age distribution has some gaps -> not sure about it effects\\

\section{Future Work}
(1) recommendations for further studies, (2) recommendation for change\\
each recommendation should directly trace a direct conclusion\\

investigate: correlation of density and time, upper bound? or 100\% density after some X frames?\\
are communities robust regarding pipeline parameters maximum distance, minimum contact duration, window size (yes they are, tested for some values, but not systhematically)\\
pipelie parameters (maximum distance, minimum contact duration, window size) effects on network properties (degree, strength,lcc, edgeweights) ... tested but not systematically only some values and only window size up to one hour, should be done more systematically\\
compare the networks to random geometric graph~\cite{rgg2002} and model a walking bee as a random walker\\
so far nodes are attributed with only age and detection frequency, add other stuff, e.g. average speed, total distance traveled\\
compute centrality measures per community\\
weighted edges, should also inpect weighted versions of node measures (this was a bit to complicated, already implemented weighted version in networkX and igraoh do favour weights over number of hops, should look at~\textcite{opsahl2010node}, but depending on what you want to investigate you have to choose the $\alpha$)\\
compare analysis results to NON artifical bees networks\\
investigate different granularities (size of time window)\\
study longer time period not only 5 days\\
normalized edge weight (SRI - Simple Ratio Index, SRI measures
the proportion of times two individuals were seen
together out of the total number of times those individ-
uals were observed, by \textcite{croft2008exploring}) or normalized degree and stuff by frequency of detection\\
apply network reduction algorithm to reduce density and noise, by applying (no simple edge thresholding), disparity filter~\textcite{serrano2009extracting}
